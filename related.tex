\section{Related work}
\label{sec:related}

There is a surge of interest in open and portable languages that raise the level of abstraction for robotics~\cite{Buzzlanguage,Bohrer:2018:VVC:3192366.3192406,reactlang,williams2003model} (see the survey of robotics Domain Specific Languages (DSLs) for more details~\cite{Nordmann2014}). Most of these languages are proprietary or generate executables that are tied to specific platforms.
%
\begin{table}[!ht]
    \footnotesize
    \centering
    \begin{tabular}{|l| c @{\hspace{0.5mm}} c @{\hspace{1mm}}c c  c @{\hspace{0.5mm}} c|}
        \hline
            & \tb{Dist.} & \tb{Hetero-} & \tb{Sim}   & \tb{Prog.}         & \tb{Comp.} & \tb{V\&V}  \\
        \tb{Name}                             & \tb{Sys.}  & \tb{geneous} &            & \tb{Lang.}         &            &            \\ \hline
        ROSBuzz~\cite{ROSBuzz}               & \checkmark & \checkmark   & \checkmark & Buzz               & \checkmark &            \\
        PythonRobotics                      &            & \checkmark   & \checkmark & Python             &            &            \\
        PyRobot~\cite{pyrobot2019}          &            & \checkmark   & \checkmark & Python             &            &            \\
        MRPT~\cite{MRPT}                     &            & \checkmark   &            & C++                &            &            \\
        Robotarium~\cite{robotarium}          &            & \checkmark   & \checkmark & Matlab             &            &            \\
        Drona~\cite{desai2017drona}           & \checkmark &              & \checkmark & P~\cite{Planguage} & \checkmark & \checkmark \\
        Live~\cite{campusanofabry:lrp2016}    &            & \checkmark   &            & LPR                & \checkmark &            \\
        \lgname                             & \checkmark & \checkmark   & \checkmark & \lgname            & \checkmark & \checkmark \\ \hline
    \end{tabular}
            \caption{}
        \label{tab:summary}
\end{table}

Here we focus on platform-agnostic languages and systems. {\em Robot Operating System (ROS)\/}~\cite{ros} is the predominant member in this category. At its core, ROS supports a publish-subscribe-based communication system and the ROS community has built libraries for  many  different sensors and platforms.
Our implementation of the $\lgname$ abstractions for the quadcopter and vehicle platforms use ROS just like thousands of other robotics products and  projects.
 
 Table~\ref{tab:summary} gives a summary of robotics languages and libraries that have been deployed on hardware platforms.
%\sayan{Add sentences about Drona, ROSBuzz, etc. Where is PyRobot etc in the table?}
 ROSBuzz~\cite{ROSBuzz} supports the Buzz language, which doesn't provide abstractions like $\lgname$ for path planning, de-conflicting, and shared variables. The Live Robot Programming language~\cite{campusanofabry:lrp2016} not only provides a higher-level programming abstraction in terms of nested state machines, but also allows the program to be changed while running, hence reducing the feedback loop across writing, compiling, and testing of robot programs. Programming systems using the shared memory paradigm have been developed for several distributed computing systems~\cite{dsm1991,Adve96sharedmemory,Azure,Cassandra,Dynamo}. P~\cite{Planguage}  and PSync~\cite{PSyncLanguage} are DSLs for asynchronous partially distributed systems, but cyberphysical interactions are not supported. P has been integrated into DRONA, a framework for building reliable distributed mobile robotics applications, which focuses specifically on the task and the motion planning aspects of these applications, as opposed to $\lgname$'s model for various types of sensors and actuators, possibly related to perception, and computer vision available to a robotics system.


The $\lgname$ approach towards programming and analysis of CyberPhysical systems by interleaving discrete and continuous system behaviors is a hybrid system model that has been explored in various other modeling languages. Zelus~\cite{zelus} is a modeling language for hybrid systems using a synchronous language compiler to handle the discrete logical time transitions and an ODE solver to handle its continuous time evolution transitions. BIP~\cite{bip} is a modeling and analysis framework for heterogeneous real-time components of system, including interactions between the components.
%

\rg{In a 4-page workshop abstract, Ghosh et al. discussed some of the ideas underlying the abstractions embodied in $\lgname$~\cite{applied}. The implementation and deployment of the Task application using $\lgname$ on real hardware platforms is discussed in~\cite{arxiv}. That technical report does not discuss language design, semantics, nor the V\&V approaches.}.

