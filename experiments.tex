\newcommand{\Koord}{\ensuremath{\lgname}\xspace}
\newcommand{\CyPhyHouse}{CyPhyHouse\xspace}
\newcommand{\Gazebo}{Gazebo\xspace}

\newcommand{\ScanToMap}{\ensuremath{\mathit{scanToMap}}\xspace}
\newcommand{\TSync}{\ensuremath{\mathit{tSync}}\xspace}
\newcommand{\PathToFrontier}{\ensuremath{\mathit{pickPathToFrontier}}\xspace}

\section{Experiments}
\label{sec:experiments}

\chiao{1\textasciitilde1.5 pages. Include figures.}

For our case study, we test the \dmap application using the simulator from \CyPhyHouse~\cite{ghosh2019cyphyhouse} for distributed \Koord applications.
We choose the MIT RACECAR model~\cite{MIT_RACECAR} with Lidar sensor included in \CyPhyHouse for our experiment.
To implement external functions defined in~\ref{sec:analysis},
we extend the \CyPhyHouse tool chain and examine our assumptions over sensor data and implemented functions.
In the section, we first discuss our implementation of external functions;
then we present the simulation result of \dmap with multiple vehicles.

\subsection{Simulation Result}

We design three different worlds with the same area of $20\times20$ square meters,
and each can be quantized as a $20 \times 20$ map.
To compare across different numbers of vehicles more easily,
we ensure that all unoccupied rectangles are connected in each map,
so every car can explore the whole map.
Each car is equipped with a Lidar providing each scan with radius 2.5m and angles from $-\pi$ to $\pi$.
For simplicity, we configure the Lidars so that they do not detect other cars as obstacles.
The maximum speed of all cars is set to 3m/s.
For each map, we run three simulations with 2, 3, or 4 cars for 20 minutes of simulation time.
Our simulation is run on a computer with 8 Intel Core i7-6700HQ @ 2.60GHz CPU cores, 16 GB main memory, and a GeForce GTX 970M GPU.

Figure~\ref{fig:percent} shows that more cars explore more percentages in any given amount of time in all maps.
It is not surprising because more cars start with more areas explored initially and explore concurrently.
A more interesting result is that the final saturating percentages by more cars are higher as well.
From our observation in the visualization, one major reason is due to the aforementioned limitations in finding paths to frontier.
When there are only few unexplored grid rectangles remaining,
it becomes much more difficult to find such paths.
The chance to successfully find such path with more cars is simply higher.
The U-shaped obstacle in Figure~\ref{fig:U-map-progress} is designed to demonstrate this problem.
Note that only the car at the top can reach the center of the map without much trouble.
However, we deliberately place it to face the boundary wall.
As a consequence, the center of the map is much less explored.

In short, our experiment shows our \dmap application can be deployed to more devices with little effort.
With more devices, it can explore the map faster and more complete.
We also see that the difficulty to achieve completeness largely depends on the limitations of underlying devices and path planning algorithms.
