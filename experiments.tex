\newcommand{\Koord}{\ensuremath{\lgname}\xspace}
\newcommand{\CyPhyHouse}{CyPhyHouse\xspace}
\newcommand{\Gazebo}{Gazebo\xspace}

\newcommand{\ScanToMap}{\ensuremath{\mathit{scanToMap}}\xspace}
\newcommand{\TSync}{\ensuremath{\mathit{tSync}}\xspace}

\section{Case study: Distributed Mapping application}
\label{sec:experims}

For our case study, we test the \dmap application using the simulator from \CyPhyHouse~\cite{ghosh2019cyphyhouse} for distributed \Koord applications.
We choose the MIT RACECAR model~\cite{MIT_RACECAR} with Lidar sensor included in \CyPhyHouse for our experiment.
To implement external functions defined in~\ref{sec:analysis},
we extend the \CyPhyHouse tool chain and examine our assumptions over sensor data and implemented functions.
In the section, we first discuss our implementation of external functions;
then we present the simulation result of \dmap with multiple vehicles.


\subsection{Implementation of External Functions}

\paragraph{Sound Mapping from Synchronized Sensor Data.}
To generate a 2D map from the scan data~(\ScanToMap),
it is required to synchronize the current vehicle position and the Lidar scan data~(\TSync).
An existing solution is to filter the time stamps of both sensor data streams
and only choose pairs of data within a given time difference threshold $\epsilon$.
This can be achieved by \texttt{ApproximateTimeSynchronizer} in the ROS package named \texttt{message\_filter}.
The frequency of the synchronized data is then limited to the sensor with the lowest frequency.
In our simulation, the lowest frequency is 100 Hz for sensing the vehicle position and is sufficient for updating the mapping.

The time difference $\epsilon$ however can introduce an error distance between real and sensed object locations proportional to the vehicle speed.
We believe this error distance is inevitable.

\subsubsection{removing false positives (rate of success)}


\subsection{Simulation Result}


