\section{Complete \lgname Syntax and Semantics}

\input{syntax.tex}

% We implemented \lgname semantics in \K, which a rewriting-based executable framework for defining language semantics. One can view a language semantics naturally as a set of reduction rules over configurations. Components of a configuration are called {\em cells} in \K. To implement the semantics discussed in \refsect{semantics}, the configuration in \K includes several bookkeeping variables as well.
%
%


%Semantics in \K is expressed using configurations, which organize the components in elements called {\em cells}. Cells are labelled, have types indicating what kind of elements can be contained in them, and help specify the rewrite rule in context. \K allows underspecification of rewrite rules, meaning, only the rewrite rules affecting part of the configurations need to be specified if the rule doesn't affect the other parts of the rule. Our notion of configurations translate in a straightforward manner to \K configurations. We present a couple of example rewrite rules in \K to demonstrate how the executable semantics is implemented.
%
%
%We use a top level {\em system} cell with nested cells corresponding to the elements discussed in \refsect{semantics}, as well as other cells which are used to store information like number of robots in the system, simulation parameters of the program, and other bookkeeping information we used to implement the semantics. There is a special cell called the {\em k} cell, which is used to store the current computation in the program. Each robot has a {\em k} cell to store its computation. Recall that a local variable is updated only in the local memory. The rule for this checks whether the variable is not in the keys of the map {\em sysLoc} from shared variables to their locations. The following rule in \K that corresponds to our \textsc{lvar-assign} rule in the semantics.
%\vspace{2pt}
%\begin{mdframed}
%\begin{Verbatim}[fontsize=\tiny]
%<robot> ... <k> V:Id = I:Val => .  ... </k>
%            <loc> ... V |-> L ... </loc>
%            <locstore> ... L |-> _ => I ... </locstore>
%        ... </robot>
%<sysloc> Rho </sysloc>                    when L notBool inKeys(Rho)
%\end{Verbatim}
%\end{mdframed}
%


\section{Code and Figures}
\chiao{Appendix should not be included for PLDI and should be provided separately as supplementary materials.
We just put figures and code here for now to count the pages.}


\subsection{Distributed Task Allocation}


\begin{figure}[h!]
    \includegraphics[width=\columnwidth]{figs/taskalloc_w_marker.png}

    \caption{Screenshot in the simulation of Distributed Task Allocation}
\end{figure}




\subsection{Distributed Mapping Problem}
\sayan{
    Informally, the problem requires a set of robots to collaboratively mark the position of static \emph{obstacles} within a given area $D$, which any robot should avoid while moving in $D$.The key difference between distributed SLAM and this application is that we assume that the robots know their \emph{global coordinates} within the area of deployment. They are only attempting to map the static obstacles within this area. We currently assume that the only sensors available for sensing obstacles are LIDAR based, and the robots are constrained to move in a 2-D space.
}

\subsection{\dmap Application}

\reffig{flowmap1} shows a simple idea for solving the mapping problem problem for each robot, and \reffig{mapapp} shows our solution to $\dmap$ written using $\lgname$, a high level language with native support for multi-robot systems designed to interact with a physical environment. The design of solution to the mapping problem in \reffig{flowmap1} captures an occupancy map of the 2D space in a variable $\gmap$. \reffig{flowmap}. The variable $\lmap$ is a local mapping constructed by each robot $i$ using sensors, and information from other robots shared via $\gmap$. The robot first updates its $\lmap$ from $\gmap$, which stores the currently computed occupancy map \emph{by all robots}.  The robot then picks a new point in the grid known to be unoccupied in its $\lmap$ and follows a path ($\mathit{Motion.Path}$) moving only over grid rectangles known to be unoccupied by its $\lmap$. While the robot hasn't reached the target rectangle, it keeps updating its $\lmap$ with sensed data (occupied and unoccupied grid points). When it reaches the target, it updates the $\gmap$ with new data from its $\lmap$.


\begin{figure}[!htbp]
    \centering
    \includegraphics[width=\linewidth]{figs/map_flowchart.png}
    \caption{Flowchart for a simple solution to 2D distributed mapping problem\vspace{-2mm}}
    \label{fig:flowmap1}
\end{figure}

 An \emph{allwrite} variable is a shared variable which all robots can read from and write to. The shared \emph{allwrite} variable $\gmap$ is used to construct a shared map of obstacles within the domain $D$, and has type $\mathit{GridMap}$, which is a 2-D array representing a grid over $D$. The \emph{local} variable $\lmap$ represents each robot's \emph{local} knowledge of the domain $D$, and has the same type as $D$. A robot executing the \emph{NewPoint} event, first updates a \emph{local variable} $\lmap$ from the shared variable $\gmap$, using a combination operator $\oplus$, described in more detail in \refsect{prelims}. $\lgname$ allows the user to use library functions, like the $\mathit{findPath}$ function, which uses a path planner to find a path to a point while avoiding a set of \emph{obstacles}. The point is picked using the $\mathit{pickFrontierPos}$ function which is a user defined function implemented in $\lgname$. The details of the path planner, and how the point is picked are discussed in and \refsect{prelims} \refsect{experiments}.

In $\dmap$, there are two modules \emph{Motion} and \emph{Lidar} which provide interfaces to different sensors and actuators on the robot. The $\mathit{Lidar.ldata}$ sensor module is used to read the LIDAR scan of the actual robot. In the \emph{NewPoint} event, the controller driving the robot directs it along a path set at the actuator port $\mathit{Motion.path}$. The sensor port $\mathit{Motion.psn}$ gives the position of the robot (in a fixed coordinate system) and $\mathit{Motion.reached}$ indicates whether the controller is active or inactive.

The $\lgname$ program corresponding to this solution has three \emph{events}: \emph{NewPoint, LUpdate, GUpdate}. Each robot in the system runs an instance of the \emph{same} program. At runtime, the $\lgname$ program executes within the runtime system of a single robot, or a collection of programs execute on different robots. The $\lgname$ language semantics ensures that the execution of the program in the distributed system occurs in \emph{rounds} of duration $\delta$. In each round, each robot executes the statements in the effect(\textbf{eff}) of at most one \emph{enabled} event : an event whose precondition(\textbf(pre)) is satisfied. If no event is enabled, the robot does nothing. Before the next round of execution, the robot may continue to interact with the physical environment as directed by its controllers. The $\lgname$ semantics imposes a synchronous model of execution for $\lgname$ programs for multi-robot systems and its implementation in CyPhyHouse toolchain ensures that this schedule is maintained by the multi-robot system executing the $\lgname$ program, despite potentially imprecise synchronization of local clocks.

In the design of the $\dmap$ application, we employ the use of \emph{sampled} sensors, which is essentially a sequence of timestamped sensor readings between rounds. The \emph{LUpdate} event can occur while each robot is traversing a path and hasn't reached the final destination, during which the robot uses a sampled sensor reading of the positions $\mathit{Motion.trace}$, of type $\mathit{PosStamped}[]$, which denotes an array of timestamped positions; and a sampled sensor reading of LIDAR scans $\mathit{Lidar.ldata}$ of type $\mathit{ScanStamped}$, which denotes an array of timestamped LIDAR scans. These sampled sensor readings are then synchronized to associate a LIDAR scan with a position.

Mutual exclusion is an essential feature required in a distributed system with shared variables. The robot updates the shared variable $\gmap$ in the event \emph{GUpdate} using the value of its $\lmap$, which may have been updated with newly detected obstacles



%
% Algorithm~\ref{ag1}  computes reachable states of a given system based on the above functions and check $n$-invariance.
%$\mathit{Unique}(\mathit{Perms}(\UINS), P_\mathit{events})$ is the set of permutations of the robots that may result in unique behaviors,
% and we compute this set by analyzing the shared variables written to during enabled and urgent events of each robot during a round of program transitions.
% The input to \refalg{ag1} is \begin{inparaenum}[(i)]
%    \item $P$: the program,
%    \item $\mathit{inv}$: a candidate invariant function,
%    \item {\UINS}: set of robots,
%    \item $\delta$: time step size and
%    \item $n$: length executions to check.
%\end{inparaenum}
%
%%
%
%\begin{algorithm}
%\scriptsize
%\textbf{Input}: $P$, $\mathit{inv}$, $\UINS$, $\delta$, $n$\\
%$c\leftarrow \mathit{Init}(P,N)$\quad$p \leftarrow \mathit{Unique}(\mathit{Perms}(\UINS),P_{\mathit{events}})$\\
%\lIf{
%$ \mathit{Sat}(c,\neg\mathit{inv})$} { \Return `unsafe'}
%$C \leftarrow$ \{c\} \\
%\For{$i = 0$ \KwTo n}{
%\For{$j = 0$ \KwTo $\mathit{len}(p)$ }{
%$C^\prime\leftarrow\Post(C,p[j])$\\
%\For{$c$ $\mathbf{in}$ $C'$} {
%\lIf{
%$ \mathit{Sat}(c,\neg\mathit{inv})$} { \Return (`unsafe',c)}
%}
%$C^\prime \leftarrow \Final(C,p[j])$\\
%\rg{\emph{continuous reachability here}}
%}}
%\Return `safe'
%\caption{Bounded invariant checking algorithm}\label{ag1}
%\end{algorithm}
%  In the next iteration, $C$ is set to be the frontier set of configurations.
%Theorem~\ref{thm:bmcsound} summarizes the soundness of \refalg{ag1}.
%
%\begin{theorem}
%    \label{thm:bmcsound}
%    If Algorithm~\ref{ag1}  returns `unsafe' then there exists a counter-example to \emph{inv} of length at most $n$, \rg{state assumption on continuous behavior here}.
%    if Algorithm~\ref{ag1} returns `safe' then \emph{inv} is an $n$-invariant.
%\end{theorem}
%\begin{proof}
%{\em Sketch.} If algorithm \ref{ag1} returns {\em safe}, then for the $n$ loop iterations of the outer loop, given a set of states $C$ and the property \texttt{inv}, according to the algorithm $\forall c \in \Post(C), c \models \texttt{inv}$. $\Post(C)$ is the set of configurations that can be reached from $C$ using any possible sequence of rewrite rules allowed by the system before an environment transition. Therefore, if the algorithm returns safe, then the set of configurations reached from $C$ before an environment transition is also safe. \rg{continue proof of soundness of reachability of continuous behavior}
%\end{proof}
%\fTBD{Relate this to assumptions}
%


\subsection{Specify Constraints over Modules}
\label{sec:module}
\chiao{As mentioned in \refsect{semantics}, the assumptions over modules will be directed by the target invariant/property.
    What does \lgname provide to help identify the assumptions?}


\subsection{Proofs}
\begin{proof}
    Proof for Lemma~\ref{lem:noninter}:
    Suppose that given an arbitrary configuration $c$, such that $\bigwedge_{i\in \UINS}\bigwedge_{e \in \Event} \eec{\mathit{inv}}{\Post(\left\{c\right\}, i, e)}$ holds true. Consider a permutation $\incurly{i_1,i_2,\ldots,i_{\NMAX}}$ of robots executing events $\incurly{e_1, e_2, \ldots , e_{\NMAX}}$ respectively in a program transition during a round, reaching configurations $c_1, c_2, \ldots, c_{\NMAX}$ respectively. By assumption, since $\Post(\incurly{c},i,e) = c_1$, $\eec{\mathit{inv}}{c_1}$. Since our assumption was about any arbitrary (symbolic) configuration $c$, the same argument therefore can be extended to all the configurations reached by the system during this program transition, i.e. $\forall c' \in \incurly{c_1, c_2, \ldots, c_n}, \eec{\mathit{inv}}{c}$.
\end{proof}