\subsection{Occupancy Map Construction}

We now set up the terminology for analysis of the construction of $\map_i$ in an $\lgname$ application.  Recall that \fTBD{setup Koord in overview}

\subsubsection{Approach}


\begin{figure}[!htbp]
    \centering
    \includegraphics[width=\linewidth]{figs/map_flowchart.png}
    \caption{Flowchart for a simple solution to 2D distributed mapping problem\vspace{-2mm}}
    \label{fig:flowmap}
\end{figure}



We discuss now discuss how our algorithm implemented $\lgname$ shown in \reffig{mapapp} tackles $\mapprob$. \reffig{flowmap} shows a simple idea for solving this problem for each robot:

The variable $\lmap$ refers to the current mapping $\map_i$ constructed by each robot $i$ using the algorithm. The function $\mathit{MaxExp}$ informally, determines whether there is a grid rectangle in the frontier of the current $\map_i$ . If not, the robot first updates its $\lmap$ from $\gmap$, which is used for sharing the currently computed occupancy maps by all agents so far. The robot then picks a new point in a rectangle known to be unoccupied in its $\lmap$ and follows a path ($\mathit{Motion.Path}$) moving only over grid rectangles known to be unoccupied by its $\lmap$. While the robot hasn't reached the target rectangle, it keeps updating its $\lmap$ with sensed data (occupied and unoccupied rectangles). When it reaches the target, it updates the $\gmap$ from its $\lmap$.





%We assume that if robot $i$ is at position $\pos(i)$, and it detects an obstacle in $q \in \sensarea(\qfunc(\pos(i)))$, then $q$ does contain an obstacle, otherwise it does not. The point of this presentation isn't dealing with the issue of potentially false positives while identifying an obstacle. A robot partially constructs a grid occupancy function by assigning values to the grid rectangles in its \emph{sensed} area.




%\begin{lemma}
%    \label{ext}
%    Suppose robot $i$ is at $\pos(i)\in D$, and $\exists(q^\prime)\in \sensarea(\pos(i))$,
%    s.t $\map_i(q) = -1$. Consider a mapping, $\map^\prime_i:Q \mapsto \left\{-1,0,1\right\}$
%    such that, $\forall q\in Q \setminus \sensarea(\pos(i)), map^\prime_i(q) = map_i(q)$
%    and $\forall q \in \sensarea(\pos(i)), \map^\prime_i(q) = \world_Q(q)$.
%    Then, $\map^\prime_i(q)$ is sound if $\map_i$ is sound.
%\end{lemma}
%
%The proof for this is straightforward and follows from the definition of $\map^\prime_i$ combined with \defn{soundness}.
%
%Recall from the definition of the sensing area of a robot $i$ at $\pos(i)$, that it can reliably compute the ground truth mapping $\world_Q$ for all $q \in \sensarea(\pos(i))$. This lemma essentially states that given a sound occupancy mapping $\map_i$, it can be used to compute to another \emph{sound} occupancy mapping $\map^\prime_i$ by setting the values of $\map^\prime_i(\sensarea(\pos(i)))$ to the ground truth function.
%
%\begin{definition}
% We say that $\Vec{x_n}\in D$ is \emph{reachable} from $\Vec{x_0}$ if there is a \emph{path} $p = \Vec{x_0},\Vec{x_1}, \Vec{x_2},\ldots, \Vec{x_n}$ , such that
%\begin{itemize}
%\item $\forall i \in [0..n],\world_Q(\qfunc{\Vec{x_i}}) = 0$
%\item a robot can move from $\Vec{x_i}$ to $\Vec{x_{i+1}}$ for $i \in [0..n]$, while staying within $\qfunc(\Vec{x_i}) \cup \qfunc(\Vec{x_{i+1}})$.
%\end{itemize}
%\end{definition}
%
%A grid rectangle $q\in Q$ is reachable in general, if $\exists \Vec{x}\in D, q = \qfunc(\Vec{x})$ such that $\Vec{x}$ is reachable from either \begin{inparaenum} [(a)]\item the initial position of a robot, or \item another reachable point $\Vec{x^\prime}\in D$. We denote the reachability of a grid rectangle using a predicate $\mathit{Reach} : Q \mapsto \left\{\mathit{True}, \mathit{False}\right\}$, where $\mathit{Reach}(q) = \mathit{True}$ if its reachable,  $\mathit{False}$ otherwise.
%\end{inparaenum}
%
%\begin{definition}
%    Given a sound occupancy mapping $\map_i$, the \emph{frontier} of $\map_i$, denoted by $\ff(\map_i)$ is defined as follows:
%    $$ \left\{ q\in Q \mid Reach(q) \wedge \exists q \in \sensarea(\Vec{x}), \map_i(q) = -1\right\} $$
%\end{definition}
%
%Given a sound occupancy mapping $\map_i$, another sound mapping $\map_i^\prime$ can be constructed as shown in \lem{ext}. Taken in conjunction with our assumption on the initial positions of each robot, this leads to a strategy for computing sound occupancy mapping functions. Further, given a set of sound mappings $\left\{\map_i\right\}_{i\in[N]}$, we can construct a sound mapping $\map$ from them as follows :
%
%\begin{lemma}
%    Given a set of sound mappings $\left\{\map_i\right\}_{i\in[N]}$, the mapping described by $\map: Q \mapsto \left\{-1,0,1\right\}$ such that $\map(q) = \mathit{Max}_{i\in[N]}\left\{\map_i(q)\right\}$ is also sound.
%\end{lemma}
%
%\begin{proof}
%    If $\map(q) = 1$, $\exists i\in [N]$, s.t $\map_i(q) = 1$. By \lem{comptbl}, $\forall j \in [N], j\neq i \Rightarrow \map_j(q) = 1\vee \map_j(q) = -1$. Since $\map_i$ is sound, $\map(q) = \world_Q(q)$. The case for $\map(q) = 0$ is also similar. Therefore, $\map(q) = 0\vee \map(q) = 1\Rightarrow \map(q) = \world_Q(q)$. Otherwise, by definition, $\map(q) = -1$, which together with the previous statement, implies that $\map$ is sound.
%\end{proof}
%
%
%

%
%
%
\subsubsection{External (Library) Functions}

% restriction of the world function for sensing. accuracy of sensor statement.
% domain of mapping function for which value is 0 or 1.
% make compatibility a definition instead of lemma.