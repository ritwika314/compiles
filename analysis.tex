

\subsection{Analysis of \dmap}
\label{sec:analysis}


In this section, we present an analysis of the $\lgname$ program for the mapping problem.
% initial states of $\A$.


Consider a state $\vs\in \Reach(\A)$. For $\vs$, the individual soundness, and the soundness requirements of the mapping problem are captured by the following invariants of $\A$ respectively.


\begin{invariant}
\label{ind-sound}
For all $i \in [N]$ $$\forall q_1 \in \domain(\vs.\lmap_i) , \vs.\lmap_i(q) =  \world_Q(q)$$
\end{invariant}

\begin{invariant}
\label{sound}
For all $i \in [N]$, $$\forall q_1 \in \domain(\vs.\gmap_i) , \vs.\gmap_i(q) =  \world_Q(q)$$
\end{invariant}

The consistency requirement of the mapping problem is captured as follows:
\begin{invariant}
\label{consistency}
For all $i,j \in [N]$, for all $q \in \domain(\vs.\lmap_i)\cup \domain(\vs.\lmap_j)$, $$\vs.\lmap_i = \vs.\lmap_j$$
\end{invariant}

We omitted the initialization of the mappings in the presentation of the program in \reffig{mapapp}. From \asum{initval}, for each robot $i$, given its initial state $\vs\in C_0$, $\vs.\lmap_i$ is a sound mapping and $\vs.\gmap_j$ is initialized to be empty, Therefore all initial states satisfy \inv{ind-sound}, and \inv{sound}.


Given $\map_i$, $\map_j$, and $q\in Q_i \cup Q_j$ , let $\map_i(q) = 1$, and $\map_j(q) = 0$. Suppose $\map_i$ is sound, then $\world_Q(q) = 1$, which implies $\map_j$ is not sound. By the same argument, if $\map_j$ is sound, $\map_i$ is not. This implies that $\vs\in C$ has individual soundness only if it also has consistency. Therefore, given $\vs\in C_0$, it satisfies \inv{consistency}.

Suppose, a state of a system $\vs\in C$ satisfies \inv{sound}, \inv{ind-sound} and \inv{consistency} , we now prove that given a transition $(\vs, a, \vs')\in \D$, $\vs'$ also satisfies them. Since these statements are only about \emph{program} variables, $\delta$-transitions don't affect them. Hence, we only need to prove that each event transition preserves these invariants.

\paragraph{NewPoint}
For every robot $i$, $\vs.\gmap_i = \vs'.\gmap_i$. Thus, this event trivially preserves \inv{sound}. So far, we haven't described the \emph{merge} function in detail, but we do so now. Given two mappings $m:Q'\mapsto \mathbb{B}$ and $n:Q''\mapsto \mathbb{B}$, the $\mathit{merge}$ function returns a mapping $\mathit{mn}: Q'\cup Q'' \mapsto \mathbb{B}$ where, \begin{itemize}
      \item for every $q \in Q'\setminus Q''$, $\mathit{mn}(q) = m(q)$
      \item for every $q \in Q''\setminus Q'$, $\mathit{mn}(q) = n(q)$
      \item for every $q \in Q \cap Q''$, $\mathit{mn}(q) = m(q) $
\end{itemize}
If $m$ and $n$ are sound, then $\mathit{merge}(m,n)$ also returns a sound mapping. Given that for every $i\in [N]$, $\vs'.\lmap_i = \mathit{merge}(\vs.\lmap_i,\vs.\gmap_i)$, and by assumption $\vs.\lmap_i$ and $\vs.\gmap_i$ are sound, $\vs'.\lmap_i$ is also sound. Therefore, this event preserves \inv{ind-sound}.

Now suppose that \inv{consistency} is not preserved, i.e. there are robots  $i, j \in [N]$, and a common $q \in \domain(\vs'.\lmap_i)\cap\domain(\vs'.\lmap_j)$, such that $\vs'.\lmap_i[q] \neq \vs'.\lmap_j[q]$. However, since $\vs'$ satisfies \inv{ind-sound}, we have proved that this event preserves \inv{consistency} by contradiction.

\paragraph{LUpdate}
Again, for every robot $i$, $\vs.\gmap_i = \vs'.\gmap_i$. Thus, this event trivially preserves \inv{sound}.


let $\vs.\mathit{Motion.trace}_i = \{(t_j, p_j)\}_{j \in [M]}$ where $p_j$ is the position of the robot $i$ at time $t_j$ from the beginning of the round and  $\vs.\mathit{Lidar.ldata}_i = \{t_j^\prime, l_j^\prime\}_{j^\prime \in M^\prime}$ where $l_j^\prime$ is the LIDAR scan at time $t_j^\prime$ from the beginning of the round.

The function $\mathit{tSync}$ is used to \emph{synchronize} the readings, where we compute a mapping $\mathit{pScan}: \mathit{Pos} \mapsto \mathit{Scan}$, where given a state $\vs\in Q$, a robot $i\in [N]$,  $\vs.\mathit{pscan}_i = \{(p_j^{\prime\prime}, l_j^{\prime\prime})\}_{j^{\prime\prime} \in M ^{\prime\prime}}$, where  $(p, l) \in \vs.\mathit{pscan}_i$ if given $\epsilon > 0$,  $(t,p) \in \vs.\mathit{Motion.trace}, (t^\prime,l )\in \vs.\mathit{Lidar.ldata}_i, |t - t^\prime| \leq \epsilon$.

The function $\mathit{scanToMap}$ is the same one discussed in \asum{maptoscan}, and it returns a \emph{sound} map by definition. This combined with the fact that $\vs.\lmap_i$ is sound by the induction hypothesis, $\vs'.\lmap_i$ is also sound, thus preserving \inv{ind-sound}.
As before, we can prove as that since \emph{LUpdate} preserves \inv{ind-sound}, it also preserves \inv{consistency}. Note that the function $\mathit{tSync}$ may return an empty set, which doesn't affect the soundness requirement.

\paragraph{GUpdate}
Since $\mathit{\gmap}$ is updated using the keyword \emph{atomic}, only one agent $i\in [N]$ writes to the shared variable $\gmap$.  $\vs.\lmap_i$ and $\vs.\gmap_i$ are sound, by definition of \emph{merge}, $\vs'.\gmap_i$ is also sound. By \asum{shared}, the same value is propagated to all other agents, meaning $\forall j \in [N], \vs'.\gmap_j$ is sound as well. Therefore, this event preserves \inv{sound}. \emph{GUpdate} preserves \inv{ind-sound} trivially as it doesn't modify $\vs.\lmap_i$, and as a result, it also preserves \inv{consistency}.




%\begin{lemma}
%    \label{ext}
%    Given a robot $i$, suppose the mapping $\map_i:Q_i \mapsto \mathbb{B}$, where $Q_i\subseteq Q$, is sound. Suppose further, that the robot $i$ is is within $q'\in Q$. Consider a mapping, $\map^\prime_i: Q_i \cup \sensarea(q') \mapsto \mathbb{B}$
%    such that, $\forall q\in Q_i, map^\prime_i(q) = map_i(q)$
%    and $\forall q \in \sensarea(q'), \map^\prime_i(q) = \sensfunc(q)$. Then, $\map'_$
%\end{lemma}
%
%The proof for this is straightforward and follows from the definition of $\map^\prime_i$ combined with \defn{soundness}.
%
%Recall from the definition of the sensing area of a robot $i$ at $\pos(i)$, that it can reliably compute the ground truth mapping $\world_Q$ for all $q \in \sensarea(\pos(i))$. This lemma essentially states that given a sound occupancy mapping $\map_i$, it can be used to compute to another \emph{sound} occupancy mapping $\map^\prime_i$ by setting the values of $\map^\prime_i(\sensarea(\pos(i)))$ to the ground truth function.

%Given a mapping $\map_i$, $\domain(\map_i)$ denotes $ Q_i \subset Q$, such that $\forall q \in Q_i \map_i(q) = 0 \vee \map_i(q) = 1$.


%$Given two sound maps $\lmap$ and $\gmap$, $\lmap \oplus \gmap$ corresponds to a combined map construction as outlined in \defn{cons}, and is sound. The event \emph{NewPoint} doesn't modify $s_i.M.\lmap$ further, and therefore, $\lmap$ remains \emph{sound} during the execution of this event.

%\begin{definition}
%    We say that $\Vec{x_n}\in D$ is \emph{reachable} from $\Vec{x_0}$ if there is a \emph{path} $p = \Vec{x_0},\Vec{x_1}, \Vec{x_2},\ldots, \Vec{x_n}$ , such that
%    \begin{itemize}
%        \item $\forall i \in [0..n],\world_Q(\qfunc{\Vec{x_i}}) = 0$
%        \item a robot can move from $\Vec{x_i}$ to $\Vec{x_{i+1}}$ for $i \in [0..n]$, while staying within $\qfunc(\Vec{x_i}) \cup \qfunc(\Vec{x_{i+1}})$.
%    \end{itemize}
%\end{definition}
%
%A grid rectangle $q\in Q$ is reachable in general, if $\exists \Vec{x}\in D, q = \qfunc(\Vec{x})$ such that $\Vec{x}$ is reachable from either \begin{inparaenum} [(a)]
%                                                                                                                                                   \item the initial position of a robot, or \item another reachable point $\Vec{x^\prime}\in D$. We denote the reachability of a grid rectangle using a predicate $\mathit{Reach} : Q \mapsto \left\{\mathit{True}, \mathit{False}\right\}$, where $\mathit{Reach}(q) = \mathit{True}$ if its reachable,  $\mathit{False}$ otherwise.
%\end{inparaenum}
%
%\begin{definition}
%    Given a sound occupancy mapping $\map_i$, the \emph{frontier} of $\map_i$, denoted by $\ff(\map_i)$ is defined as follows:
%    $$ \left\{ q\in Q \mid Reach(q) \wedge \exists q \in \sensarea(\Vec{x}), q \notin \domain(\map_i)\right\} $$
%\end{definition}





%\begin{lemma}
%    \label{ext}
%    Suppose robot $i$ is at $\pos(i)\in D$, and $\exists(q^\prime)\in \sensarea(\pos(i))$,
%    s.t $\map_i(q) = -1$. Consider a mapping, $\map^\prime_i:Q \mapsto \left\{-1,0,1\right\}$
%    such that, $\forall q\in Q \setminus \sensarea(\pos(i)), map^\prime_i(q) = map_i(q)$
%    and $\forall q \in \sensarea(\pos(i)), \map^\prime_i(q) = \world_Q(q)$.
%    Then, $\map^\prime_i(q)$ is sound if $\map_i$ is sound.
%\end{lemma}
%
%The proof for this is straightforward and follows from the definition of $\map^\prime_i$ combined with \defn{soundness}.
%
%Recall from the definition of the sensing area of a robot $i$ at $\pos(i)$, that it can reliably compute the ground truth mapping $\world_Q$ for all $q \in \sensarea(\pos(i))$. This lemma essentially states that given a sound occupancy mapping $\map_i$, it can be used to compute to another \emph{sound} occupancy mapping $\map^\prime_i$ by setting the values of $\map^\prime_i(\sensarea(\pos(i)))$ to the ground truth function.
%
%\begin{definition}
% We say that $\Vec{x_n}\in D$ is \emph{reachable} from $\Vec{x_0}$ if there is a \emph{path} $p = \Vec{x_0},\Vec{x_1}, \Vec{x_2},\ldots, \Vec{x_n}$ , such that
%\begin{itemize}
%\item $\forall i \in [0..n],\world_Q(\qfunc{\Vec{x_i}}) = 0$
%\item a robot can move from $\Vec{x_i}$ to $\Vec{x_{i+1}}$ for $i \in [0..n]$, while staying within $\qfunc(\Vec{x_i}) \cup \qfunc(\Vec{x_{i+1}})$.
%\end{itemize}
%\end{definition}
%
%A grid rectangle $q\in Q$ is reachable in general, if $\exists \Vec{x}\in D, q = \qfunc(\Vec{x})$ such that $\Vec{x}$ is reachable from either \begin{inparaenum} [(a)]\item the initial position of a robot, or \item another reachable point $\Vec{x^\prime}\in D$. We denote the reachability of a grid rectangle using a predicate $\mathit{Reach} : Q \mapsto \left\{\mathit{True}, \mathit{False}\right\}$, where $\mathit{Reach}(q) = \mathit{True}$ if its reachable,  $\mathit{False}$ otherwise.
%\end{inparaenum}
%
%\begin{definition}
%    Given a sound occupancy mapping $\map_i$, the \emph{frontier} of $\map_i$, denoted by $\ff(\map_i)$ is defined as follows:
%    $$ \left\{ q\in Q \mid Reach(q) \wedge \exists q \in \sensarea(\Vec{x}), \map_i(q) = -1\right\} $$
%\end{definition}
%
%Given a sound occupancy mapping $\map_i$, another sound mapping $\map_i^\prime$ can be constructed as shown in \lem{ext}. Taken in conjunction with our assumption on the initial positions of each robot, this leads to a strategy for computing sound occupancy mapping functions. Further, given a set of sound mappings $\left\{\map_i\right\}_{i\in[N]}$, we can construct a sound mapping $\map$ from them as follows :
%

%
%

%
%
%

% restriction of the world function for sensing. accuracy of sensor statement.
% domain of mapping function for which value is 0 or 1.
% make compatibility a definition instead of lemma.