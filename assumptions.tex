\subsection{Assumptions}
\label{sec:formal:sensing}

The formal semantics of $\lgname$ is defined under certain timing and system information assumptions. Additionally, given that the application involves interaction with physical plants, we formally state several assumptions about low-level sensing concerns which are abstracted away by $\lgname$.

\subsubsection{Perception}


 % $\map_i(q) = 1$ indicates that according to robot $i$, there is an obstacle in $q$, $\map_i(q) = 0$ indicates that according to robot $i$, $q$ is unoccupied, and $\map_i(q) = -1$ indicates that robot $i$ doesn't have information about $q$.

%We assume that given $\Vec{x} \in q, \forall \Vec{x^\prime} \in q, \sensarea(\Vec{x}) \subseteq \sensarea(\Vec{x^\prime})$.


The sensors used by a robot for perception of the ground truth around it are \emph{Motion.trace} and \emph{Lidar.ldata}, which are \emph{sampled} sensors. Given that $\delta$ is the duration of a \emph{round},

\begin{itemize}
    \item $\mathit{Motion.trace}: [0,\delta] \mapsto \mathit{Pos}$. Given a state $\vs\in Q$ ,  $\forall i \in [N]\vs.\mathit{Motion.trace}_i = \{(t_j, p_j)\}_{j \in [M]}$ where $p_j$ is the position of the robot $i$ at time $t_j$ from the beginning of the round.
    \item  $\mathit{Lidar.ldata} : [0,\delta]\mapsto \mathit{Scan}$. Given a state $\vs\in Q$ , $\forall i \in [N], \vs.\mathit{Lidar.ldata}_i = \{t_j^\prime, l_j^\prime\}_{j^\prime \in M^\prime}$ where $l_j^\prime$ is the LIDAR \emph{scan} at time $t_j^\prime$ from the beginning of the round.
\end{itemize}

\begin{assumption}
    Given a point $p_i$, if there is a corresponding lidar scan $l_i$ when the robot is in $\qfunc(p_i)$, then there is a function $\sensfunc$ such that $\forall q\in Q$ such that $q \in \domain(\sensfunc), \sensfunc(q) = \world_Q(q)$.
\end{assumption}

$\domain(\sensfunc)$ of a robot is defined as its sensing area. This assumption is required to ensure the \emph{Individual soundess} of a mapping updated using LIDAR scan information.

%The sampling frequencies of these sensors may be different, hence the potentially different number of readings, and potentially different timestamps of the readings.

%The function $\mathit{tSync}$ is used to \emph{synchronize} the readings, where we compute a mapping $\mathit{pScan}: \mathit{Pos} \mapsto \mathit{Scan}$, where given a state $\vs\in Q$, a robot $i\in [N]$,  $\vs.\mathit{pscan}_i = \{(p_j^{\prime\prime}, l_j^{\prime\prime})\}_{j^{\prime\prime} \in M ^{\prime\prime}}$, where  $(p, l) \in \vs.\mathit{pscan}_i$ if given $\epsilon > 0$,  $(t,p) \in \vs.\mathit{Motion.trace}, (t^\prime,l )\in \vs.\mathit{Lidar.ldata}_i, |t - t^\prime| \leq \epsilon$.

%The function $\mathit{scanToMap}: \mathit{Scan} \times \mathit{Pos}\mapsto \mathit{GridMap}$, given a position $p_i$ and its corresponding scan $l_i$, computes a quantized domain function $\sensfunc$ for .

%By definition, this returns a \emph{sound} mapping. We stated earlier that the operator $\oplus$ applied to two sound mappings returns a sound mapping. Therefore, given $s_i^\prime$ such that $\mathit{s_i^\prime \in \mathit{trans}(s_i,\mathit{LUpdate})}$,  $s_i^\prime.M.\lmap$ is sound if $s_i.M.\lmap$ is sound.


\begin{definition}
   Given a robot at a position $\Vec{x}\in D, \qfunc{\Vec{x}} = q$, the \emph{sensing area} around $q$ is a function $\sensarea: Q \mapsto 2^{Q}$ where $\forall q^\prime \in \sensarea(q), \sensfunc(q) = \world_Q(q)$.
\end{definition}

%\fTBD{define the data types in the software section}


%The function $\mathit{scanToMap}: \mathit{Scan} \times \mathit{Pos}\mapsto \mathit{GridMap}$, given a position $p_i$ and its corresponding scan $l_i$, computes the quantized domain function$\sensfunc$, in $\sensarea(\qfunc(p_i))$. By definition, this returns a \emph{sound} mapping. We stated earlier that the operator $\oplus$ applied to two sound mappings returns a sound mapping. Therefore, given $s_i^\prime$ such that $\mathit{s_i^\prime \in \mathit{trans}(s_i,\mathit{LUpdate})}$,  $s_i^\prime.M.\lmap$ is sound if $s_i.M.\lmap$ is sound.


\subsubsection{Problem Setup}

A vacuously correct (sound) solution to $\mapprob$ given $D, \qfunc$ and $\world_Q$ is $\forall i \in [N]$, $$\map_i : Q_i \mapsto \left\{0,1\right\}, Q_i = \phi$$ To allow for potentially more interesting solutions than the one stated above, we make the following assumption on the
\begin{assumption}
Initially, each robot $i\in[N]$ starts at a grid rectangle with no obstacle, the sensed area of each robot $i$ is non empty
$\forall i \in [N], \world_Q(q^0_i) = 0\wedge \sensarea(q^0_i) \neq \phi$
where $q^0_i = \qfunc(\pos_0(i))$, and $\pos_0(i)$ denotes the initial position of robot $i$.
\end{assumption}

This ensures that a robot doesn't violate the \emph{Individual soundness} requirement for a correct solution the mapping problem.

\subsubsection{Planning and actuation}

\begin{definition}
    We say that $\Vec{x_n}\in D$ is \emph{accessible} from $\Vec{x_0}$ if there is a \emph{path} $p = \Vec{x_0},\Vec{x_1}, \Vec{x_2},\ldots, \Vec{x_n}$ , such that
    \begin{itemize}
        \item $\forall i \in [0..n],\world_Q(\qfunc{\Vec{x_i}}) = 0$
        \item a robot can move from $\Vec{x_i}$ to $\Vec{x_{i+1}}$ for $i \in [0..n]$, while staying within $\qfunc(\Vec{x_i}) \cup \qfunc(\Vec{x_{i+1}})$.
    \end{itemize}
\end{definition}

A grid rectangle $q\in Q$ is accessible in general, if $\exists \Vec{x}\in D, q = \qfunc(\Vec{x})$ such that $\Vec{x}$ is accessible from either \begin{inparaenum} [(a)] \item the initial position of a robot, or \item another accessible point $\Vec{x^\prime}\in D$. We denote the accessibility of a grid rectangle using a predicate $\Access : Q \mapsto \left\{\mathit{True}, \mathit{False}\right\}$, where $\Access(q) = \mathit{True}$ if its accessible,  $\mathit{False}$ otherwise.
\end{inparaenum}

\begin{assumption}
    The path planner used by the $\lgname$ implementation of the $\dmap$ application returns a path $p = \Vec{x_0},\Vec{x_1}, \Vec{x_2},\ldots, \Vec{x_n}$ , such that the robot can traverse it while staying within $\qfunc(\Vec{x_i})$ for every $\Vec{x_i}\in p$.
\end{assumption}

This assumption can be shown to ensure for the \emph{Safe location} property of a solution, if for robot $i$, the path planner can be restricted to find paths within $\domain(\map_i)$.

%\begin{definition}
%    Given a sound occupancy mapping $\map_i$, the \emph{frontier} of $\map_i$, denoted by $\ff(\map_i)$ is defined as follows:
%    $$ \left\{ q\in Q \mid \Access(q) \wedge \exists q \in \sensarea(\Vec{x}), q \notin \domain(\map_i)\right\} $$
%\end{definition}

%Assume that the planner returns a path if \begin{inparaenum}[(a)] \item the frontier is non-empty \item the grid rectangle picked on the frontier is reachable from the current point \end{inparaenum}. We also constrain the robots to move only on \emph{known} unoccupied grid rectangles, i.e $q\in Q, s_i.M.\lmap[q] = 0$. In implementation, we achieve this by ensuring that the path planner treats all the unknown ($q$ such that $s_i.M.\lmap[q] = -1$) grid rectangles as obstacles as well.


\subsubsection{Synchronization and consistency}
The following are the timing assumptions on $\lgname$ programs.
\begin{assumption} A program step has zero logical execution time.
\end{assumption}
This assumption is reasonable if the time taken to complete a program transition step is negligible compared to $\delta$.  The $\lgname$ middleware ensures that the participating agents begin executing their events synchronously.

\begin{assumption} Shared memory updates are propagated to all agents instantanously.
\end{assumption}
The $\lgname$ middleware uses message passing to implement shared memory, and compared to $\delta$, the time taken to observe a shared memory update as a received message from the time that it is sent is much smaller. We therefore can assume that in this setting, the shared memory semantics of $\lgname$ wherein an update is visible in the next round of program transitions, is a reasonable implementation deliverable. This assumption in this particular example, is required for \emph{Eventual Completeness}, but in other applications, it may be used to correctness.
%$\marginpar{\scriptsize\sayan{MAke this a list of {\bf Assumptions} with discussion on why they hold, or what they mean in English.}}


% \rg{We can \emph{combine} the elements of the set of \emph{local maps}, $\{\map_i\}_{i\in [N]}$ to form a \emph{global} occupancy mapping. }



%
%\begin{definition}
%    Given  $i , j \in [N]$, two proposed occupancy mappings $\map_i: Q_i\mapsto\left\{0,1\right\}$ and $\map_j: Q_j\mapsto \left\{0,1\right\}$, are consistent only if $\forall q \in Q_i \cup Q_j, \map_i(q) = \map_j(q)$.
%\end{definition}

