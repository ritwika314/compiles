\section{Koord Software Stack}
\label{sec:software}

\subsection{Runtime system}



To run a $\lgname$ program (hardware or simulation), the user has to provide a configuration file, with
\begin{inparaenum}
    \item the number of agents,
    \item in case of simulation, the initial positions of the agents and the length of the simulation and
    \item in case of hardware deployment their IP addresses,
    and the localization system.
\end{inparaenum}

\subsection{Key environment assumptions}


\subsubsection{Periodic event execution semantics}


\subsubsection{Shared variable implementation over message passing}


\subsubsection{Known set of participants}
\subsubsection{Portability and heterogeneity}


\subsection{Simulator}
\subsubsection{gazebo environment}
\subsubsection{car model}
\subsubsection{lidar}
\subsubsection{positioning}
\subsubsection{sampled sensing}
\subsubsection{synchronization issues}

\subsection{The Distributed Mapping Problem}
In this section, we introduce the distributed mapping problem. Informally, the problem requires a set of robots to collaboratively mark the position of static \emph{obstacles} within a given area $D$, which any robot should avoid while moving in $D$.The key difference between distributed SLAM and this application is that we assume that the robots know their \emph{global coordinates} within the area of deployment. They are only attempting to map the static obstacles within this area. We currently assume that the only sensors available for sensing obstacles are LIDAR based, and the robots are constrained to move in a 2-D space.


\subsubsection{Preliminaries}
\label{sec:prelims}
We first set up the terminology and assumptions to discuss our approach to this problem.

We define the mapping problem over a domain $D$ which is bounded rectangle $[a_1,a_2]\times [b_1,b_2]$ in $\mathbb{R}^2$ corresponding to the physical arena. 
The \emph{ground truth} about the occupancy of obstacles in this arena is modeled by a function $\world : D\mapsto \left\{0,1\right\}$, where $\forall \Vec{x} \in D$, 

\begin{align}
\world(\Vec{x}) = 
		\left\{
		\begin{array}{ll}
			1 & \mathit{there \ is\ an\ obstacle\ at}\ \Vec{x}\\
			0 & \mathit{otherwise}.
		\end{array}
		\right.
\end{align}

The map created by agents will be a data-structure and it will (ideally) be a quantized version of the $\world$.
Let $\qdom$ be a $(n_x\times n_y)$-quantized representation of $D$, for some resolution constants $n_x,n_y \in \mathbb{N}$. That is, $\qdom = \{q_{ij}\}_{i\in [1..n_x], j\in [1..n_y]}$
such that every $q_{ij}$ corresponds to a disjoint $[x_i, x_{i+1}] \times [y_j, y_{j+1}]$ in $D$.

\begin{definition}
    The \emph{quantization function}  $\qfunc:D \mapsto \qdom$ maps points in $D$ to their quantized versions. 
    That is $\qfunc(\Vec{x}) = q_{ij}$ iff $\Vec{x} \in [x_i, x_{i+1}] \times [y_j, y_{j+1}]$.
   \end{definition}

\begin{definition}
    Given any function $f:\qdom \rightarrow S$ and a subset $Q' \subseteq Q$, 
    the restriction of $f$ to $Q'$, written as $f \lceil Q'$, is the function
    $g:Q' \rightarrow S$ such that for all $q \in Q'$, $g(q) = f(q)$.
\end{definition}

We will fix the domain $D$, the $\world$, and the quantization of $\qdom$ of $D$ throughout this section, which also fixes the quantization function $\qfunc$ and defines a quantized version of the world $\world_Q : \qdom \mapsto \left\{0,1\right\}$, where
$$\world_Q(q) = \begin{cases}
        1\ \Leftrightarrow \exists \Vec{x}\in \qfunc^{-1}(q), \world(\Vec{x}) = 1, \\
        0\ \mbox{otherwise}
\end{cases}
$$


\rg{Consider that there is a set of ground robots $[N]$, which are tasked with creating a mapping of static obstacles in $Q$ collaboratively by constructing local mappings based on sensed information. Each robot has a \emph{software state}, $\sensfunc : Q \mapsto \left\{0,1\right\}$ associated with it}.



\begin{definition}
   Given a robot at a position $\Vec{x}\in D, \qfunc{\Vec{x}} = q$, we define the \emph{sensing area} of $q$ follows: $\sensarea: Q \mapsto 2^{Q}$, such that  $\forall q^\prime \in \sensarea(q), \sensfunc(q) = \world_Q(q)$.  \end{definition}


Let $\map_i:Q\mapsto \left\{-1,0\right\}$ denote a \qdfunc \emph{local} to robot $i$, which we can see as a software state for a given robot $i$ as well.  % $\map_i(q) = 1$ indicates that according to robot $i$, there is an obstacle in $q$, $\map_i(q) = 0$ indicates that according to robot $i$, $q$ is unoccupied, and $\map_i(q) = -1$ indicates that robot $i$ doesn't have information about $q$.



We assume that given $\Vec{x} \in q, \forall \Vec{x^\prime} \in q, \sensarea(\Vec{x}) \subseteq \sensarea(\Vec{x^\prime})$. We can now state the 2-d distributed mapping problem, $\mapprob$ as follows. \begin{quote}
{\em Given a set of robots $[N]$ , for each robot $i \in [N]$ construct an \emph{occupancy map}, $\map_i: Q_i \mapsto \left\{1,0\right\}$, $Q_i\subseteq Q$.
}
\end{quote}

% \rg{We can \emph{combine} the elements of the set of \emph{local maps}, $\{\map_i\}_{i\in [N]}$ to form a \emph{global} occupancy mapping. }

Having stated the problem, we now define the notion of soundness of a proposed occupancy map.
\begin{definition}
    \label{soundness}
    For a robot $i$, a proposed occupancy mapping over $Q_i\subset Q$, given by $\map_i: Q_i \mapsto \left\{-1,0\right\}$ is \emph{sound} if
     $\world_Q(q) =  \map_i(q) \vee q\notin Q_i$
\end{definition}


These two statements collectively state that given a proposed occupancy map, \emph{any grid rectangle in the domain of the occupancy map marked as 0 is indeed obstacle free, and if it is marked as 1 then there is indeed an obstacle at least partially in it.}

%We define $\mathit{Adj}_{q_{ij}}$ as the set $\{q_{i^\prime j^\prime}\mid i^\prime \in \{i, i+1,i-1\}\wedge j^\prime \in \{j,j+1,j-1\}\wedge q_{i^\prime j^\prime} \neq q_{ij}\}$


A vacuously correct (sound) solution to $\mapprob$ given $D, \qfunc$ and $\world_Q$ is $\forall i \in [N]$, $$\map_i : Q_i \mapsto \left\{0,1\right\}, Q_i = \phi$$ To allow for potentially more interesting solutions than the one stated above, we assume that we are given that initially, each robot $i\in[N]$ starts at a grid rectangle with no obstacle, the sensed area of each robot $i$ is non empty
$\forall i \in [N], \world_Q(q^0_i) = 0\wedge \sensarea(q^0_i) \neq \phi$
where $q^0_i = \qfunc(\pos_0(i))$, and $\pos_0(i)$ denotes the initial position of robot $i$.


\begin{definition}
    Given  $i , j \in [N]$, two proposed occupancy mappings $\map_i: Q_i\mapsto\left\{0,1\right\}$ and $\map_j: Q_j\mapsto \left\{0,1\right\}$, are consistent only if $\forall q \in Q_i \cup Q_j, \map_i(q) = \map_j(q)$.
\end{definition}

    Given $\map_i$, $\map_j$, and $q\in Q_i \cup Q_j$ , let $\map_i(q) = 1$, and $\map_j(q) = 0$. Suppose $\map_i$ is sound, then $\world_Q(q) = 1$, which implies $\map_j$ is not sound. By the same argument, if $\map_j$ is sound, $\map_i$ is not. Each mapping in a set of proposed mappings $\left\{\map_i\right\}_{i\in [N]}$ can only be sound if they are pairwise consistent.

Given a mapping $\map_i$, $\mathit{dom}(\map_i)$ denotes $ Q_i \subset Q$, such that $\forall q \in Q_i \map_i(q) = 0 \vee \map_i(q) = 1$.
\begin{definition}
    \label{cons}
Consider a set of sound mappings $\left\{\map_i\right\}_{i\in[N]}$. The \emph{ combined mapping} described by $\map: Q^\prime \mapsto \left\{0,1\right\}$ where $Q^\prime = \bigcup_{i\in[N]} \mathit{dom}(\map_i)$ , and $\exists j \in [N], q\in \mathit{dom}(\map_j)\Rightarrow \map(q) = \map_j(q)$ is also sound.
\end{definition}

$\map$ is sound by construction, and consistency of the sound mappings it is constructed from.
