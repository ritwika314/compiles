\section{Formal modeling and analysis}
\label{sec:formal}
In this section, we present the formal definition of the distributed mapping problem. Then we develop a state machine model of the \dmap application executing in the physical environment, and finally, using this model we show that the it meet the key requirements of the mapping problem. One of the outcomes of this analysis is the identification of a list of precise
 assumptions of sensors, CyPhyHouse middleware implementation, quantization, and \sayan{fill in}, that need to be checked separately for obtaining end-to-end, system-level guaranteed.

\subsection{Formal definition of mapping problem}
\label{sec:prelims}

\begin{figure*}[t]
\newcommand{\grid}[2]{\ensuremath{\langle#1,#2\rangle}}

\tikzstyle{grid map} = [
scale=0.4,
every node/.style = {font=\scriptsize,align=center},
]

\newcommand\irregularcircle[2]{% radius, irregularity
    \pgfextra {\pgfmathsetmacro\len{(#1)+rand*(#2)}}
    +(0:\len pt)
    \foreach \a in {10,20,...,350}{
        \pgfextra {\pgfmathsetmacro\len{(#1)+rand*(#2)}}
        -- +(\a:\len pt)
    } -- cycle
}

\pgfmathsetseed{1138}

\begin{tikzpicture}[scale=0.5]
\draw[step=1, gray, very thin] (-5.5,-5.5) grid (5.5,5.5); 

\draw[thick,->] (-6,-5) -- (6,-5) coordinate (x axis);
\draw[thick,->] (-5,-6) -- (-5,6) coordinate (y axis);

\draw[thick] (-5, -5) rectangle ++ (10, 10);

\coordinate (obs1) at (-2.1,1);
\fill[darkgray ,rounded corners=1mm] (obs1) \irregularcircle{2cm}{0.5cm};
\coordinate (obs2) at (2.3,-3);
\fill[darkgray, rounded corners=1mm] (obs2) \irregularcircle{1.5cm}{0.2cm};

\fill[gray, pattern=north west lines] % quantize obs1
       (0,0)
    -- (0,1)
    -- (1,1)
    -- (1,2)
    -- (0,2)
    -- (0, 3)
    -- (-1,3)
    -- (-1,4)
    -- (-3,4)
    -- (-3,3)
    -- (-5,3)
    -- (-5,-1)
    -- (-2,-1)
    -- (-2,-2)
    -- (-1,-2)
    -- (-1,-1)
    -- (0, -1)
    -- cycle;

\fill[gray, pattern=north west lines] % quantize obs2
       (1,-1)
    -- (1,-2)
    -- (0,-2)
    -- (0,-4)
    -- (1,-4)
    -- (1,-5)
    -- (4,-5)
    -- (4,-1)
    -- cycle;

\end{tikzpicture}
\begin{tikzpicture}[grid map]

% Globally unknown area
\filldraw[lightgray]
(-5, 0) rectangle ++ (4, 5)
(1, -5) rectangle ++ (4, 3)
(-1, 3) rectangle ++ (6, 2)
;

\begin{scope}
    \draw[step=1, gray, very thin] (-5.2,-5.2) grid (5.5,5.5);
    \draw[thick,->] (-6,-5) -- (6,-5) node (x axis) [right] {$x$};
    \draw[thick,->] (-5,-6) -- (-5,6) node (y axis) [above] {$y$};
    \fill[gray, pattern=dots] % quantize obs1
    (0,0)
    -- (0,1)
    -- (1,1)
    -- (1,2)
    -- (0,2)
    -- (0, 3)
    -- (-1,3)
    -- (-1,4)
    -- (-3,4)
    -- (-3,3)
    -- (-5,3)
    -- (-5,-1)
    -- (-2,-1)
    -- (-2,-2)
    -- (-1,-2)
    -- (-1,-1)
    -- (0, -1)
    -- cycle;

    \fill[gray, pattern=dots] % quantize obs2
    (1,-1)
    -- (1,-2)
    -- (0,-2)
    -- (0,-4)
    -- (1,-4)
    -- (1,-5)
    -- (4,-5)
    -- (4,-1)
    -- cycle;
\end{scope}


% robot 1
\draw[thick] (-5, -5) rectangle ++ (6, 5);
\node[fill=white, draw, rectangle] at (-2, -2.5) {bot1};
\fill[gray, pattern=north west lines] % detected obs
    (-2,-2)
 -- (-1,-2)
 -- (-1,-1)
 -- ( 0,-1)
 -- ( 0, 0)
 -- (-5, 0)
 -- (-5,-1)
 -- (-2,-1)
 -- cycle
 
    ( 0,-4)
 -- ( 1,-4)
 -- ( 1,-2)
 -- ( 0,-2)
 -- cycle
;

% robot 2
\draw[thick] (-1, -2) rectangle ++ (6, 5);
\node[fill=white, draw, rectangle] at (2, 0.5) {bot2};
\fill[gray, pattern=north east lines] % detected obs
   (-1,-1)
-- ( 0,-1)
-- ( 0, 1)
-- ( 1, 1)
-- ( 1, 2)
-- ( 0, 2)
-- ( 0, 3)
-- (-1, 3)
-- cycle

   ( 1,-2)
-- ( 4,-2)
-- ( 4,-1)
-- ( 1,-1)
-- cycle
;

\fill[gray, pattern=north east lines] (0,-1) rectangle ++ (1, 1); % unsound and inconsistent
\draw[densely dotted]
    (-0.5, -0.5) -- (-0.8, -5.5) node {\grid{-0.5}{-0.5} is consistent}
;
\draw[densely dotted]
    (0.5, -0.5) -- (0.8, 6) node [above] {\grid{0.5}{-0.5} of bot2 is unsound and inconsistent}
;


\end{tikzpicture}

\caption{$D = [0.0,10.0]^2 \subset \mathbb{R}^2$, $Q=\{0.5, \cdots, 9.5\}^2 \subset \mathbb{Q}^2$, for example, $\world_Q(\grid{5.5}{0.5}) = 0$ and $\world_Q(\grid{6.5}{0.5}) = 1$}
\end{figure*}

\paragraph{Notations.}
$\mathbb{R}, \mathbb{Q},\mathbb{N}$, and $\mathbb{B}$ denote the sets of real, rational, natural, and boolean numbers.
For any $N \in \mathbb{N}$, $[N]$ is the set $\{1,2,\ldots,N\}$.
%
Given any function $f:A \rightarrow S$, we denote the {\em domain\/} of $s$ as $\domain(f) = A$.
%
Given a subset $A' \subseteq \domain(f)$,
    the restriction of $f$ to $A'$, is written as $f \lceil A'$, and it is defined as the function $g:A' \rightarrow S$ such that for all $a \in A'$, $g(a) = f(a)$.

\paragraph{Mapping problem.}
The mapping problem is defined in terms of the following parameters.
\begin{enumerate}
	\item A positive integer $N$ which is the number of participating robotxs or robots. We will assume that the number and identity of robotxs is known. foe convenience, here we assume that the robots have unique identifiers from the set $[N]$.
	\item A domain $D$ which is bounded rectangle $[a_1,a_2]\times [b_1,b_2]$ in $\mathbb{R}^2$ corresponding to the physical arena.
	\item A \emph{ground truth} function $\world : D\mapsto \left\{0,1\right\}$ that gives the actual  occupancy of obstacles in this arena. That is, $\forall \Vec{x} \in D$,
\begin{align}
\world(\Vec{x}) =
		\left\{
		\begin{array}{ll}
			1 & \mathit{if}\ \Vec{x} \ \mathit{is \ occupied}\\
			0 & \mathit{otherwise}.
		\end{array}
		\right.
\end{align}
	\item A set $\qdom \subseteq D \cap \mathbb{Q}^2$ which is a quantized representation of $D$.
	\item A Koord\fTBD{should we say koord here, considering that it is a general statment} program variable, let us call it $\mathit{map}$, which is a shared view of the map for the system of robots $N$. Assume that each robot $i \in [N]$ has a quantized view of the map, let us call it $\map_i$. $\map$ is constructed by combining the views $\map_i$ in some way.
\end{enumerate}
%%
%%
For the sake of specificity, we assume that $\qdom$ is a $(n_x\times n_y)$-grid representation of $D$ for some resolution constants $n_x,n_y \in \mathbb{N}$. That is, $\qdom = \{q_{ij} \in \mathbb{Q}^2\}_{i\in [1..n_x], j\in [1..n_y]}$ such that every $q_{ij}$ uniquely represents a disjoint $[x_i, x_{i+1}] \times [y_j, y_{j+1}]$ in $D$.
%
%\begin{definition}
    The \emph{quantization function}  $\qfunc:D \mapsto \qdom$ maps points in $D$ to their quantized versions. That is $\qfunc(\Vec{x}) = q_{ij}$ iff $\Vec{x} \in [x_i, x_{i+1}] \times [y_j, y_{j+1}]$.
The inverse is defined accordingly: $\qinv(q_{ij}) =  [x_i, x_{i+1}] \times [y_j, y_{j+1}]$ for any $q_{ij} \in Q$.
%\end{definition}
%
The quantization defines a quantized version of the world $\world_Q : \qdom \mapsto \left\{0,1\right\}$, where
$$\world_Q(q) = \begin{cases}
        1\ \Leftrightarrow \exists \Vec{x}\in \qinv(q), \world(\Vec{x}) = 1, \\
        0\ \mbox{otherwise}.
\end{cases}
$$
%

For each robot $i$,  program variable $\mathit{map}_i$ will ideally store a quantized restriction of $\world$. That is, the type of this variable will be $\mathit{map}_i: Q' \rightarrow \mathbb{B}$, for some $Q' \subseteq Q$. The robots \emph{collaboratively} construct a mapping $\mathit{map}_i: Q'' \rightarrow \mathbb{B}$, such that $Q'' = \bigcup_{i\in[N]} \domain(\map_i)$, stored in a \emph{shared} program variable $\map$.


%
Finally, we will assume that each robot $i$ occupies space within the arena $D$ and it's position is available from a sensor port $\mathit{pos}_i$ which takes values in $D$.

Now we can formally state the desirable requirements of the mapping application.
\begin{enumerate}
	\item {\em (Sound)\/} Always, the shared variable $\map$ is a quantized restriction of the ground truth. That is, $\map: Q' \mapsto\mathbb{B}$, where $Q'\subseteq Q$ and $\map(q) = \world_Q(q)$ for every $q \in Q'$.
 \item {\em (Individually sound)\/} Always, each robot's map is a quantized restriction of the ground truth. That is, $\map_i: Q_i \mapsto \mathbb{B}$, where $Q_i\subseteq Q$ and  $\map_i(q) = \world_Q(q)$ for every $q\in Q_i$.
	\item {\em (Consistent)\/} Always, the local robot maps are consistent. That is, for any two robots $i,j \in [N]$,  $\map_i(q) = map_j(q)$  for any  $q\in \domain(\map_i) \cap \domain(map_j)$.
	\item {\em (Safe location)\/} Always, each robot is located in a part of the arena that is known to be free.
	That is, for any  $i \in [N]$,  $\qfunc(\pos_i) = map_i(q)$  for any  $q\in \domain(\map_i) \cap \domain(map_j)$.
	\item {\em (Eventually complete)} Eventually, the constructed maps cover $\world$. That is, for each robot $i\in [N], \domain(\map_i) = Q$.
\end{enumerate}

%\marginpar{\scriptsize\sayan{Say that we will prove 1,2 and why 3 is in general hard world may be partitioned. Path planning might hard. Quantization may be too coarse for finding safe paths...}}

