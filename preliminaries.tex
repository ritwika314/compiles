\section{Koord Software Stack}
\label{sec:software}

\subsection{Runtime system}



To run a $\lgname$ program (hardware or simulation), the user has to provide a configuration file, with
\begin{inparaenum}
    \item the number of agents,
    \item in case of simulation, the initial positions of the agents and the length of the simulation and
    \item in case of hardware deployment their IP addresses,
    and the localization system.
\end{inparaenum}

\subsection{Key environment assumptions}


\subsubsection{Periodic event execution semantics}


\subsubsection{Shared variable implementation over message passing}


\subsubsection{Known set of participants}
\subsubsection{Portability and heterogeneity}


\subsection{Simulator}
\subsubsection{gazebo environment}
\subsubsection{car model}
\subsubsection{lidar}
\subsubsection{positioning}
\subsubsection{sampled sensing}
\subsubsection{synchronization issues}

\section{Formal modeling and analysis}
\label{sec:formal}

In this section, we present the formal definition of the distributed mapping problem. Then we develop a state machine model of the \dmap application executing in the physical environment, and finally, using this model we show that the it meet the key requirements of the mapping problem. One of the outcomes of this analysis is the identification of a list of precise 
 assumptions of sensors, CyPhyHouse middleware implementation, quantization, and \sayan{fill in}, that need to be checked separately for obtaining end-to-end, system-level guaranteed. 
 
\subsection{Formal definition of mapping problem}
\label{sec:prelims}

\paragraph{Notations.}
$\mathbb{R}, \mathbb{Q},\mathbb{N}$, and $\mathbb{B}$ denote the sets of real, rational, natural, and boolean numbers.
For any $N \in \mathbb{N}$, $[N]$ is the set $\{1,2,\ldots,N\}$.
%
Given any function $f:A \rightarrow S$, we denote the {\em domain\/} of $s$ as $\domain(f) = A$.
%
Given a subset $A' \subseteq \domain(f)$, 
    the restriction of $f$ to $A'$, is written as $f \lceil A'$, and it is defined as the function $g:A' \rightarrow S$ such that for all $a \in A'$, $g(a) = f(a)$.

\paragraph{Mapping problem.}
The mapping problem is defined in terms of the following parameters.
\begin{enumerate}
	\item A positive integer $N$ which is the number of participating agents or robots. We will assume that the number and identity of agents is known. foe convenience, here we assume that the robots have unique identifiers from the set $[N]$.
	\item A domain $D$ which is bounded rectangle $[a_1,a_2]\times [b_1,b_2]$ in $\mathbb{R}^2$ corresponding to the physical arena.
	\item A \emph{ground truth} function $\world : D\mapsto \left\{0,1\right\}$ that gives the actual  occupancy of obstacles in this arena. That is, $\forall \Vec{x} \in D$, 
\begin{align}
\world(\Vec{x}) = 
		\left\{
		\begin{array}{ll}
			1 & \mathit{if}\ \Vec{x} \ \mathit{is \ occupied}\\
			0 & \mathit{otherwise}.
		\end{array}
		\right.
\end{align}
	\item A set $\qdom \subseteq D \cap \mathbb{Q}^2$ which is a quantized representation of $D$.
	\item A Koord program variable, let us call it $\mathit{map}_i$, for each robot $i \in [N]$, that stores the $\qdom$-quantized, shared map.
\end{enumerate}
%%
%%
For the sake of specificity, we assume that $\qdom$ is a $(n_x\times n_y)$-grid representation of $D$ for some resolution constants $n_x,n_y \in \mathbb{N}$. That is, $\qdom = \{q_{ij} \in \mathbb{Q}^2\}_{i\in [1..n_x], j\in [1..n_y]}$ such that every $q_{ij}$ uniquely represents a disjoint $[x_i, x_{i+1}] \times [y_j, y_{j+1}]$ in $D$.
%
%\begin{definition}
    The \emph{quantization function}  $\qfunc:D \mapsto \qdom$ maps points in $D$ to their quantized versions. That is $\qfunc(\Vec{x}) = q_{ij}$ iff $\Vec{x} \in [x_i, x_{i+1}] \times [y_j, y_{j+1}]$.
The inverse is defined accordingly: $\qinv(q_{ij}) =  [x_i, x_{i+1}] \times [y_j, y_{j+1}]$ for any $q_{ij} \in Q$. 
%\end{definition}
%
The quantization defines a quantized version of the world $\world_Q : \qdom \mapsto \left\{0,1\right\}$, where
$$\world_Q(q) = \begin{cases}
        1\ \Leftrightarrow \exists \Vec{x}\in \qinv(q), \world(\Vec{x}) = 1, \\
        0\ \mbox{otherwise}.
\end{cases}
$$
%

For each robot $i$, the program variable $\mathit{map}_i$ will ideally store a qunantized restriction of $\world$. That is, the type of this variable will be $\mathit{map}_i: Q' \rightarrow \mathbb{B}$, for some $Q' \subseteq Q$. 

Now we can formally state the desirable requirements of the mapping application.
\begin{enumerate}
	\item {\em (Individually sound)} Always, each robot's map is a quantized restriction of the ground truth. That is, $\map_i: Q_i \mapsto \mathbb{B}$, where $Q_i\subseteq Q$ and  $\map_i(q) = \world_Q(q)$ for every $q\in Q_i$.
	\item {\em (Consistent)\/} Always, robot maps are consistent. That is, for any two robots $i,j \in [N]$,   $\map_i(q) = map_j(q)$  for any  $q\in \domain(\map_i)  \cap \domain(map_j)$.
	\item {\em (Eventual completeness)} Eventually, the constructed maps cover $\world$. That is, for each robot $i\in [N], \domain(\map_i) = Q$.
\end{enumerate}


\subsection{Assumptions}
\label{sec:formal:sensing}

\begin{definition}
   Given a robot at a position $\Vec{x}\in D, \qfunc{\Vec{x}} = q$, we define the \emph{sensing area} of $q$ follows: $\sensarea: Q \mapsto 2^{Q}$, such that  $\forall q^\prime \in \sensarea(q), \sensfunc(q) = \world_Q(q)$.  
\end{definition}




Let $\map_i:Q\mapsto \left\{-1,0\right\}$ denote a \qdfunc \emph{local} to robot $i$, which we can see as a software state for a given robot $i$ as well.  % $\map_i(q) = 1$ indicates that according to robot $i$, there is an obstacle in $q$, $\map_i(q) = 0$ indicates that according to robot $i$, $q$ is unoccupied, and $\map_i(q) = -1$ indicates that robot $i$ doesn't have information about $q$.



We assume that given $\Vec{x} \in q, \forall \Vec{x^\prime} \in q, \sensarea(\Vec{x}) \subseteq \sensarea(\Vec{x^\prime})$. We can now state the 2-d distributed mapping problem, $\mapprob$ as follows. \begin{quote}
{\em Given a set of robots $[N]$ , for each robot $i \in [N]$ construct an \emph{occupancy map}, $\map_i: Q_i \mapsto \left\{1,0\right\}$, $Q_i\subseteq Q$.
}
\end{quote}

% \rg{We can \emph{combine} the elements of the set of \emph{local maps}, $\{\map_i\}_{i\in [N]}$ to form a \emph{global} occupancy mapping. }

Having stated the problem, we now define the notion of soundness of a proposed occupancy map.
\begin{definition}
    \label{soundness}
   
\end{definition}


These two statements collectively state that given a proposed occupancy map, \emph{any grid rectangle in the domain of the occupancy map marked as 0 is indeed obstacle free, and if it is marked as 1 then there is indeed an obstacle at least partially in it.}

%We define $\mathit{Adj}_{q_{ij}}$ as the set $\{q_{i^\prime j^\prime}\mid i^\prime \in \{i, i+1,i-1\}\wedge j^\prime \in \{j,j+1,j-1\}\wedge q_{i^\prime j^\prime} \neq q_{ij}\}$


A vacuously correct (sound) solution to $\mapprob$ given $D, \qfunc$ and $\world_Q$ is $\forall i \in [N]$, $$\map_i : Q_i \mapsto \left\{0,1\right\}, Q_i = \phi$$ To allow for potentially more interesting solutions than the one stated above, we assume that we are given that initially, each robot $i\in[N]$ starts at a grid rectangle with no obstacle, the sensed area of each robot $i$ is non empty
$\forall i \in [N], \world_Q(q^0_i) = 0\wedge \sensarea(q^0_i) \neq \phi$
where $q^0_i = \qfunc(\pos_0(i))$, and $\pos_0(i)$ denotes the initial position of robot $i$.


\begin{definition}
    Given  $i , j \in [N]$, two proposed occupancy mappings $\map_i: Q_i\mapsto\left\{0,1\right\}$ and $\map_j: Q_j\mapsto \left\{0,1\right\}$, are consistent only if $\forall q \in Q_i \cup Q_j, \map_i(q) = \map_j(q)$.
\end{definition}

    Given $\map_i$, $\map_j$, and $q\in Q_i \cup Q_j$ , let $\map_i(q) = 1$, and $\map_j(q) = 0$. Suppose $\map_i$ is sound, then $\world_Q(q) = 1$, which implies $\map_j$ is not sound. By the same argument, if $\map_j$ is sound, $\map_i$ is not. Each mapping in a set of proposed mappings $\left\{\map_i\right\}_{i\in [N]}$ can only be sound if they are pairwise consistent.

Given a mapping $\map_i$, $\mathit{dom}(\map_i)$ denotes $ Q_i \subset Q$, such that $\forall q \in Q_i \map_i(q) = 0 \vee \map_i(q) = 1$.
\begin{definition}
    \label{cons}
Consider a set of sound mappings $\left\{\map_i\right\}_{i\in[N]}$. The \emph{ combined mapping} described by $\map: Q^\prime \mapsto \left\{0,1\right\}$ where $Q^\prime = \bigcup_{i\in[N]} \mathit{dom}(\map_i)$ , and $\exists j \in [N], q\in \mathit{dom}(\map_j)\Rightarrow \map(q) = \map_j(q)$ is also sound.
\end{definition}

$\map$ is sound by construction, and consistency of the sound mappings it is constructed from.
