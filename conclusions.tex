\subsection{Conclusion}
\label{sec:conclusion}

The divide between theory and implementation concerns of cyberphysical systems is an obstacle in development of portable, reusable and safe applications. Programming language research in this area can present a unified approach to bridging the aforementioned divide. We have presented  an event-triggered language for coordination and control, and implemented its formal executable semantics. The shared memory based communication model we used enables us to naturally code protocols for robotics and distributed systems. Our work is a move in the direction of providing users without programming expertise with an ability to safely program distributed applications.  The $\lgname$ language uses abstractions for components like path planning, localization, and shared memory, to simplify code and separate concerns. We also discussed how varying relevant system parameters affect their analysis.

The $\lgname$ semantics model is realistic by design to deploy on hardware. Our shared memory semantics are informed by the actual implementation using message passing protocols, the language and platform interfaces are used to define the configuration elements on which rewrite rules are specified. Conversely, the formal analysis driven by the semantics guides the requirements upon the hardware as well, narrowing the gap between theory and implementation of the language semantics. 

 $\lgname$ and supporting compiler can  enable users to develop and run distributed robotics applications in a platform independent fashion; $\lgname$ programs can be ported across platforms automatically by linking with the right platform dependent libraries; and the \emph{Simulator} and \kbmc\ tools can be used to simulate and verify the same application code. 
