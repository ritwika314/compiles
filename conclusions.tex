\section{Conclusions and Future Work}


Our vision is to provide a programming methodology to enable programming safe distributed cyberphysical applications without the need of complete domain expertise in all related areas such as control theory, robotics motion control, and network protocols.
To this end, we demonstrated how \lgname application developers can write succinct multi-robot applications involving distributed coordination,
different types of sensing and actuation, and path planning in three case studies, each of which requires only preliminary knowledge
in shared memory and basic concurrency control via atomic blocks and assumptions on sensor and actuator ports of controllers.
V\&V engineers with deep understanding in distributed computing are able to focus on formally analyzing invariant properties of \lgname programs via symbolic execution, and roboticists can validate the feasibility of assumptions by examining rigorously defined proof obligations.

We acknowledge the fact that our \portasum based abstractions may not cover various vastly different types of robots.
\K semantics framework can allow us to extend our language to tailor to specific robot types on demand
while retaining the same framework for formal analysis. We also plan to extend this work to include specification and verification of progress properties under fairness constraints for \lgname applications.
