
\documentclass[sigconf]{acmart}
\def\BibTeX{{\rm B\kern-.05em{\sc i\kern-.025em b}\kern-.08emT\kern-.1667em\lower.7ex\hbox{E}\kern-.125emX}}
\copyrightyear{2018}
\acmYear{2018}
\setcopyright{acmlicensed}
\acmConference[]{}
\acmBooktitle{}
\acmPrice{15.00}
\acmDOI{10.1145/1122445.1122456}
\acmISBN{978-1-4503-9999-9/18/06}

\usepackage{caption}
\DeclareCaptionFont{white}{\color{white}}
\DeclareCaptionFormat{listing}{%
  \parbox{\textwidth}{\colorbox{gray}{\parbox{\textwidth}{#1#2#3}}\vskip-4pt}}
\captionsetup[lstlisting]{format=listing,labelfont=white,textfont=white}


\usepackage{amssymb}
%\usepackage{bm}
\usepackage{mdframed}
\usepackage{lipsum}
\usepackage{amsfonts}
\usepackage{paralist}
\usepackage{enumerate}
\usepackage{enumitem}
\usepackage{bbding}
\usepackage{pifont}
%\usepackage[fleqn]{amsmath}
\usepackage{fancyvrb}
\usepackage{wrapfig}
\usepackage{subfig}
\usepackage{color}
\usepackage[T1]{fontenc}% http://ctan.org/pkg/fontenc
\usepackage[outline]{contour}% http://ctan.org/pkg/contour
\usepackage{xcolor}% http://ctan.org/pkg/xcolor
\usepackage{amsmath}
\usepackage{mathtools}
\DeclarePairedDelimiter{\floor}{\lfloor}{\rfloor}
\usepackage{hyperref}
\hypersetup{
	colorlinks=true,
	citecolor={blue},
	linkcolor = {blue},
	bookmarksopen=false,
	bookmarksnumbered=true
}


\usepackage[mathscr]{euscript}
\usepackage{nameref}
% Package to generate and customize Algorithm as per ACM style
\usepackage[vlined]{algorithm2e}
\usepackage{adjustbox}
\usepackage{mathpartir,xparse}
\usepackage{bbm}

\usepackage{listings}
%\SetAlFnt{\algofont}
%\SetAlCapFnt{\algofont}
%\SetAlCapNameFnt{\algofont}
%\SetAlCapHSkip{0pt}
%\IncMargin{-\parindent}
%\renewcommand{\algorithmcfname}{ALGORITHM}
\definecolor{light-blue}{rgb}{0.4,0,0.9}
\newcommand{\starl}[1]{\textcolor{light-blue}{#1}}
\makeatletter % allow us to mention @-commands
\def\arcr{\@arraycr}
\makeatother
\newcommand{\smnew}[1]{\textcolor{blue}{#1}}
\usepackage{wasysym}
\usepackage{verbatim}
\usepackage{epstopdf}
\usepackage{tikz}
\usetikzlibrary{calc,patterns,decorations.pathmorphing,decorations.markings,shapes, arrows}



% General
\newcommand{\tb}[1]{\textbf{#1}}
\newtheorem{invariant}{Invariant}
\newtheorem{assumption}{Assumption}

\newcommand{\A}{\mathcal{A}}
\newcommand{\D}{\mathcal{D}}
\newcommand{\vs}{{\textbf{s}}}

\newcommand{\w}[1]{\ensuremath{\textit{#1}}}
\newcommand{\m}[1]{\ensuremath{\texttt{#1}}}
\newcommand{\s}[1]{\ensuremath{\textsf{#1}}}
\newcommand{\p}[1]{\ensuremath{\left(#1\right)}}
\newcommand{\pl}[1]{\ensuremath{\left\langle#1\right\rangle}}
\newcommand{\OR}{\mbox{ }|\mbox{ }}
\newcommand{\NMAX}{\texttt{N\_SYS}\xspace}
\newcommand{\st}{.\mbox{ }}
\newcommand{\finto}{\ensuremath{\stackrel{\mathtt{fin}}{\longrightarrow}}}
\newcommand{\defeq}{\ensuremath{\stackrel{\mathtt{def}}{=}}}
\newcommand{\cond}[1]{\ensuremath{\left\{\begin{array}{ll} #1 \end{array}\right.}}
\newcommand{\lst}[1]{\begin{itemize} {#1} \end{itemize}}
\newcommand{\pby}[1]{\hspace*{\fill}{#1}}
\newcommand{\slide}[2][]{ \begin{frame} \frametitle{#1} {#2} \end{frame} }
\newcommand{\etal}{\textit{et al.}\xspace}
\newcommand{\MW}{\sf{allwrite}}
\newcommand{\SW}{\sf{allread}}
\newcommand{\lgname}{\emph{Koord}\xspace}
\newcommand{\toolname}{\emph{CyPhyHouse}\xspace}
\newcommand{\appname}{\emph{AppName}\xspace}
\newcommand{\UINS}{\texttt{ID}\xspace}
\newcommand{\kbmc}{\emph{KoordBMC}}
\newcommand{\kiic}{\emph{KoordProver}}
\newcommand{\ksem}{\emph{Koord Semantics}}
\newcommand{\sympost}{\emph{Symbolic Post Generation}}
\newcommand{\myuin}{\texttt{pid}\xspace}
\newcommand{\Var}{\mathit{Var}\xspace}
\newcommand{\Port}{\mathit{Port}\xspace}
\newcommand{\Event}{\mathit{Events}}

\newcommand{\Val}{\mathit{Val}\xspace}
\newcommand{\Cfield}{\mathit{CPorts}\xspace}

\newcommand{\cName}{\mathit{cName}\xspace}
\newcommand{\dmap}{{\sf DMap}\xspace}
\newcommand{\domain}{\mathit{domain}}

\newcommand{\End}{\mathit{End}\xspace}

\newcommand{\Post}{\mathit{Post}\xspace}
\newcommand{\Final}{\mathit{Final}\xspace}
\newcommand{\traj}{\mathit{traj}\xspace}
\newcommand{\pt}[2]{\mathit{Pt}_{[#1,#2]}}
\newcommand{\ft}[1]{\mathit{Pt}_{#1}}
%evaluation property semantics
\newcommand{\eval}{\mathit{eval}}
\newcommand{\sat}{\rightarrow_\mathit{sat}}
\newcommand{\frontier}{\mathit{Fr}}
\newcommand{\Reach}{\mathit{Reach}}
% For references
\newcommand{\fig}[1]{Figure~\ref{#1}}
\newcommand{\lem}[1]{Lemma~\ref{#1}}
\newcommand{\asum}[1]{Assumption~\ref{#1}}
\newcommand{\portasum}{port assumption}
\newcommand{\inv}[1]{Invariant~\ref{#1}}
\newcommand{\theo}[1]{Theorem~\ref{#1}}
\newcommand{\coro}[1]{Corollary~\ref{#1}}
\newcommand{\defn}[1]{Definition~\ref{#1}}
\newcommand{\rmrk}[1]{Remark~\ref{#1}}
\newcommand{\myexample}[1]{Example~\ref{#1}}
\newcommand{\sect}[1]{$\S$~\ref{#1}}
\newcommand{\refsect}[1]{Section~\ref{sec:#1}}
\newcommand{\reffig}[1]{Figure~\ref{fig:#1}}
\newcommand{\refalg}[1]{Algorithm~\ref{#1}}

\newcommand{\ev}{\mathit{ev}}
\definecolor{orange}{rgb}{1,0.5,0}
\definecolor{darkgreen}{rgb}{0.0, 0.5, 0.0}
\newcommand{\logicarg}[2]{% \logicarg{<premise>}{<conclusion>}
  \begin{tabular}[t]{@{}c@{}}
    #1 \\ \hline #2
  \end{tabular}%
}
\newcommand{\sayan}[1]{\textcolor{blue}{Sayan: #1}}
\newcommand{\rg}[1]{\textcolor{red}{Ritwika: #1}}
\newcommand{\chiao}[1]{\textcolor{green}{Chiao: #1}}
\newcommand{\two}[4]{
	\parbox{.98\columnwidth}{\vspace{1pt} \vfill
		\parbox[t]{#1\columnwidth}{#3}%
		\hspace{.5cm}
		\parbox[t]{#2\columnwidth}{#4}%
	}}
\newcommand{\three}[5]{
	\parbox{.98\columnwidth}{\vspace{1pt} \vfill
		\parbox[t]{#1\columnwidth}{#3}%
		\hspace{.5cm}
		\parbox[t]{#2\columnwidth}{#4}%\vfill
				\hspace{.5cm}
		\parbox[t]{#2\columnwidth}{#5}%
	}}
	
\lstdefinelanguage{Koord}{
    basicstyle=\scriptsize,
    identifierstyle=\it,
    mathescape=true,
    tabsize=20,
    sensitive=false,
    columns=fullflexible,
    keepspaces=true,
    flexiblecolumns=true,
    basewidth=0.05em,
    moredelim=[il][\rm]{//},
    moredelim=[is][\bf ]{*}{*},
    keywordstyle=[1]\bf,
    keywords=[1]{
        and, agent, actuator, actuators, allread, allwrite, atomic, assume,
        break,
        def,
        eff, else, elseif, end,
        for, foreach, forget,
        if, import, in, init, input, internal, invariant,
        local,
        module,
        or,
        pre,
        return,
        sensor, sensors, stop,
        then, type, thread, to,
        using,
        variables, vocabulary,
        when, where, while, with},
    keywordstyle=[2]\it\color{olive},
    keywords=[2]{
        bool,
        int,
        list,
        pathplanner, point,
        set,
        task},
    keywordstyle=[3]\tt\color{brown},
    keywords=[3]{
        false,
        len,
        pid,
        true},
    literate=
    {(}{{$($}}1
    {)}{{$)$}}1
    % LaTeX math symbols
    {\\in}{{$\in\ $}}1
    {\\preceq}{{$\preceq\ $}}1
    {\\subset}{{$\subset\ $}}1
    {\\subseteq}{{$\subseteq\ $}}1
    {\\supset}{{$\supset\ $}}1
    {\\supseteq}{{$\supseteq\ $}}1
    {\\forall}{{$\forall$}}1
    {\\le}{{$\le\ $}}1
    {\\ge}{{$\ge\ $}}1
    {\\gets}{{$\gets\ $}}1
    {\\cup}{{$\cup\ $}}1
    {\\cap}{{$\cap\ $}}1
    {\\langle}{{$\langle$}}1
    {\\rangle}{{$\rangle$}}1
    {\\exists}{{$\exists\ $}}1
    {\\bot}{{$\bot$}}1
    {\\rip}{{$\rip$}}1
    {\\emptyset}{{$\emptyset$}}1
    {\\notin}{{$\notin\ $}}1
    {\\not\\exists}{{$\not\exists\ $}}1
    {\\ne}{{$\ne\ $}}1
    {\\to}{{$\to\ $}}1
    {\\implies}{{$\implies\ $}}1
    % LSL symbols (one-character)
    {<*>}{{$\langle*\rangle$}}1
    {=}{{$=$}}1
    {~}{{$\neg\ $}}1
    {|}{{$\mid$}}1
    {'}{{$^\prime$}}1
    % LSL symbols (two characters)
    {\\A}{{$\forall\ $}}1
    {\\E}{{$\exists\ $}}1
    {\\/}{{$\vee\,$}}1
    {\\vee}{{$\vee\,$}}1
    {/\\}{{$\wedge\,$}}1
    {\\wedge}{{$\wedge\,$}}1
    {=>}{{$\Rightarrow\ $}}1
    {->}{{$\rightarrow\ $}}1
    {<=}{{$\ \leq\ $}}1
    {<-}{{$\leftarrow\ $}}1
    {==}{{$=\mathrel{\mkern-3mu}=$}}2
    {<=}{{$\leq$}}1
    {>=}{{$\geq$}}1
    {~=}{{$\neq\ $}}1
    {\\U}{{$\cup\ $}}1
    {\\I}{{$\cap\ $}}1
    {|-}{{$\vdash\ $}}1
    {-|}{{$\dashv\ $}}1
    {<<}{{$\ll\ $}}2
    {>>}{{$\gg\ $}}2
    {||}{{$\|$}}1
    {\\<}{{$\langle$}}1
    {\\>}{{$\rangle$}}1
    {[}{{$[$}}1
    {]}{{$]$}}1
    {<=>}{{$\Leftrightarrow\ $}}2
    {<->}{{$\leftrightarrow\ $}}2
    {(+)}{{$\oplus\ $}}1
    {(-)}{{$\ominus\ $}}1
    {_i}{{$_{i}$}}1
    {_j}{{$_{j}$}}1
    {_{i,j}}{{$_{i,j}$}}3
    {_{j,i}}{{$_{j,i}$}}3
    {_0}{{$_0$}}1
    {_1}{{$_1$}}1
    {_2}{{$_2$}}1
    {_n}{{$_n$}}1
    {_k}{{$_n$}}1
    %        {-}{{$\ms{-}$}}1
    {@}{{}}0
    {\\delta}{{$\delta$}}1
    {\\R}{{$\R$}}1
    {\\Rplus}{{$\Rplus$}}1
    {\\N}{{$\N$}}1
    {\\times}{{$\times\ $}}1
    {\\tau}{{$\tau$}}1
    {\\alpha}{{$\alpha$}}1
    {\\beta}{{$\beta$}}1
    {\\gamma}{{$\gamma$}}1
    {\\ell}{{$\ell\ $}}1
    {--}{{$-\ $}}1
    {\\TT}{{\hspace{1.5em}}}3
}
	
	
	\lstdefinelanguage{NumKoord}[]{Koord}
	{
		numbers=left,
		numbersep=15pt,
		numberstyle=\tiny,
		stepnumber=1,
		numbersep=4pt,
	}
	
	\lstdefinelanguage{KoordNum}[]{Koord}
	{
		numbers=right,
		numbersep=15pt,
		numberstyle=\tiny,
		stepnumber=1,
		numbersep=4pt,
	}


% numbers sets

\newcommand{\nnt}{{\sf T}^{\geq 0}}             %nonnegative time points
\newcommand{\post}{{\sf T}^{>0}}                %positive time points
\newcommand{\Variables}{{\sf V}}                %variables

\newcommand{\num}[1]{\relax\ifmmode \mathbb #1\else $\mathbb #1$\fi}
\newcommand{\nnnum}[1]{\relax\ifmmode 
	{\mathbb #1}_{\geq 0} \else ${\mathbb #1}_{\geq 0}$
	\fi}
\newcommand{\npnum}[1]{\relax\ifmmode 
	{\mathbb #1}_{\leq 0} \else ${\mathbb #1}_{\leq 0}$
	\fi}
\newcommand{\pnum}[1]{\relax\ifmmode 
	{\mathbb #1}_{> 0} \else ${\mathbb #1}_{> 0}$
	\fi}
\newcommand{\nnum}[1]{\relax\ifmmode 
	{\mathbb #1}_{< 0} \else ${\mathbb #1}_{< 0}$
	\fi}
\newcommand{\plnum}[1]{\relax\ifmmode 
	{\mathbb #1}_{+} \else ${\mathbb #1}_{+}$
	\fi}
\newcommand{\nenum}[1]{\relax\ifmmode 
	{\mathbb #1}_{-} \else ${\mathbb #1}_{-}$
	\fi}
\newcommand{\K}{$\mathbb{K}\ $}
\newcommand{\reals}{{\num R}}                    %reals
\newcommand{\booleans}{{\num B}}                    %reals
\newcommand{\nnreals}{{\nnnum R}}                    %nonnegative reals
\newcommand{\realsinfty}{{\num R} \cup \{\infty, -\infty\}}                    %nonnegative reals
\newcommand{\plreals}{{\plnum R}}                    %positive reals
\newcommand{\naturals}{{\num N}}                      %natural numbers
\newcommand{\integers}{{\num Z}}                      %integers
\newcommand{\rationals}{{\num Q}}                      %rationals
\newcommand{\nnrationals}{{\nnnum Q}}                   % nonnegative rationals
\newcommand{\Time}{{\num T}}  



%% Sasa, commenting commands:

\definecolor{WowColor}{rgb}{.75,0,.75}
\definecolor{SubtleColor}{rgb}{0,0,.50}
\definecolor{TODOColor}{rgb}{0,.50,0}

% inline
\newcommand{\NA}[1]{\textcolor{SubtleColor}{ {\tiny \bf ($\star$)} #1}}
\newcommand{\TODO}[1]{\textcolor{TODOColor}{ {\tiny \bf TODO:} #1}}
\newcommand{\LATER}[1]{\textcolor{SubtleColor}{ {\tiny \bf ($\dagger$)} #1}}
\newcommand{\TBD}[1]{\textcolor{SubtleColor}{ {\tiny \bf (!)} #1}}
\newcommand{\PROBLEM}[1]{\textcolor{WowColor}{ {\bf (!!)} {\bf #1}}}
\newcommand{\ST}[1]{ \textcolor{SubtleColor}{ {\tiny \bf (!!)} \sout{#1} } }

\newcommand{\fTBD}[1]{\textcolor{SubtleColor}{$\,^{(\incdisplaycounter)}$}\marginpar{\tiny\textcolor{SubtleColor}{ {\tiny $(\displaycounter)$} #1}}}
\newcommand{\fTODO}[1]{\textcolor{TODOColor}{$\,^{(\incdisplaycounter)}$}\marginpar{\tiny\textcolor{TODOColor}{ {\tiny TODO $(\displaycounter)$}:  #1}}}
\newcommand{\fPROBLEM}[1]{\textcolor{WowColor}{$\,^{((\incdisplaycounter))}$}\marginpar{\tiny\textcolor{WowColor}{ {\bf $\mathbf{((\displaycounter))}$} {\bf #1}}}}
\newcommand{\fLATER}[1]{\textcolor{SubtleColor}{$\,^{(\incdisplaycounter\dagger)}$}\marginpar{\tiny\textcolor{SubtleColor}{ {\tiny $(\displaycounter\dagger)$} #1}}}

%semantic macros
\newcommand{\pid}{\mathit{pid}}
\newcommand{\en}{\mathit{en}}
\newcommand{\ur}{\mathit{ur}}
\newcommand{\cp}{\mathit{cp}}
\newcommand{\turn}{\mathit{turn}}
\newcommand{\gconfig}{\mathit{\boldsymbol{c}}} %global config
\newcommand{\gset}{\mathcal{C}} %set of global configs
\newcommand{\agnt}{L}
\newcommand{\lconfig}[1]{\agnt_{#1}} %configuration of agent i
\newcommand{\lset}{\{\lconfig{i}\}} %set of agent configurations
\newcommand{\EvalExpr}[1]{\llbracket#1\rrbracket_{S, \agnt}}
\newcommand{\eec}[2]{\llbracket#1\rrbracket_{#2}}

\newcommand{\pwg}{\mathbb{C}} %all possible global configs, (power set of)
\newcommand{\pwl}{\mathbb{L}} %all possible local configs
\newcommand{\pwe}{\mathbb{E}} %all possible expressions.
\newcommand{\pws}{\mathbb{S}} %all possible global context
\newcommand{\pwstmt}{\mathit{Stmt}} %all possible statements

% as margin notes
\newcounter{margincounter}
\newcommand{\displaycounter}{{\arabic{margincounter}}}
\newcommand{\incdisplaycounter}{{\stepcounter{margincounter}\arabic{margincounter}}}
\newcommand{\env}{\mathtt{env}}
\newcommand{\prog}{\mathtt{prog}}

\newcommand{\mytitle}{Working Title}
\newcommand{\qfunc}{\mathit{quant}}
\newcommand{\qinv}{\mathit{quant}^{-1}}
\newcommand{\qdom}{\mathit{Q}}
\newcommand{\map}{\mathit{map}}
\newcommand{\world}{\mathit{world}}
\newcommand{\pos}{\mathit{pos}}
\newcommand{\mapprob}{\mathit{Map2D}}
\newcommand{\sensarea}{\mathit{sa}}
\newcommand{\lmap}{\mathit{localMap}}
\newcommand{\gmap}{\mathit{map}}
\newcommand{\ff}{\mathit{frontier}}
\newcommand{\qdfunc}{quantized domain function}
\newcommand{\sensfunc}{\mathit{scanToMap}}
\newcommand{\rmap}{\mathit{ws}}
\newcommand{\Access}{\mathit{access}}
\newcommand{\ps}{\mathit{ps}}
\newcommand{\pre}{\mathit{pre}}
\newcommand{\pvec}{\vec{p}}
\newcommand{\pseq}{(p_i)}
\newcommand{\pinit}{p_0}
\newcommand{\incurly}[1]{\left\{#1\right\}}
\newcommand{\pathvar}{\mathit{shared\_path}}


% For code listings
\lstset{frame=lrb,xleftmargin=\fboxsep,xrightmargin=-\fboxsep}
\DeclareCaptionFont{white}{\color{white}}
\DeclareCaptionFormat{listing}{%
  \parbox{\textwidth}{\colorbox{gray}{\parbox{0.478\textwidth}{#1#2#3}}\vskip-4pt}}
\captionsetup[lstlisting]{format=listing,labelfont=white,textfont=white}
\lstset{frame=lrb,xleftmargin=\fboxsep,xrightmargin=-\fboxsep,language=koordNums}

% end of the preamble, start of the body of the document source.
\begin{document}

%
% The "title" command has an optional parameter, allowing the author to define a "short title" to be used in page headers.
\title{Programming and formally analyzing a distributed coordination and mapping system using CyPhyHouse}



%
% The abstract is a short summary of the work to be presented in the article.
\begin{abstract}
We present the design and implementation of a  new programming  language for distributed cyberphysical systems (CPS) called $\lgname$ that separates the platform-independent  computations from platform-specific concerns of sensing, communication, and low-level  control. $\lgname$ raises the level of abstraction in programming CPS by providing distributed shared memory for coordination; it allows parameterized controllers for portability across different physical platforms, and it is based on a familiar  event-triggered concurrent programming model. We present the formal semantics of $\lgname$ in the \K executable semantic framework. This  narrows the gap between the theoretical model of $\lgname$  and its implementation. We  present an implementation of $\lgname$ that generates executable Python code, which has been deployed on robotic platforms.  Finally, we  introduce a high-fidelity simulator and show how it can be used to test, visualize, and debug large deployments of $\lgname$ applications.  
%The same application code can  be verified through bounded model-checking using a verification tool. Finally, 
We illustrate the various features and the design decisions involved through applications such as platooning, formation flight, and distributed task allocation. 
\end{abstract}

%
% The code below is generated by the tool at http://dl.acm.org/ccs.cfm.
% Please copy and paste the code instead of the example below.
%
%\begin{CCSXML}
%<ccs2012>
% <concept>
%  <concept_id>10010520.10010553.10010562</concept_id>
%  <concept_desc>Computer systems organization~Embedded systems</concept_desc>
%  <concept_significance>500</concept_significance>
% </concept>
% <concept>
%  <concept_id>10010520.10010575.10010755</concept_id>
%  <concept_desc>Computer systems organization~Redundancy</concept_desc>
%  <concept_significance>300</concept_significance>
% </concept>
% <concept>
%  <concept_id>10010520.10010553.10010554</concept_id>
%  <concept_desc>Computer systems organization~Robotics</concept_desc>
%  <concept_significance>100</concept_significance>
% </concept>
% <concept>
%  <concept_id>10003033.10003083.10003095</concept_id>
%  <concept_desc>Networks~Network reliability</concept_desc>
%  <concept_significance>100</concept_significance>
% </concept>
%</ccs2012>
%\end{CCSXML}
%
%\ccsdesc[500]{Computer systems organization~Embedded systems}
%\ccsdesc[300]{Computer systems organization~Redundancy}
%\ccsdesc{Computer systems organization~Robotics}
%\ccsdesc[100]{Networks~Network reliability}

\keywords{mobile robotics, swarms, simulation, semantics}


\maketitle


\section{Introduction}
\label{sec:intro}

\paragraph{Motivation}
Programming systems that interact with physical processes are central in  robotics, manufacturing, IoT, and other cyberphysical systems. Typical domain specific languages (DSL) for these systems are platform specific and they combine  low-level sensing, communication, and control tasks with the application-level logic~\cite{nordmann2014robotics}. This  tight-coupling of applications with platform-specific details can hinder development, modularity, portability, code reuse, and testing. 

This need for raising the level of abstraction and separating the \emph{platform-independent} from the \emph{platform-dependent} concerns motivates our work. For example, consider an application for distributed package delivery with mobile robots in a building. The higher-level coordination has to deal with assigning robots to visit way-points in different rooms, load-balancing, and handling failures.  For all of the reasons mentioned above, these coordination tasks {\em should be\/} separated from the  steering control of the individual vehicles, indoor positioning, and message-level communication protocols tied to the specific hardware platforms. 

Current programming languages do not natively support such abstractions for programming distributed cyberphysical systems.  The developer will have to keep track of communication between different sensors, actuators, and program variables. The {\em robotic operating system (ROS)}~\cite{rosbridge_suite,ros} does provide useful libraries and  publish-subscribe messaging standards, but these are not integrated in a programming abstraction. Similarly, coordination across multiple robots has to be built from the ground up using message passing threads. In order to simulate and test the application, one has to create different threads or instances for the developed code corresponding to different agents, and on top of that, the agent programs have to be harnessed in a physical world simulator with proper sensor and actuator models. In order to analyze correctness, one has to first define the semantics of the system. This  is again challenging  as  distributed cyber-physical systems have concurrency and data flows in multiple time-scales. 
%A further source of concern is the gap between the system model with iand the executable code that is actually . 

An ongoing, multi-year, project at our lab aims to address some of these challenges by designing, developing, and evaluating an open source programming system for distributed robotics. \footnote{Information about this project and the links to downloadable software will be made available to the reviewers upon request. We omit the information here for the sake of maintaining anonymity.} In this paper, we present the design and the implementation of the  {\em $\lgname$ language} and the supporting software tools, namely the compiler, the \K executable semantics, and the simulator. This forms the  core of this programming system. $\lgname$ programs have been deployed on F1/10 vehicles, quadcopter, and roombas, and the hardware deployment will be presented in a future article. 





%
%\begin{figure}[h!]
%\centering
%\includegraphics[width=0.48\textwidth]{figs/arch.png}
%\caption{\small CyPhyHouse framework. Major tools shown in blue.}
%\label{fig:arch}
%\end{figure}
%
\subsection{Contributions}
\subsubsection{Design and implementation of $\lgname$ language}
{\em We present a clean-slate design and implementation of an event-driven programming language, for distributed cyber-physical systems, namely $\lgname$.} 
%
$\lgname$ combines distributed shared memory abstraction for coordination across agents and a synchronous model of communication with the physical environment through sensors and actuators, all packed in a familiar precondition-effect style language.
%

Consider a simple line-formation program in which a set of $N$ robots form a equi-spaced line starting from arbitrary positions. The elementary algorithm is for each robot $i$ to repeatedly move towards the midpoint of the line joining the positions of $i-1$ and $i+1$. The extremal robots with ids $0$ and $N-1$ stay fixed. This is a archetypal protocol for synchronization, pattern formation, and consensus~\cite{Tsitsiklis:1986,Blondel,Magnusbook2010,Fax} in distributed robotics.
%
% \belowcaptionskip=-10pt
\begin{lstlisting}[label=lineform,caption=Lineform $\lgname$ program]
   using Motion:
      sensors pos psn
      actuators target
   allread: pos x $\label{lineformp}$
   TargetUpdate:
      pre True
      eff if not(pid == $\NMAX$ - 1 or pid == 0):
         Motion.target = mid(<x[pid+1],x[pid-1]>)
         x[pid] = Motion.psn
\end{lstlisting}

The above $10$ line  $\lgname$ program implements this line formation algorithm and can be simulated or deployed (see Figure~\ref{fig:shapeformplots1}). Among these $10$ lines of code, lines 1-5 import the {\em controller module\/} called {\em Motion\/} that enables this program to access the sensor port with the agent's position $(\mathit{psn})$ and the actuator port $(\mathit{target})$. An implementation of  a {\em controller module\/} has platform-specific path planners and low-level controllers for moving specific types of robots in the physical world. {\em Thus, with different implementations of the {\em Motion\/} interface the $\mathit{Lineform}$ $\lgname$  program can be ported to different platforms\/}. 

Another powerful feature of $\lgname$ used in the $\mathit{Lineform}$ program is the single-write multi-reader ({\bf allread}) shared array $x$, in which component $x[i]$ records the position of the $i^{th}$ robot in each round.  Shared variables allow the participating agents programs to coordinate without exposing the programmer to explicit message passing channels, buffers, and threads. This makes $\lgname$ programs succinct, readable, and often, as in the  above example, remarkably close to the textbook version of the algorithm. The $\lgname$ runtime system (discussed in more detail in Section~\ref{sec:software}) implements the propagation of  the shared variable write over messages. The formalization of the resulting semantics is discussed in Section~\ref{sec:K}.

%
The one and only event in the $\mathit{Lineform}$ program is $\mathit{TargetUpdate}$ (lines 5-9): it updates the target position of the agent  to be the midpoint of its neighbors.  Figure~\ref{fig:shapeformplots} shows the result of simulating a slightly modified version of $\mathit{Lineform}$ on the $\lgname$ simulator with $25$ robots forming a 3D-shape. The $\lgname$ compiler can also generate executables that can be deployed on several mobile robotic hardware platforms such as F1/10 cars~\cite{f1-10}, drones~\cite{}, and Roombas. 
%\renewcommand{\lstinputlisting}[1][]{\oldlstinputlisting[frame=lines,#1]}
 
%\begin{figure}[ht!]
%	\label{fig:lineform}
%	\noindent
%	\begin{center}
%		\scriptsize
%		\two{0.4}{0.6}
%		{\lstinputlisting[language=xyzNums,frame=lines]{code/lineform.tex}}
%	\par        
%	\end{center}
%	\caption{\small $\lgname$ program for line formation ({\em Left}) and its mathematical counterpart in robotics and control textbooks ({\em Right}).}
%\end{figure}

%The shared \emph{allread} variable $p$ (Line\ref{lineformp}) is used by the agents to communicate their position to the other robots. The function \emph{midpoint} is a part of the library functions provided for the data type \emph{pos}; where $$\mathit{midpoint}(p_1,p_2,\ldots,p_n) = pos(\frac{\Sigma_n p_i.x}{n},\frac{\Sigma_n p_i.y}{n},\frac{\Sigma_n p_i.z}{n}) $$.
%

%We have implemented a compiler for $\lgname$ that generates executable Python programs that can be either simulated in a discrete event simulator (discussed below) or 

\subsubsection{First formalization of a robotics or CPS language in \K}
% motivation
Implementations of programming languages usually contain many bugs. One common type of bugs creeps in from the gap between the official semantics of the language (as in a user manual) and the its implementation as embodied in a compiler. The \K framework~\cite{Kf} closes this gap by allowing the specification of {\em  formal executable semantics of any programming language} in terms of rewrite rules. These rewrite rules define how each and every possible statement in the  language changes, possibly nondeterministically, the ``state of the machine'' executing the program. 
%
\K  provided the most-complete-to-date formalization  of the C language~\cite{KC}. This semantics was tested against the GCC torture test suite and it successfully passed 99.2\% of 776 test programs~\cite{chuckythesis}. Recently, \K  was used to provide a formal semantics of x86-64~\cite{rusuadvepaper} and the Ethereum Virtual Machine (EVM) bytecode. 


%\K  using computations over states or configurations, and rewrite rules for said configurations. \K has several additional features including non deterministic execution, underspecification and explicit read-only and write only specifications for rewrite rules which makes \K suitable for defining concurrent and control intensive languages. 
{\em In order to close the above-mentioned gap, we have developed the complete executable formal semantics of $\lgname$ in \K.} This is the first formalization of a robotics or CPS language in \K;  and the first multi-agent language. \sayan{One challenge here is to define the ``state of the machine'' executing $\lgname$ programs. This state now has to include not only the states of the program variables of the different agents, but also the sensor and actuator ports, and the shared variables.}

\sayan{
Our key innovation for achieving this formalization is to parameterize the semantics with a set of {\em environment functions\/} that model the behavior of the physical environment, the controller, and other operations that are external to the application program.  The environment functions are called by the  back-end of \K to update certain state components (e.g., sensor and actuator ports). Thus, the formal semantics of $\lgname$ is also modular and portable across  different environment models. }

%ontroller behaviors with path-planners to the back-end of \K to specify rewrite rules for them, as the $\lgname$ language itself lets the user use predefined controller modules.
In addition, the shared memory semantics of $\lgname$ is implemented over  message passing.. \sayan{So what is challenging interesting in what we have done here? Mention synchronous round-by-round execution. Events. Shared memory propagation. }


Our \K implementation of $\lgname$ enables us to execute  programs \emph{semantically accurately}. So we can to design test suites for any implementation of the language semantics. 
\sayan{This is weak, as we have not done this.}
\sayan{Mention nondeterminism?}
To support formal analysis in case of non-specific initial states, we also implemented an interval arithmetic in \K, and support execution of $\lgname$ programs where the initial values of variables are provided in intervals. We implemented a explicit state bounded-model-checking  tool on top of the \K semantics to provide this additional feature. \sayan{What are the main conclusions of this part? Give pointer to section ahead.}

\sayan{Demonstration of feasibility?} 
\sayan{copied from before:
Design and development of the $\lgname$ programming language and the supporting $\mathbb{K}$-based~\cite{Kpaper} verification tool \kbmc\, are discussed here for the sake of completeness; those details will appear elsewhere~\cite{koordreport}.}

\subsubsection{$\lgname$ simulator and applications}

{\em We have developed a high-fidelity, multi-threaded simulator for $\lgname$ applications (\reffig{simulator}). } The simulator can execute instances with 50+ agents with heterogeneous dynamics, executing $\lgname$ applications, and therefore, is a powerful tool for debugging and performance analysis.  

The major challenges involved in designing a simulator were to create a simulation engine that exactly simulates the behavior expected on hardware deployment. 
\sayan{But this is on the surface an impossible task.}
To that end, the simulator uses motion automata, which can be provided any programmable dynamics to simulate the robot dynamics. The simulated sensed data can also be also provided realistic noises and dynamics to mimic actual system dynamics. The communication protocols to propagate shared memory messages between agents are also a part of the simulator communication module. The time between two computational steps (or discrete event loop iterations) is used to propagate messages to implement shared memory as in the actual hardware stack.

The simulator enables the user to test their discrete event loop with simple motion models to test and debug the application program logic without incurring the cost of hardware deployment in case of buggy programs. The simulator also serves as a visualization tool as it can be used to plot the behavior of any program variables, or controller variables. For instance, we implemented a robot \emph{formation} app in $\lgname$, where several robots form a shape in which they are evenly distributed.

\begin{figure}[h!]
\begin{minipage}{0.5\textwidth}
\includegraphics[width=.5\textwidth]{figs/formation1.png}\hfill
\includegraphics[width=.5\textwidth]{figs/formation4.png}\hfill%
%\includegraphics[width=.3\textwidth]{figs/Platooning_2.png}}\hfill%
\end{minipage}%
\caption{Screenshot of $\lgname$ simulator visualizing positions of a swarm of $25$ agents running a slightly modified version of $\mathit{Lineform}$ application forming a 3D-shape in space.}
\label{fig:shapeplots}
\end{figure}




%\subsubsection{Other by-products}


%Design and development of the CyPhyHouse open source software system. This includes a discrete event simulator for distributed robotic systems, the application launcher, the run-time logging and monitoring system, and an integrated indoor positioning system. All of these software tools are integrated with our new robot programming language called $\lgname$ and its compiler. 





%Design and development of the Koord programming language and the supporting verification tool KoordBMC~\cite{koordreport}--- significant, related but separate efforts---are not contributions of the current paper; we discuss their usage for the sake of completeness. 
% completely describing the framework. 
% the Demonstration of an example application development using CyPhyHouse tools and deployment on a physical system using multiple quadcopters.
%Non-contributions: Spell these out  to avoid misdirected criticisms and conflict with overlapping publications.
%\begin{itemize} 
%\item Language design
%\item Verification tools.
%\item Low-level controller design for vehicles.
%\end{itemize} 

%\begin{figure}[h!]
%\centering
%\includegraphics[width=0.45\textwidth]{figs/exp_traces.png}
%\caption{\small Experimental run in our testbed. The traces show the path of each robot for the last $2$ seconds. }
%\label{fig:exp_traces}
%\end{figure}

%
%%\footnote{\href{https://cyphyhouse.github.io/index.html}{https://cyphyhouse.github.io}}: an open source software framework for programming, rapid deployment, and testing of distributed robotics applications. 
%\sayan{The high-level $\lgname$ language enables users to write succinct distributed coordination applications without getting bogged down by messaging and thread management issues (See Examples in \reffig{lineform} and \reffig{taskapp}).}
%Using the CyPhyHouse framework around $\lgname$, a user can code, compile, launch, and run applications in a highly-automated fashion (\reffig{arch}). 
%\sayan{The framework has been built over three years and has more than 100k lines of source code.}


    
    \section{Related work}
    \label{sec:related}
    \sayan{Modern programming languages like C\# and Swift, and   compiler infrastructures like LLVM~\cite{llvm} have revolutionized the application development ecosystem in mobile computing.
%\paragraph*{D.} 
Inspired by these successes, there is a surge of interest in open and portable languages that raise the level of abstraction~\cite{Buzzlanguage,Bohrer:2018:VVC:3192366.3192406,reactlang,williams2003model} (For an earlier survey of Domain Specific Programming Languages (DSLs) for robotic systems see~\cite{Nordmann2014}. Most of these older languages are proprietary or generate executable files that are tied to specific platforms)}. 
%
Buzz~\cite{Buzzlanguage} and React~\cite{reactlang} fall in this category as does our language $\lgname$. 
The Live Robot Programming language~\cite{campusanofabry:lrp2016} not only provides a higher-level programming abstraction in terms of nested state machines, but also allows the program to be changed while running, hence reducing the feedback loop across writing, compiling, and testing of robot programs. 
%The goals of React language for robotics aligns with our goals~\cite{react-lang}
Buzz currently does not  connect with  verification tools, and the verification approach implemented with React uses precise models of the environment and performs model checking using dReal~\cite{Gao2013}. 
In contrast, our environment implementations allow for imprecise models, with a verification approach using sensitivity analysis~\cite{DryVR2017}.
%Our approach is also similar in spirit to the Reactive Model-based Programming Language (RMPL)
%~\cite{williams2003model}.
%
%There is been more recent development of domain specific languages for general cyber-physical systems (CPS)~\cite{pradhan2015chariot}. The main challenge addressed in this line of work is in supporting reconfiguration of complex, heterogeneous software components, for handling failures. 
%
%There has also been work on programming abstractions for coordinating CPS~\cite{distCPSSri,Bundle}. 
%A group-based abstraction that facilitates dynamic creation of logical collections of sensors and actuators is presented in~\cite{Bundle}. 
%
%
%%React reactive robot programming language~\cite{DogmusEP15}.
%% 
The Robotarium project provides remote end-to-end access to a  multi-robot research facility, but not languages and development tools~\cite{robotarium}. 
The  VeriPhy project~\cite{Bohrer:2018:VVC:3192366.3192406} shares a similar goal to CyPhyHouse; however, instead of a programming language, the starting point is differential dynamic logic~\cite{Bohrer:2017:FVD:3018610.3018616},  and there are significant differences  in the underlying verification engines used (KeYmaera X, HOL instead of K, Z3, DryVR).
%

``Correct-by-construction'' synthesis from high-level temporal logic specifications has been applied to mobile robotic systems (see, for example~\cite{kress2009temporal,kloetzer2008fully,wongpiromsarn2010receding,wongpiromsarn2011tulip,ulusoy2013optimality}).
% Many of these approaches have been applied to mobile robotic systems. 
\sayan{Our point of view on automating robot programming is different in that we expect that the programmer's creativity and efforts will be necessary well beyond writing high-level specs in solving distributed robotics problems; consequently only the tedious and standard steps in coordination and control are automated using the $\lgname$ compiler.}

%. A
%correct-by-construction synthesis algorithm takes as input a high-level requirement (for example, ``from room A to B and see if you find a chair'') to generate robot programs for accomplishing
%this task. In our approach, 

%\paragraph*{Languages for distributed shared memory systems}
%
Programming systems using the  shared memory paradigm have been developed for several distributed computing systems~\cite{dsm1991,Adve96sharedmemory,Azure,Cassandra,Dynamo}.
Specifically, P~\cite{Planguage}  and PSync~\cite{PSyncLanguage} are DSLs for  asynchronous partially  distributed systems, but cyber-physical interactions are not supported. 
%DSM has also been proposed as a programming model in the context of wireless networks~\cite{hcs,rs}. 
%These  programming models are defined mathematically in terms of state machines or in terms of APIs, and are  typically not embodied in a programming language with carefully designed syntax and semantics to enforce the models. 
$\lgname$  provides a distributed shared variable abstraction for coordinating multiple agents. 
%\sayan{This enables users to write succinct, textbook-like programs without struggling with messy code for message handling and thread management}. 
The consistency semantics implemented here are that the writes to shared variables are propagated to all the agents, and become visible to other agents reading the variable after one round ($\delta$ time units).
In the fault-free synchronous model considered here a gossip algorithm-based is used to implement this semantics in the $\lgname$ runtime system. 
% The framework of~\cite{Hotline_CPS_srivastava} supports shared memory over multi-hop wireless networks, with a consistency model analogous to {\em release} consistency.  
%

%
%\paragraph*{Uncertainty and Robotics Abstractions}
%$\lambda_O$~\cite{park2005probabilistic} is a probabilistic programming language in which sampling methods are used to specify probability distributions, while expressing and reasoning about these methods formally. It finds application in robot localization and mapping. In the same vein, $\mathit{Uncertain}\langle T\rangle$~\cite{bornholt2014uncertain} provides a programming language abstraction for uncertain data. It is a departure from previous probabilistic programming languages in the wide range of developers it serves, as opposed to being accessible only by experts. The language provides abstractions and semantics for uncertain data, like sensed information about location, temperature, etc. While $\lgname$ does not currently perform reasoning involving uncertainty in sensor readings or agent localization currently, these are realistic concerns that can be explored by exploiting the extensibility of the $\lgname$ semantics implemented in \K. While these languages provide semantics for uncertainity in robot abstractions and sensing issues, they do not provide distributed application design capabilities. 
%\sayan{I did not find much about this. Formal verification of mobile robot protocols: the DVE language, which is the input format of the model-checkers DiVinE and ITS tools, and formally prove the equivalence of the two models.}
%\item  
%Buzz, a novel programming language for heterogeneous robot 
%swarms. Buzz advocates a compositional approach, offering primitives to define swarm 
%behaviors both from the perspective of the single robot and of the overall swarm. 
%
%\item 

%Voltron programming system to explore the concept of team-level programming in active sensing applications. Voltron offers programming constructs to create the illusion of a simple sequential execution model while still maximizing opportunities to dynamically re-task the drones as needed. We implement Voltron by targeting a popular aerial drone platform, and evaluate the resulting system using a combination of real deployments, user studies, and emulation. Our results indicate that Voltron enables simpler code and produces marginal overhead in terms of CPU, memory, and network utilization. In addition, it greatly facilitates implementing correct and complete collaborative drone applications, compared to existing drone programming systems. (?) 
%\end{enumerate}


\section{Overview and an example}
\label{sec:overview}

%\chiao{1\textasciitilde1.5 pages}
\newcommand{\LineForm}{\textsf{LineForm}\xspace}

We will discuss the key features of the \lgname language and programming system with an example.
The \lgname application \LineForm shown in \reffig{lineform} implements a simple formation control protocol of the type  used for drone shows like the one seen in \reffig{firefly}.
\LineForm makes an arbitrary number of robots (drones) line up uniformly between two extremal robots.
Small modifications to the code make the drones form other shapes like squares, cubes, and stars.

\subsection{\lgname language}
\label{sec:koord-language}
% 1 sentence intro of the language
\lgname is a high-level, event-driven language in which application programs use \emph{shared variables} for coordination across robots
and \emph{ports} for interacting with hardware-specific subroutines.
In a distributed robotics setting, instances of the same \lgname program is executed by each participating robots to solve problems collectively.

\begin{figure}[h!]
    \two{0.32}{0.59}
    {
        \lstinputlisting[language=NumKoord, lastline=8]{code/lineform.tex}
    }
    {
        \lstinputlisting[language=NumKoord, firstline=9, firstnumber=9]{code/lineform.tex}
    }
    \caption{\lgname program \LineForm for a set of robots to form a line.}
    \label{fig:lineform}
\end{figure}

\paragraph{Modules and port abstractions.}
A \lgname program interacts with the sensors and low-level controllers of the robot platform through \emph{sensor} and \emph{actuator} ports.
%
%
The program can read data from the sensor ports and can write data to actuator ports.
%
For example, \LineForm uses a \emph{module} (library) called \emph{Motion} which provides a sensor port called \emph{position} that publishes the robot's position, and an actuator port called \emph{target} for specifying a target position.
%
%A \lgname program can use several modules providing different sensor and actuator ports.
Thus, these  ports provide an abstraction over various possible sensor and controller implementations and environments.
%
\sayan{Implementations of the modules are part of the {\em Koord runtime system\/} and they implement hardware specific functions.}
%, the actuator ports of different modules can be used to provide input to the controllers,
%which drives the underlying physical plant and environment.

\sayan{For example, our implementation of the  \emph{Motion} module for a quadcopter uses an indoor camera based positioning system to update the \emph{position} port and it uses an RRT based~\cite{} path planner and motion controller.
%
The  \emph{Motion} module abstraction is implemented for a small racing vehicle platform using the same indoor positioning system but a different model-predictive controller.
}
%\sayan{Similarly, an implementation \emph{target}}

%
%For example the \emph{position} sensor port in \LineForm is periodically updated by the implemented \emph{Motion} module with the current position of the robot using  GPS or an indoor positioning system.
%
%\marginpar{\scriptsize\sayan{1. actuator ports ``can be'' used or ``are used'' ? How else could they be used? 2. The term controller is confusing there and in Fig 1.}}
%
%In short, different controllers in use would largely affect the behavior of each robot and its interaction with the environment.

\paragraph{Local and shared variables.}
 \lgname programs can have  \emph{local} variables similar to most programming languages.
%
%\marginpar{\scriptsize\sayan{What does ``system point of view'' mean?}}
%
%\paragraph{Shared variables and system parameters.}
In addition, they can also use \emph{shared} variables for participating robots to communicate with each other.
At Line~\ref{lineform-allread} in \reffig{lineform}, an \textbf{allread} variable, $x$, is a shared array which all robots can read from,  but each robot \myuin can only write to $x[\myuin]$.  This shared array is used to share the current position of each robot with all other robots. 
%There are several system-level \emph{constant} parameters that the program can use.
\LineForm uses
\begin{inparaenum}[(a)]
    \item the unique integer identifier \myuin for itself and
    \item the number \NMAX of all participating robots.
\end{inparaenum}
A detailed list of system parameters will be discussed later in \refsect{language}.


\sayan{$\lgname$ provides concurrency control with mutual exclusion and \textbf{atomic} blocks, for multiple robot programs writing to shared variables (see  {\sf Task} and {\sf Mapping} applications  in Section~\ref{}).}
%\chiao{Should we discuss about atomic blocks and mutual exclusion here.}
%\sayan{not in any detail. 1 sentence forward pointer.}
% Sayan: Not in detail 



%As a \lgname program is run on a system of robots,
%therefore, each participating robot would have its own set of module ports and local variables, along with a local copy of each  shared variable.
\marginpar{\scriptsize\sayan{Better move this in 4.}}

\subsection{Semantics and invariant properties}

We have developed the full semantics of \lgname using \K~\cite{rosu-serbanuta-2013-k}, and present the details in \refsect{sec:language}. The execution semantics of any applications for multi-robot system are complicated by issues of asynchrony, consistency of shared memory, and interactions between software and  the physical environment.
The \K rewriting engine makes the formal language semantics \emph{executable}, and enables exhaustive exploration of non-deterministic behaviors of \lgname applications.
In Section~\ref{}, we present a method for checking invariant properties for \lgname applications using symbolic execution of the $\lgname$ semantics in \K.

For \LineForm, a natural requirement is to restrict all the robots to stay within a certain safe area, at all times (Geofencing).
More precisely, given a (hyper)rectangle $\mathit{rect}(x_{min}, x_{max})$ defined by its two corners $x_{min}$ and $x_{max}$,
if all robots are initialized within the rectangle, then all robots should always stay in the rectangle. This requirement can be stated as:
%Formally, we would like to prove following invariant:
\begin{invariant}
\label{inv:lineform}
%$ \forall \myuin \in \UINS,\ M.pos_{\myuin} \in rect(x_{min}, x_{max}) \\$
%$ \ \ \ \ \  \land\ x[\myuin] \in rect(x_{min}, x_{max}),$\\
%
\[
\begin{split}
    \forall \myuin \in \UINS,\ M.pos_{\myuin} \in rect(x_{min}, x_{max}) \\
    \land\ x[\myuin] \in rect(x_{min}, x_{max})
\end{split}
\]
\normalfont{where $M.pos$ is the shorthand for $Motion.position$.}
\end{invariant}

\sayan{Using \lgname's supporting proof tools,   an invariant like the above can be established in two steps: first, assuming that all the robot positions are in $\mathit{rect}(x_{min}, x_{max})$, we show that the targets computed by  \LineForm are also in $\mathit{rect}(x_{min}, x_{max})$. 
The \K semantics of \lgname allows us to construct the symbolic post states of the  \emph{TargetUpdate} event and we can  automatically prove this using the $\lgname$ prover (\reffig{fig:tools}) as detailed in \refsect{verification}.
}

\sayan{The second step is to show,  that for any robot, assuming that the computed target are in $\mathit{rect}(x_{min}, x_{max})$, the controller implementing \emph{Motion} module indeed keeps the robot inside $\mathit{rect}(x_{min}, x_{max})$. 
} 
For this step, one has to reason about how each robot hardware moves when its implementation of the Motion module is given a target. In order to complete the proof of Invariant~\ref{inv:lineform}, we can state this as the following assumption with 
\begin{assumption}
\label{lineform-assume}
\[
\forall t \in [0, \delta], f(M.pos, M.tgt, t) \in rect(M.pos, M.tgt)
\]
\end{assumption}
\normalfont {where $M.tgt$ is the shorthand for $Motion.target$, 
$f$ is a function giving the position of the robot at time $t$, moving to $M.tgt$, from $M.pos$.}.
This assumption states that the robot's motion module should ensure that it is moving within the bounding rectangle between its position and target within the duration of a round.
%
%Given a black-box function $f$ returning positions of the device in continuous time,
%
In Section~\ref{}, we will discuss how these types of assumptions about the control system can be discharged using verification engines for reasoning about continuous behavior of dynamical systems.

%\chiao{Explain why we need assumptions over modules.}
%\sayan{Drop the extra line after Invariant~1. Why bother with the shorthand M.pos? M.tgt does not even appear in the invariant.}
%\sayan{Ok, you can state the invariant...then what? can it be proved automatically using K? We should say that then with the appropriate qualifiers, and give a forward pointer.}

%\chiao{Give the formula representing the symbolic post or transition relation of \emph{TargetUpdate}}
%\rg{Shouldnt we postpone this to the actual section?}
\begin{figure}
\begin{tikzpicture}[
    every node/.style={draw},
]
    \node (sym) {\K Symbolic Execution};
    \node [below of=sym] (prover) {\lgname prover};
    \node [diamond, aspect=2, below of=prover] (z3) {z3};
    \node [below left=1cm and 1cm of z3] (proven) {Proven};
    \node [below right=1cm and 1cm of z3] (incon) {Inconclusive};

    \draw [->] (sym) edge (prover)
               (prover) edge (z3)
               (z3) edge node[draw=none, left] {UNSAT} (proven)
               (z3) edge node[draw=none, right] {SAT} (incon)
               ;
\end{tikzpicture}
\caption{\K semantics based invariant checking for \lgname.}
\label{fig:tools}
\end{figure}
\marginpar{\scriptsize\sayan{Shouldn't this flowchart also have inputs Koord application code/requirement? Also, too much white space.}}

\subsection{Simulation based assumption validation}

\asum{lineform-assume} seems reasonable and sufficiently weak at the first glance,
but it actually imposes a very strong constraint over the $Motion$ module and the environment.
Using the simulation engine, we can monitor the sensor and actuator values to evaluate how often the assumption is violated.
For \asum{lineform-assume}, we can identify and demonstrate two common scenarios
where it is invalidated.
\chiao{We can use simulator screenshot to show the violation.}

First, if a robot has to avoid obstacles,
then it may have to go around the obstacle and hence out of the bound.
Second, the assumption simply fails over certain types of devices.
For example, four wheel cars doing parallel parking may easily go out of the rectangle
because the cars need more space to make turns.

In short, users can use the simulator to early detect whether
the assumptions for correctness are too strong under specific scenarios,
and revise the assumptions iteratively.
For example, to run the line formation program on cars,
a different module including the orientation of cars as well as more relaxed assumptions are needed.

\subsection{Compilation and Deployment.}
In addition to the formal language, semantics, analysis, and simulation,
our complete tool chain also includes compilation and deployment to heterogeneous platforms including drones and race cars.
Once developers install our ROS~\cite{ros} based run-time libraries~(middleware) on a platform
and provides a device specific configuration denoting the mapping from \lgname module ports
to low level sensor and actuator ROS messages,
our module port based abstraction then allows the same \lgname program to run on this platform.
Detailed description of our tool chain is available in~\cite{ghosh2019cyphyhouse}.

\section{Koord Language and Semantics}
\subsection{Koord model for distributed CPS}

The interaction model of an agent executing a $\lgname$ program is shown in \reffig{arch}. 
\sayan{This has to be rewritten. Too much unstructured text.}

\sayan{The notion of turn has to be introduced.}
$\lgname$ design comprises of a \emph{turn}-based alternating semantics of discrete and dynamic behavior of each agent in the distributed system. The agents can be viewed as performing \emph{rounds} of execution, where each round consists of a \begin{inparaenum} \item \emph{A program transition} which is a discrete computational step, and \item \emph{An environment transition} which is the dynamic behavior of the agents specified by the controllers \end{inparaenum}. In our semantics, we represent whether it is the turn of a program transition or an environment transition to be executed on each agent by a $\mathit{turn}$ variable.  

In a program transition in a round, each agent first nondeterministically chooses an \emph{enabled} event, or an event whose precondition evaluates to \verb|True|. It then executes the statements in the effect of that event. %This may involve reading sensor ports, performing computations using local and shared variables, and writing to actuator ports. 
Then, the agent performs the environment transition in the round, during which it interacts with the physical environment and behaves as dictated by the actuator ports on the controller for $\delta$ amount of time.  \footnote{ $\delta>0$ is a constant \emph{sampling period} parameter. The sampling parameter can be provided by the user, and each controller that $\lgname$ supports has a default sampling parameter which has been set based on the experiments using the controller. See \refsect{experiments} for more details.} %During this time, the program variables and the actuator port values remain constant, the state of the environment (controller and plant) and the values of the sensor ports change according to the dynamics of the environment.

\sayan{Moved from later.
The agent program specifies the controller which determines the dynamic behavior of each agent and its interaction with the environment. \reffig{arch} shows how  each controller has sensor ports and actuator ports which the agent programs use to interact with the controller and consequently, with the physical environment. The agent program can read from the sensor ports read and write to the actuator ports during the program transition. The sensor ports are updated during the environment transition according to the controller dynamics. 
}

Compared to $\delta$, the time taken by the agent to execute the computational step is negligible, and in our formalization we treat it as zero logical time. Because each program transition takes zero logical time, we can assume all the agents start and finish executing their program transitions at the same time, and their environment transitions at the same time respectively. As our design choices impose this synchronicity on all agent executions, the system can also be seen to be executing program and environment transition by \emph{turns}, where all the agents execute the environment transitions in a round only after all of them have finished executing the program transitions in that round.  

%Collectively, the (distributed) \emph{system} of all the agents executing this program itself can be captured by an alternating semantics of discrete computational and dynamic behavior.

 In the rest of this section, we present the formal semantics of the $\lgname$ programming language. The complete formalization of the semantics in \K is available from~\cite{} \sayan{webpage/github}. The syntax of the language is defined using a BNF grammar, provided in full for the interested reader in \rg{the appendix}.
 
\subsection{Program (cyber) variables and physical variables}
\label{sec:variables}
We first introduce the types of variables used by the $\lgname$ programs. Throughout, we refer to the $\mathit{Task}$ application as an exmaple.

\paragraph{Program variables}
%$\lgname$ provides several types of access for program variables. 
An agent's $\lgname$ program can access three types of variables. 
%
\begin{itemize}
	\item {\em Local program variables\/} record the state of the program. For example, the variable $p$ of type $\mathit{path}$ (Line~\ref{pathvar}) stores a path for the agent. 
\item {\em Distributed shared variables\/} are used for communication across agents.  For example, {\bf allwrite} $\mathit{taskList}$ (Line~\ref{awvar}) is a multi-writer list of tasks which can be written-to and read-from by all participating agents. Shared variables can also be single-writer multi-reader ({\bf allread}). 
\item {\em Port variables\/} are used to read from and write to sensor and actuator ports of the agent. For example, the $\mathit{Motion}$ controller for the $\mathit{Task}$ app  drives the agent  through a route, as directed by value set at the actuator port $\mathit{Motion.route}$. The sensor port $\mathit{Motion.psn}$ gives the position of the agent (in a fixed coordinate system) and $\mathit{Motion.done}$ indicates  whether the $\mathit{Motion}$ controller is active or inactive.
\end{itemize}
The valuation of these variables define the configuration for an agent. 
%Also, should 
The agent program also has access to two system-level constant parameters (a) a unique integer identifier $\myuin$ for itself and (b) a list $\UINS$ of identifiers of all participating agents\footnote{Our current system implementation assumes that the set of possible participating agents is known. This assumption will be relaxed in the future.}. 

\paragraph{Agent configurations}
The  syntax and grammar for $\lgname$ is given in the Appendix. 
Let us fix $P$ to be a syntactically correct agent program. Let $\Var$ be the set of program and shared variables in $P$. Let $\mathit{Sens}$ be the set of sensor ports of the controller used in $P$. Let $\mathit{Val}$ be the set possible valuations that these variables can take. 
%Given that $\mathbb{P}$ is the set of all the productions in the $\lgname$ grammar,  being used in $P$, 
A {\em configuration\/} of agent $\mathit{i}$, $i \in \UINS$, running $\lgname$ program $P$ is a tuple $ L_i = (\mathit{pc}, {M},\mathit{sm},\turn)$, where
\begin{itemize}
 \item $\mathit{pc}$ is a syntactic production that generates $P$. 
\item ${M} : \Var \mapsto \Val$ is a {\em local context\/} mapping each variable to a corresponding valuation.
 \item $\mathit{sm} : \mathit{Sens} \mapsto \Val$ is a mapping of each sensor port to a valuation.
 \item $\turn:\{\mathtt{prog,env}\}$ is a bookkeeping variable; $\turn = \mathtt{env}$ indicates that agent $i$ has finished executing its program for the current round.  
\end{itemize}
The components of an agent configuration in the tuple $L_i$ are accessed using the usual dot ($\cdot$) notation. That is, $L_i.M$ refers to the local context $M$ in agent configuration $L_i$, for the agent with pid $i$ in the system.

\subsection{System configurations}

We define the semantics of the overall system in terms of a nondeterministic, hybrid automaton~\cite{TIOAmon}. The state of the automaton is defined in terms of {\em configurations\/} consisting of the state components of the individual agents and the physical environment. 

A {\em system configuration\/} with sampling parameter $\delta>0$ is a tuple 
$\gconfig_\delta = (\lset,{S},\tau,\turn)$, where
\begin{itemize}
	\item ${L} = \{\lconfig{i}\}_{i\in\UINS}$ is a list of {\em agent configurations\/}; $\lconfig{i}$ is the  configuration of the $i^{\mathit{th}}$ agent.
	\item ${S} : \mathit{Var} \mapsto \mathit{Val}$ is the {\em global context\/}, which is a mapping of all shared variable names to their values, 
	\item  $\tau:\mathbb{R}^{\geq 0}$ is the {\em global time\/}.
	\item $\turn\in\{\mathtt{prog,env}\}$ is a binary bookkeeping variable which represents whether the system is performing a program or environment transition. 
\end{itemize}

%Bookkeeping variables are invisible in the language syntax, and only used in the semantics. We now define the \emph{agent configurations}, which are used to specify the semantics of each agent. \newline

\sayan{Stopping here. Several things undefined at this point. Koord grammar. Why is agent configuration $l$ and not $L_i$ as defined above?}

\subsection{Update rules}
We provide the semantics for some of the syntactic productions in $\lgname$, which will help us discuss the overall semantics of program and environment transitions. Program transitions are mainly comprised of execution of statements. The statement semantics are given by update rules of the type
$$\stmtrule\ \subseteq (\pws \times \pwl\times \mathit{Stmts} \cup \{\cdot\}) \times \mathscr{P}(\pws\times \pwl \times \mathit{Stmts} \cup \{\cdot\}),$$
%\fTBD{$\pwstmt$ is very clunky looking, need to change it perhaps?}
where $\mathit{Stmts}$ refers to the set of all possible statements allowed by language syntax. We introduce the symbol `$\cdot$' to indicate an ``empty" statement, which does not affect the configurations. The update relation takes as input a tuple of (1) an agent configuration, (2) a global context, and (3) a statement, and maps it to a set of such tuples. 
\subsubsection{Program variable/program statement updates}

The following rules display the variable lookup and assignment semantics for $\lgname$. The variable lookup rule \textsc{Var-Lookup} states that every agent has a local copy of every variable in the program, and if an agent is evaluating an expression involving variable $x$, it will replace $x$ with the current value $v$ from the local context $M$. $M[x\mapsto v]$ here refers to the fact that in the map $M$, the key $x$ maps to the value $v$. \textsc{Lvar-assign} describes the semantics of a local variable assignment. $x\notin \mathit{Keys}(S) \wedge x \in \mathit{Keys(L.M)}$ represents the fact that there is no mapping from $x$ in the global context $S$, but there is one in the local context $L.M$. 

\begin{mdframed}
\scriptsize
\begin{mathpar}
%
  \hspace{-1in}\inferrule*[Right=\textsc{Var-Lookup}]
    {\agnt.M[x\mapsto v_1]} 
    {\langle{S},\agnt, x \rangle \exprule  \langle{S},\agnt, v_1 \rangle } \label{vl1} \and \qquad\qquad 
  \\

\hspace{-1in} \inferrule*[Right=\sc{lvar-assign}]
    {\begin{array}{l}
    x \notin \mathit{Keys}({S}) \wedge x \in \mathit{Keys}(\agnt .{M}) \cr \wedge {\agnt = (P, {M},\mathit{sm},\mathtt{prog})\ \wedge\ \agnt^\prime = (P,{M}[x\mapsto c],\mathit{sm},  \mathtt{prog})}\end{array} }
    {\langle{S},\agnt, x = c \rangle {\rightarrow_{S}}  \langle{S},\agnt^\prime,\cdot\rangle}\label{va2} \and \qquad\qquad \\
    \end{mathpar}
\end{mdframed}

\noindent

\subsubsection{Shared memory updates}

 The set of all copies of the local copies shared variables can be seen as the \emph{distributed shared memory}. The correctness of a program relies on consistent values of the agents' local values of shared variables. When an agent writes to a variable, it updates its local copy and our implementation of the $\lgname$ framework uses message passing to inform the other agents of the change. 
%
\begin{mdframed}
\scriptsize
\begin{mathpar}
\hspace{-.5in}\inferrule*[Right=\sc{svar-assign}]
    {\begin{array}{l}
    x \in \mathit{Keys}({S}) \wedge x \in \mathit{Keys}(\agnt .{M}) 
    \wedge\ \agnt = (P, {M},\mathit{sm},\mathtt{prog}) \cr
    \wedge {S}^\prime = {S}[x \mapsto v]
    \wedge\ \agnt^\prime = (P,{M}[x\mapsto v],\mathit{sm}, \mathtt{prog})
    \end{array}
    }
    {\langle{S},\agnt, x = v \rangle {\rightarrow_{S}}  \langle{S}^\prime,\agnt^\prime,\cdot\rangle}\label{va1} \\
    \end{mathpar}
\end{mdframed}
\subsection{Event semantics}

The statement semantics are only defined when the turn of the agent was set to \texttt{prog}. We first present some rules for control flow in the program. The rules \textsc{stmt-seq-1} ensures that a statement are processed completely before moving on to processing the next statement, and \textsc{stmt-seq-2} ensures that once a statement that has been processed completely, the next statement comes up for processing. 
\begin{mdframed}
	\scriptsize
\begin{mathpar}
\inferrule*[Right=\sc{stmt-seq-1}]
    {\langle{S},\agnt, St_1 \rangle {\rightarrow_{S}}  \langle{S}^\prime,\agnt^\prime, St_1^\prime\rangle}
    {\langle{S},\agnt, St_1\ St_2 \rangle {\rightarrow_{S}}  \langle{S}^\prime,\agnt^\prime, St_1^\prime\ St_2 \rangle}\label{ss1} \and \qquad \\
\inferrule*[Right=\sc{stmt-seq-2}]
   {\;}
    {\langle{S},\agnt, \cdot \ St_2 \rangle {\rightarrow_{S}}  \langle{S},\agnt, St_2 \rangle}\label{ss2}\and \qquad \qquad \qquad \qquad \\
   
\end{mathpar}
\end{mdframed}


Given the control flow statements, the following rules capture two facts. Rule \textsc{event} states that only the effect of an an enabled event may be executed, and a marker \emph{endEvent} is added after the effect of that event to indicate that the event has executed. Rule \textsc{prog-to-env} states that
after the event is executed, the $\turn$ of the agent is set to $\mathtt{env}$ indicating that after this execution, an environment transition occurs. We will demonstrate in \refsect{experiments} that the system can display non deterministic behaviors arising from different agents executing their events in different orders.
\begin{mdframed}
\scriptsize
\begin{mathpar}
\hspace{-.5in}
\inferrule*[Right=\sc{event}]
    {\begin{array}{l} \mathit{ev} = \s{pre } \mathit{Cond} \s{ eff }\mathit{Ss} \cr \wedge {\agnt= (P,{M},\mathit{sm},\mathtt{prog}) \wedge\ \agnt^\prime =  (P,{M},\mathit{sm},\mathtt{prog})\wedge \mathit{eval(\s{pre}, L.M)} }\end{array}} 
    {\langle{S},\agnt, \cdot \rangle {\rightarrow_{S}}  \langle{S},\agnt^\prime, \mathit{Ss;endEvent} \rangle }\label{e1}\\
\hspace{-.5in}
    \inferrule*[Right=\sc{prog-to-env}]
    {\begin{array}{l} {\agnt= (P,{M},\mathit{sm},\mathtt{prog}) \wedge\ \agnt^\prime =  (P,{M},\mathit{sm},\mathtt{env}) }\end{array}} 
    {\langle{S},\agnt, \mathit{endEvent} \rangle {\rightarrow_{S}}  \langle{S},\agnt^\prime, \cdot \rangle }\label{e1}
        \end{mathpar}
\end{mdframed}




\subsection{Alternating synchronous cyber-physical semantics} 


\noindent The advancement of time and environment transitions govern the changes in the system configuration. The rewrite rule is a mapping from an initial system configuration to a set of configurations. It has the type
$$\rightarrow_G\ \subseteq (\mathbb{L}\times \pws \times \mathbb{R}^+\times \{\mathtt{prog,env}\}) \times \mathscr{P}(\mathbb{L}\times\pws \times \mathbb{R}^+ \times \{\mathtt{prog,env}\}) $$

First, we present the semantics of executing the events for all agents. A $\lgname$ program can be seen as sequences of $\stmtrule$ rules, which determine event executions followed by an environment transition. To express this, we define a rule that shows the the changes in the configuration of the overall system due to all the agents executing their program just before an environment transition is applicable.  


\begin{mdframed}
\scriptsize
\begin{mathpar}
\hspace{0.2in}\inferrule*[Right=\sc{run},Left={$\forall i\in\UINS$}]
{ (S,\lconfig{i},\lconfig{i}.p)\stmtrule(S^\prime,\lconfig{i}^\prime,p^\prime)\stmtrule\ldots\stmtrule (S^{\prime\prime},\lconfig{i}^{\prime\prime},\ \cdot\ )\wedge \lconfig{i}^{\prime\prime}.\turn=\mathtt{env}}
{({L}, S, \tau,  \mathtt{prog})\rightarrow_G ({L^{\prime\prime}}, S^{\prime\prime}, \tau, \mathtt{env})}\label{runsys}
\end{mathpar}
\end{mdframed}


\fTBD{FIXING Rule REFERENCES}The rule \textsc{run}  expresses that starting from a global configuration $c = ({L}, S, \tau, \mathtt{env})$, each agent $i$ with local configuration $\lconfig{i}$ processes its program $p$ using a sequence of $\stmtrule$ rewrites, until its event is executed, and its $\turn$ set to $\mathtt{env}$ at the end of the event execution. Overall, the system goes from a configuration $c$ to $c^{\prime\prime}= ({L^{\prime\prime}}, S^{\prime\prime}, \tau, \mathtt{env})$, with possibly different agent configurations and global context depending on whether any of the events executed resulted in writes to shared variables.



%\fTBD{Agent local states and propagation over messages.}
Now that we have the semantics for system to finish the program transition, we present the semantics for environment transitions, and how the \emph{turn} resets to \verb|prog|. Suppose the function $f:\mathit{Sens} \times Val \times \mathbb{R} \mapsto \Val$ captures the behavior of the controller, and can be used to update the sensor ports after time $\delta$. The rule \textsc{envtrans} shows the semantics of the system configuration after the rule \ref{runsys}.
\begin{mdframed}
    	\scriptsize
    \begin{mathpar}
    \inferrule*[Right =\sc{envtrans}]{ \begin{array} \forall i \in \UINS,\lconfig{i}^\prime.\mathit{sm} = \agnt.\mathit{f}(\lconfig{i}.\mathit{sm},\delta) \wedge\cr \forall x \in \mathit{Keys}({S}),\lconfig{i}'.{M}[x \mapsto S[x]] \wedge \lconfig{i}.\turn = \mathtt{env} \end{array}}
    { (L, {S}, \tau,\mathtt{prog})\rightarrow_G ({L}^\prime, {S}, \tau + \delta,\mathtt{env}) \label{env}     }
    \\
  \hspace{-1in}  
 \inferrule*[Right=\sc{agent-env-to-prog}]
     {\begin{array}{l} \forall i \in \UINS , \lconfig{i}.\turn = \mathtt{env} \wedge \lconfig{i}^\prime.\turn = \mathtt{prog} \end{array}}
     {  ({L}, S, \tau, \mathtt{env})\rightarrow_G ({L^\prime}, S, \tau, \mathtt{env}) \label{agntenvturn}   \and \qquad \qquad }\\
    
    
    
     \inferrule*[Right=\sc{env-to-prog}]
     { \forall i \in \UINS , \lconfig{i}.\turn = \mathtt{prog} }
     {  ({L}, S,\tau, \mathtt{env})\rightarrow_G ({L}, S, \tau, \mathtt{prog}) \label{envturn}     }
\noindent   
\end{mathpar}
    \end{mdframed}

Rule \textsc{agnt-env-to-prog} is the semantic rule for setting the $\mathit{turn}$ of each agent back to $\mathtt{prog}$ from $\mathtt{env}$ after the environment transition has been completed. The semantics synchronizes the program transitions of each agent, ensuring that the event executions for each happen every $\delta$ time.
    
    
Rule \textsc{envtrans} captures the fact that the global context $S$ is copied into local context of each agent $i \ (\lconfig{i}.M)$, thus ensuring that all agents have consistent shared variable values before the next program transition. After that, rule \textsc{env-to-prog} changes the turn of each agent back to program. In an actual execution, the controller  would run the program on hardware, whose sensor ports evolve for $\delta$ time between program transitions. 

\fTBD{Relate this to assumptions}
\subsection{Implementation in K}

The language syntax is first parsed using a standard indentation parser implemented in python. We implemented $\lgname$ semantics in \K, which a rewriting-based executable framework for defining language semantics. One can view a language semantics naturally as a set of reduction rules over configurations. Components of a configuration are called {\em cells} in \K. To implement the semantics discussed in \refsect{semantics}, the configuration in \K includes several bookkeeping variables as well. 


Semantics in \K is expressed using configurations, which organize the components in elements called {\em cells}. Cells are labelled, have types indicating what kind of elements can be contained in them, and help specify the rewrite rule in context. \K allows underspecification of rewrite rules, meaning, only the rewrite rules affecting part of the configurations need to be specified if the rule doesn't affect the other parts of the rule. Our notion of configurations translate in a straightforward manner to \K configurations. We present a couple of example rewrite rules in \K to demonstrate how the executable semantics is implemented.  


We use a top level {\em system} cell with nested cells corresponding to the elements discussed in \refsect{semantics}, as well as other cells which are used to store information like number of agents in the system, simulation parameters of the program, and other bookkeeping information we used to implement the semantics. There is a special cell called the {\em k} cell, which is used to store the current computation in the program. Each agent has a {\em k} cell to store its computation. Recall that a local variable is updated only in the local memory. The rule for this checks whether the variable is not in the keys of the map {\em sysLoc} from shared variables to their locations. The following rule in \K that corresponds to our \textsc{lvar-assign} rule in the semantics. 
\vspace{2pt}
\begin{mdframed}
\begin{Verbatim}[fontsize=\tiny]
<agent> ... <k> V:Id = I:Val => .  ... </k> 
            <loc> ... V |-> L ... </loc> 
            <locstore> ... L |-> _ => I ... </locstore> 
        ... </agent> 
<sysloc> Rho </sysloc>                    when L notBool inKeys(Rho)        
\end{Verbatim}
\end{mdframed}

We use bookkeeping variables in the \K configuration to ensure that all agents have their turn set to \texttt{env} before the system environment transition.We also  ensure that an agent has updated its sensor ports after the environment transition, to implement the rule \textsc{agent-env-to-prog}. Once all the agents in the system are ready to perform the next program transition, the rule \textsc{env-to-prog} is implemented as follows. If the number of agents which have set their $\mathit{turn}$ to $\mathit{env}$ (indicated by the cell {\em envToProg}) is the same as the number of all agents in the system, then the system sets its turn from $\mathit{Env}$ to $\mathit{Prog}$. 
\vspace{2pt}
\begin{mdframed}
\begin{Verbatim}[fontsize=\tiny]
<system>...
 <envToProg> N </envToProg>
 <numAgents> N </numAgents>  
 <turn> Env => Prog </turn> ... 
</system>                      
\end{Verbatim}
\end{mdframed}

\section{Koord Software Stack}
\label{sec:software}

\subsection{Runtime system}



To run a $\lgname$ program (hardware or simulation), the user has to provide a configuration file, with
\begin{inparaenum}
    \item the number of agents,
    \item in case of simulation, the initial positions of the agents and the length of the simulation and
    \item in case of hardware deployment their IP addresses,
    and the localization system.
\end{inparaenum}

\subsection{Key environment assumptions}


\subsubsection{Periodic event execution semantics}


\subsubsection{Shared variable implementation over message passing}


\subsubsection{Known set of participants}
\subsubsection{Portability and heterogeneity}


\subsection{Simulator}
\subsubsection{gazebo environment}
\subsubsection{car model}
\subsubsection{lidar}
\subsubsection{positioning}
\subsubsection{sampled sensing}
\subsubsection{synchronization issues}

\subsection{The Distributed Mapping Problem}
In this section, we introduce the distributed mapping problem. Informally, the problem requires a set of robots to collaboratively mark the position of static \emph{obstacles} within a given area $D$, which any robot should avoid while moving in $D$.The key difference between distributed SLAM and this application is that we assume that the robots know their \emph{global coordinates} within the area of deployment. They are only attempting to map the static obstacles within this area. We currently assume that the only sensors available for sensing obstacles are LIDAR based, and the robots are constrained to move in a 2-D space.


\subsubsection{Preliminaries}
\label{sec:prelims}
We first set up the terminology and assumptions to discuss our approach to this problem.

The mapping problem is defined over a (\emph{bounded}) domain $D$, a bounded rectangle in $\mathbb{R}^2$ given by $[a_1,a_2]\times [b_1,b_2]$.

\begin{definition}
    A \emph{quantization} of a bounded domain $D$ is defined as a mapping $\qfunc:D \mapsto \qdom$ where $\qdom = \left\{q_{ij}\right\}_{i\in [1..n_x], j\in [1..n_y]}$ such that every $q_{ij}$ corresponds to a \emph{grid rectangle} $[x_i, x_{i+1}] \times [y_j, y_{j+1}]$.
\end{definition}

We assume the existence of a \emph{ground truth} function $\world : D\mapsto \left\{0,1\right\}$, where $\forall \Vec{x} \in D$, $$\world(\Vec{x}) = \begin{cases}
                                                                                                                                                                        1\ \mbox{if there is an obstacle at }\Vec{x}\\
                                                                                                                                                                        0\ \mbox{otherwise}
\end{cases}
$$


\begin{definition}
    Given a quantized domain $Q$, a \emph{\qdfunc} over $Q$ is any function $F$ with the signature $Q^\prime \mapsto \left\{0,1\right\}$, where $F$ is only defined on $Q^\prime \subset Q$, and it maps each $q_i \in Q^\prime$ to either 0 or 1.
\end{definition}


Given any quantization $Q$ of the domain $D$ given by $\qfunc:D\mapsto Q$, the corresponding \qdfunc
$\world_Q : Q \mapsto \left\{0,1\right\}$ is defined as follows,  $$\world_Q(q) = \begin{cases}
                                                                                      1\ \Leftrightarrow \exists \Vec{x}\in D, \qfunc(\Vec{x}) = q \wedge \world(\Vec{x}) = 1 \\
                                                                                      0\ \mbox{otherwise}
\end{cases}
$$


\rg{Consider that there is a set of ground robots $[N]$, which are tasked with creating a mapping of static obstacles in $Q$ collaboratively by constructing local mappings based on sensed information.}


\begin{definition}
    The \emph{sensing area} of a robot $i$ at position $\pos(i)\in D$ is defined by $\sensarea_i: Q \mapsto 2^{Q}$, such that there is a \qdfunc\ $\rmap : \sensarea_i \mapsto \left\{0,1\right\}$, where $$\forall q \in \sensarea_i(\qfunc(\pos_i)), \rmap(q) = \world_Q(q).$$
    Given a robot $i \in [N]$, $\sensarea_i$ may depend both on the position of the robot as well as the range of the LIDAR scanner.
\end{definition}



Let $\map_i:Q\mapsto \left\{-1,0\right\}$ denote a \qdfunc \emph{local} to robot $i$, which we see as a \emph{software state} for a given robot $i$.  % $\map_i(q) = 1$ indicates that according to robot $i$, there is an obstacle in $q$, $\map_i(q) = 0$ indicates that according to robot $i$, $q$ is unoccupied, and $\map_i(q) = -1$ indicates that robot $i$ doesn't have information about $q$.



We assume that given $\Vec{x} \in q, \forall \Vec{x^\prime} \in q, \sensarea(\Vec{x}) \subseteq \sensarea(\Vec{x^\prime})$. We can now state the 2-d distributed mapping problem, $\mapprob$ as follows. \begin{quote}
{\em Given a quantization, $\qfunc:D\mapsto Q$ of a 2-d domain $D\subset \mathbb{R}^2$, a ground truth function $\world:D \mapsto \left\{0,1\right\}$, a set of robots $[N]$ , for each robot $i \in [N]$ construct an \emph{occupancy map} which is a \qdfunc, $\map_i: Q_i \mapsto \left\{1,0\right\}$, $Q_i\subseteq Q$.
}
\end{quote}

% \rg{We can \emph{combine} the elements of the set of \emph{local maps}, $\{\map_i\}_{i\in [N]}$ to form a \emph{global} occupancy mapping. }

Having stated the problem, we now define the notion of soundness of a proposed occupancy map.
\begin{definition}
    \label{soundness}
    For a robot $i$, a proposed occupancy mapping over $Q_i\subset Q$, given by $\map_i: Q_i \mapsto \left\{-1,0\right\}$ is \emph{sound} if :
    \begin{itemize}
        \item $\map_i(q) = v \Rightarrow \world_Q(q)  = v$
        \item $\forall v \in \left\{0,1\right\}\world_Q(q) = v \Rightarrow \map(q) = v \vee q\notin Q_i$
    \end{itemize}
\end{definition}


These two statements collectively state that given a proposed occupancy map, \emph{any grid rectangle in the domain of the occupancy map marked as 0 is indeed obstacle free, and if it is marked as 1 then there is indeed an obstacle at least partially in it.}

A vacuously correct (sound) solution to $\mapprob$ given $D, \qfunc$ and $\world_Q$ is $\forall i \in [N]$, $$\map_i : Q_i \mapsto \left\{0,1\right\}, Q_i = \phi$$ To allow for more interesting solutions than the one stated above, we assume that we are given that initially, each robot $i\in[N]$ starts at a grid rectangle with no obstacle, the sensed area of each robot $i$ is non empty, and there is at least one obstacle free rectangle within the sensed area of each robot:
$\forall i \in [N], \world_Q(q^0_i) = 0 \wedge \exists q \in \sensarea(q^0_i), \world_Q(q) = 0 $ and
where $q^0_i = \qfunc(\pos_0(i))$, and $\pos_0(i)$ denotes the initial position of robot $i$.


\begin{definition}
    Given  $i , j \in [N]$, two proposed occupancy mappings $\map_i: Q_i\mapsto\left\{0,1\right\}$ and $\map_j: Q_j\mapsto \left\{0,1\right\}$, are consistent only if $\forall q \in Q_i \cup Q_j, \map_i(q) = \map_j(q)$.
\end{definition}

    Given $\map_i$, $\map_j$, and $q\in Q_i \cup Q_j$ , let $\map_i(q) = 1$, and $\map_j(q) = 0$. Suppose $\map_i$ is sound, then $\world_Q(q) = 1$, which implies $\map_j$ is not sound. By the same argument, if $\map_j$ is sound, $\map_i$ is not. A set of proposed mappings $\left\{\map_i\right\}_{i\in [N]}$ can only be sound if they are pairwise consistent.





\subsubsection{Approach}


\begin{figure}[!htbp]
\centering
\includegraphics[width=\linewidth]{figs/map_flowchart.png}
\caption{Flowchart for a simple solution to 2D distributed mapping problem\vspace{-2mm}}
\label{fig:flowmap}
\end{figure}



We discuss now discuss how our algorithm implemented $\lgname$ shown in \reffig{mapapp} tackles $\mapprob$. \reffig{flowmap} shows a simple idea for solving this problem for each robot:

The variable $\lmap$ refers to the current mapping $\map_i$ constructed by each robot $i$ using the algorithm. The function $\mathit{MaxExp}$ informally, determines whether there is a grid rectangle in the frontier of the current $\map_i$ . If not, the robot first updates its $\lmap$ from $\gmap$, which is used for sharing the currently computed occupancy maps by all agents so far. The robot then picks a new point in a rectangle known to be unoccupied in its $\lmap$ and follows a path ($\mathit{Motion.Path}$) moving only over grid rectangles known to be unoccupied by its $\lmap$. While the robot hasn't reached the target rectangle, it keeps updating its $\lmap$ with sensed data (occupied and unoccupied rectangles). When it reaches the target, it updates the $\gmap$ from its $\lmap$.


\subsubsection{Analysis}
A (distributed) system of robots which execute $\lgname$ can be viewed as a nondeterministic, timed automaton~\cite{TIOAmon}. The robots execute the program in  \emph{rounds} of duration $\delta$, and in each round, each robot performs an action, which is expressed in $\lgname$ as an event, and the state of each agent can be seen as evolving according to the controllers driving the robot according to the actuators that may have been set in the event. The program and sensor variables of the robot can be seen as the state variables for each agent.


\noindent
Given the set of all variables in the $\lgname$ applications is $\Var$, and the set of all values that the variables can take is $\Val$. The state of a robot $i$ is a tuple $$s_i = ( \{M\}, \{cp\}, r), \mbox{ where}$$
\begin{enumerate}
\item $\{M\} : \Var \mapsto \Val$ is the mapping of all program variables in the application known by the robot to their values.
\item $\{cp\} : \Var \mapsto \Val$ is the mapping of actuator and sensor variables to their values.
    \item $r\in \mathbb{Z},r \geq 0$ is the \emph{round} of program execution.
\end{enumerate}
The components of the tuple $s_i$ are accessed using the dot $.$ notation, for instance, $s_i.M$, etc. The value of a program variable $v$ at a state $s_i$ is denoted by $s_i.M.v$.



Having defined the state of the agent, we now analyze the $\lgname$ program shown in \reffig{mapapp}. Given a robot $i$, $s_i.M.\lmap$ represents the \emph{local map} $\map_i$ constructed by each agent. $s_i.M.\gmap$ represents the \emph{global} map $\map$ constructed from the local maps as outlined in \defn{cons}. In implementation, if $\map_i$ is not defined on $q\in Q$, then we set $\lmap[q] = -1$. Consequently, if $\map$ is not defined, we set $\gmap[q] = -1$.


We omit the initialization of the mappings in the presentation of the program in \reffig{mapapp}. From our assumption on the initial conditions discussed in \refsect{prelims}, for each robot $i$, $s_i.r = 0\Rightarrow s_i.M.\lmap[q_0] = 0$ where $q_0\in Q$ is the initial grid rectangle the robot is in. We also assumed that $\sensarea(q_0)\neq \phi$, and assume that $s_i.M.\lmap$ is intialized with the values of all the corresponding \emph{sound} values. $s_i.M.\gmap$ is initialized as $s_i.M.\gmap[q] = -1$ for all $q\in Q$. Thus, initially, both $s_i.M.\lmap$ and $s_i.M.\gmap$ are sound mappings.

\subsubsection{NewPoint}
In this event, the robot first updates its value of $s_i.M.\lmap$ from the currently known value of $s_i.M.\gmap$ where the operator $\oplus$ is defined as follows:
 $$\forall q \in Q, f[q] \oplus g[q] = \mathit{Max}(f[q], q[q])$$

Given two sound maps $\lmap$ and $\gmap$, $\lmap \oplus \gmap$ corresponds to a combined map construction as outlined in \defn{cons}, and is sound. The event \emph{NewPoint} doesn't modify $s_i.M.\lmap$ further, and therefore, $\lmap$ remains \emph{sound} during the execution of this event.

\begin{definition}
 We say that $\Vec{x_n}\in D$ is \emph{reachable} from $\Vec{x_0}$ if there is a \emph{path} $p = \Vec{x_0},\Vec{x_1}, \Vec{x_2},\ldots, \Vec{x_n}$ , such that
\begin{itemize}
\item $\forall i \in [0..n],\world_Q(\qfunc{\Vec{x_i}}) = 0$
\item a robot can move from $\Vec{x_i}$ to $\Vec{x_{i+1}}$ for $i \in [0..n]$, while staying within $\qfunc(\Vec{x_i}) \cup \qfunc(\Vec{x_{i+1}})$.
\end{itemize}
\end{definition}

A grid rectangle $q\in Q$ is reachable in general, if $\exists \Vec{x}\in D, q = \qfunc(\Vec{x})$ such that $\Vec{x}$ is reachable from either \begin{inparaenum} [(a)]\item the initial position of a robot, or \item another reachable point $\Vec{x^\prime}\in D$. We denote the reachability of a grid rectangle using a predicate $\mathit{Reach} : Q \mapsto \left\{\mathit{True}, \mathit{False}\right\}$, where $\mathit{Reach}(q) = \mathit{True}$ if its reachable,  $\mathit{False}$ otherwise.
\end{inparaenum}

\begin{definition}
    Given a sound occupancy mapping $\map_i$, the \emph{frontier} of $\map_i$, denoted by $\ff(\map_i)$ is defined as follows:
    $$ \left\{ q\in Q \mid Reach(q) \wedge \exists q \in \sensarea(\Vec{x}), q \notin \mathit{dom}(\map_i)\right\} $$
\end{definition}

In the event \emph{NewPoint}, the operation the robot tries to find a path to a point on the \emph{frontier} of $s_i.M.\lmap$. Assume that the planner returns a path if \begin{inparaenum}[(a)]\item the frontier is non-empty [(b)] the grid rectangle picked on the frontier is reachable from the current point \end{inparaenum}. We also constrain the robots to move only on \emph{known} unoccupied grid rectangles, i.e $q\in Q, s_i.M.\lmap[q] = 0$. In implementation, we achieve this by providing all the unknown ($s_i.M.\lmap[q] = -1$) squares as obstacles to the path planner.

If the path is empty, then the robot sets its $s_i.M.\mathit{update}$ variable to $True$, \emph{enabling} the \emph{GUpdate} event, which we presently discuss. Otherwise, until the robot finishes traversing the path, the \emph{LUpdate} event is enabled.

%\begin{lemma}
%    \label{ext}
%    Suppose robot $i$ is at $\pos(i)\in D$, and $\exists(q^\prime)\in \sensarea(\pos(i))$,
%    s.t $\map_i(q) = -1$. Consider a mapping, $\map^\prime_i:Q \mapsto \left\{-1,0,1\right\}$
%    such that, $\forall q\in Q \setminus \sensarea(\pos(i)), map^\prime_i(q) = map_i(q)$
%    and $\forall q \in \sensarea(\pos(i)), \map^\prime_i(q) = \world_Q(q)$.
%    Then, $\map^\prime_i(q)$ is sound if $\map_i$ is sound.
%\end{lemma}
%
%The proof for this is straightforward and follows from the definition of $\map^\prime_i$ combined with \defn{soundness}.
%
%Recall from the definition of the sensing area of a robot $i$ at $\pos(i)$, that it can reliably compute the ground truth mapping $\world_Q$ for all $q \in \sensarea(\pos(i))$. This lemma essentially states that given a sound occupancy mapping $\map_i$, it can be used to compute to another \emph{sound} occupancy mapping $\map^\prime_i$ by setting the values of $\map^\prime_i(\sensarea(\pos(i)))$ to the ground truth function.
%
%\begin{definition}
% We say that $\Vec{x_n}\in D$ is \emph{reachable} from $\Vec{x_0}$ if there is a \emph{path} $p = \Vec{x_0},\Vec{x_1}, \Vec{x_2},\ldots, \Vec{x_n}$ , such that
%\begin{itemize}
%\item $\forall i \in [0..n],\world_Q(\qfunc{\Vec{x_i}}) = 0$
%\item a robot can move from $\Vec{x_i}$ to $\Vec{x_{i+1}}$ for $i \in [0..n]$, while staying within $\qfunc(\Vec{x_i}) \cup \qfunc(\Vec{x_{i+1}})$.
%\end{itemize}
%\end{definition}
%
%A grid rectangle $q\in Q$ is reachable in general, if $\exists \Vec{x}\in D, q = \qfunc(\Vec{x})$ such that $\Vec{x}$ is reachable from either \begin{inparaenum} [(a)]\item the initial position of a robot, or \item another reachable point $\Vec{x^\prime}\in D$. We denote the reachability of a grid rectangle using a predicate $\mathit{Reach} : Q \mapsto \left\{\mathit{True}, \mathit{False}\right\}$, where $\mathit{Reach}(q) = \mathit{True}$ if its reachable,  $\mathit{False}$ otherwise.
%\end{inparaenum}
%
%\begin{definition}
%    Given a sound occupancy mapping $\map_i$, the \emph{frontier} of $\map_i$, denoted by $\ff(\map_i)$ is defined as follows:
%    $$ \left\{ q\in Q \mid Reach(q) \wedge \exists q \in \sensarea(\Vec{x}), \map_i(q) = -1\right\} $$
%\end{definition}
%
%Given a sound occupancy mapping $\map_i$, another sound mapping $\map_i^\prime$ can be constructed as shown in \lem{ext}. Taken in conjunction with our assumption on the initial positions of each robot, this leads to a strategy for computing sound occupancy mapping functions. Further, given a set of sound mappings $\left\{\map_i\right\}_{i\in[N]}$, we can construct a sound mapping $\map$ from them as follows :
%

%
%

%
%
%
\subsubsection{External (Library) Functions}

% restriction of the world function for sensing. accuracy of sensor statement.
% domain of mapping function for which value is 0 or 1.
% make compatibility a definition instead of lemma.
\newcommand{\Koord}{\ensuremath{\lgname}\xspace}
\newcommand{\CyPhyHouse}{CyPhyHouse\xspace}
\newcommand{\Gazebo}{Gazebo\xspace}

\newcommand{\ScanToMap}{\ensuremath{\mathit{scanToMap}}\xspace}
\newcommand{\TSync}{\ensuremath{\mathit{tSync}}\xspace}
\newcommand{\PathToFrontier}{\ensuremath{\mathit{pickPathToFrontier}}\xspace}

\section{Case study: Distributed Mapping application}
\label{sec:experims}

For our case study, we test the \dmap application using the simulator from \CyPhyHouse~\cite{ghosh2019cyphyhouse} for distributed \Koord applications.
We choose the MIT RACECAR model~\cite{MIT_RACECAR} with Lidar sensor included in \CyPhyHouse for our experiment.
To implement external functions defined in~\ref{sec:analysis},
we extend the \CyPhyHouse tool chain and examine our assumptions over sensor data and implemented functions.
In the section, we first discuss our implementation of external functions;
then we present the simulation result of \dmap with multiple vehicles.


\subsection{Implementation of External Functions}

\paragraph{Sound Mapping from Synchronized Sensor Data~(\ScanToMap).}
To generate a 2D map from the scan data,
it is required to synchronize the current vehicle position and the Lidar scan data~(\TSync).
An existing solution is to filter the time stamps of both sensor data streams
and only choose pairs of data within a given time difference threshold $\epsilon$.
This can be achieved by \texttt{ApproximateTimeSynchronizer} in the ROS package named \texttt{message\_filter}.
The frequency of the synchronized data is then limited to the sensor with the lowest frequency.
In our simulation, the lowest frequency is 100 Hz for sensing the vehicle position and sufficient for updating the map.

The Lidar scan data provided in the tool chain are pairs of a distance to obstacles and an angle with respect to the heading angle of the vehicle.
To implement \ScanToMap, it is then straightforward to compute the positions of obstacles and mark occupied grid rectangles,
and to mark a rectangle obstacle free is to simply check if all four corners of a grid rectangle are covered in the scan.
The time difference $\epsilon$ however can introduce an error between real and perceived distances to obstacles.
Reducing this error is beyond the scope of this work.
Our implementation of \ScanToMap therefore may falsely mark a grid rectangle as occupied when obstacles are in nearby grids.

\paragraph{Path to Frontier~(\PathToFrontier).}
In the \CyPhyHouse tool chain, several path planning algorithms are already available.
We select a particular Rapidly exploring Random Tree with near neighbor search and rewiring tree~(RRT*) algorithm,
and it is able to avoid obstacles and is fine tuned to the car model for simulation.
Our \PathToFrontier implementation hence only needs to choose a point in frontier grid rectangles and provide the positions of obstacles.
To choose such a point, a basic solution is to use Breadth-First Search~(BFS) on the map and seek for a unexplored grid rectangle.
BFS ensures that either all reachable grid rectangles are marked,
or there are connected unoccupied rectangles to an unexplored rectangle, and hence a path must exist.
Nonetheless, it does not guarantee that the underlying path planner can always find a path because,
for instance, the car is incapable of make sharp turns.
Our soundness claim still holds under such circumstance, but we may not be able to progress even when paths to frontier exist.


\subsection{Simulation Result}



\section{Conclusions and Future Work}


Our vision is to provide a programming methodology to enable programming safe distributed cyberphysical applications without the need of complete domain expertise in all related areas such as control theory, robotics motion control, and network protocols.
To this end, we demonstrated how \lgname application developers can write succinct multi-robot applications involving distributed coordination,
different types of sensing and actuation, and path planning in three case studies, each of which requires only preliminary knowledge
in shared memory and basic concurrency control via atomic blocks and assumptions on sensor and actuator ports of controllers.
V\&V engineers with deep understanding in distributed computing are able to focus on formally analyzing invariant properties of \lgname programs via symbolic execution, and roboticists can validate the feasibility of assumptions by examining rigorously defined proof obligations.

We acknowledge the fact that our \portasum based abstractions may not cover various vastly different types of robots.
\K semantics framework can allow us to extend our language to tailor to specific robot types on demand
while retaining the same framework for formal analysis. We also plan to extend this work to include specification and verification of progress properties under fairness constraints for \lgname applications.

%\newpage\newpage
\bibliographystyle{abbrv-fr}
\bibliography{cyphyhouse.bib,sayan1.bib}
\section{Complete \lgname Syntax and Semantics}

\input{syntax.tex}

% We implemented \lgname semantics in \K, which a rewriting-based executable framework for defining language semantics. One can view a language semantics naturally as a set of reduction rules over configurations. Components of a configuration are called {\em cells} in \K. To implement the semantics discussed in \refsect{semantics}, the configuration in \K includes several bookkeeping variables as well.
%
%


%Semantics in \K is expressed using configurations, which organize the components in elements called {\em cells}. Cells are labelled, have types indicating what kind of elements can be contained in them, and help specify the rewrite rule in context. \K allows underspecification of rewrite rules, meaning, only the rewrite rules affecting part of the configurations need to be specified if the rule doesn't affect the other parts of the rule. Our notion of configurations translate in a straightforward manner to \K configurations. We present a couple of example rewrite rules in \K to demonstrate how the executable semantics is implemented.
%
%
%We use a top level {\em system} cell with nested cells corresponding to the elements discussed in \refsect{semantics}, as well as other cells which are used to store information like number of robots in the system, simulation parameters of the program, and other bookkeeping information we used to implement the semantics. There is a special cell called the {\em k} cell, which is used to store the current computation in the program. Each robot has a {\em k} cell to store its computation. Recall that a local variable is updated only in the local memory. The rule for this checks whether the variable is not in the keys of the map {\em sysLoc} from shared variables to their locations. The following rule in \K that corresponds to our \textsc{lvar-assign} rule in the semantics.
%\vspace{2pt}
%\begin{mdframed}
%\begin{Verbatim}[fontsize=\tiny]
%<robot> ... <k> V:Id = I:Val => .  ... </k>
%            <loc> ... V |-> L ... </loc>
%            <locstore> ... L |-> _ => I ... </locstore>
%        ... </robot>
%<sysloc> Rho </sysloc>                    when L notBool inKeys(Rho)
%\end{Verbatim}
%\end{mdframed}
%


\section{Code and Figures}
\chiao{Appendix should not be included for PLDI and should be provided separately as supplementary materials.
We just put figures and code here for now to count the pages.}


\subsection{Distributed Task Allocation}


\begin{figure}[h!]
    \includegraphics[width=\columnwidth]{figs/taskalloc_w_marker.png}

    \caption{Screenshot in the simulation of Distributed Task Allocation}
\end{figure}




\subsection{Distributed Mapping Problem}
\sayan{
    Informally, the problem requires a set of robots to collaboratively mark the position of static \emph{obstacles} within a given area $D$, which any robot should avoid while moving in $D$.The key difference between distributed SLAM and this application is that we assume that the robots know their \emph{global coordinates} within the area of deployment. They are only attempting to map the static obstacles within this area. We currently assume that the only sensors available for sensing obstacles are LIDAR based, and the robots are constrained to move in a 2-D space.
}

\subsection{\dmap Application}

\reffig{flowmap1} shows a simple idea for solving the mapping problem problem for each robot, and \reffig{mapapp} shows our solution to $\dmap$ written using $\lgname$, a high level language with native support for multi-robot systems designed to interact with a physical environment. The design of solution to the mapping problem in \reffig{flowmap1} captures an occupancy map of the 2D space in a variable $\gmap$. \reffig{flowmap}. The variable $\lmap$ is a local mapping constructed by each robot $i$ using sensors, and information from other robots shared via $\gmap$. The robot first updates its $\lmap$ from $\gmap$, which stores the currently computed occupancy map \emph{by all robots}.  The robot then picks a new point in the grid known to be unoccupied in its $\lmap$ and follows a path ($\mathit{Motion.Path}$) moving only over grid rectangles known to be unoccupied by its $\lmap$. While the robot hasn't reached the target rectangle, it keeps updating its $\lmap$ with sensed data (occupied and unoccupied grid points). When it reaches the target, it updates the $\gmap$ with new data from its $\lmap$.


\begin{figure}[!htbp]
    \centering
    \includegraphics[width=\linewidth]{figs/map_flowchart.png}
    \caption{Flowchart for a simple solution to 2D distributed mapping problem\vspace{-2mm}}
    \label{fig:flowmap1}
\end{figure}

 An \emph{allwrite} variable is a shared variable which all robots can read from and write to. The shared \emph{allwrite} variable $\gmap$ is used to construct a shared map of obstacles within the domain $D$, and has type $\mathit{GridMap}$, which is a 2-D array representing a grid over $D$. The \emph{local} variable $\lmap$ represents each robot's \emph{local} knowledge of the domain $D$, and has the same type as $D$. A robot executing the \emph{NewPoint} event, first updates a \emph{local variable} $\lmap$ from the shared variable $\gmap$, using a combination operator $\oplus$, described in more detail in \refsect{prelims}. $\lgname$ allows the user to use library functions, like the $\mathit{findPath}$ function, which uses a path planner to find a path to a point while avoiding a set of \emph{obstacles}. The point is picked using the $\mathit{pickFrontierPos}$ function which is a user defined function implemented in $\lgname$. The details of the path planner, and how the point is picked are discussed in and \refsect{prelims} \refsect{experiments}.

In $\dmap$, there are two modules \emph{Motion} and \emph{Lidar} which provide interfaces to different sensors and actuators on the robot. The $\mathit{Lidar.ldata}$ sensor module is used to read the LIDAR scan of the actual robot. In the \emph{NewPoint} event, the controller driving the robot directs it along a path set at the actuator port $\mathit{Motion.path}$. The sensor port $\mathit{Motion.psn}$ gives the position of the robot (in a fixed coordinate system) and $\mathit{Motion.reached}$ indicates whether the controller is active or inactive.

The $\lgname$ program corresponding to this solution has three \emph{events}: \emph{NewPoint, LUpdate, GUpdate}. Each robot in the system runs an instance of the \emph{same} program. At runtime, the $\lgname$ program executes within the runtime system of a single robot, or a collection of programs execute on different robots. The $\lgname$ language semantics ensures that the execution of the program in the distributed system occurs in \emph{rounds} of duration $\delta$. In each round, each robot executes the statements in the effect(\textbf{eff}) of at most one \emph{enabled} event : an event whose precondition(\textbf(pre)) is satisfied. If no event is enabled, the robot does nothing. Before the next round of execution, the robot may continue to interact with the physical environment as directed by its controllers. The $\lgname$ semantics imposes a synchronous model of execution for $\lgname$ programs for multi-robot systems and its implementation in CyPhyHouse toolchain ensures that this schedule is maintained by the multi-robot system executing the $\lgname$ program, despite potentially imprecise synchronization of local clocks.

In the design of the $\dmap$ application, we employ the use of \emph{sampled} sensors, which is essentially a sequence of timestamped sensor readings between rounds. The \emph{LUpdate} event can occur while each robot is traversing a path and hasn't reached the final destination, during which the robot uses a sampled sensor reading of the positions $\mathit{Motion.trace}$, of type $\mathit{PosStamped}[]$, which denotes an array of timestamped positions; and a sampled sensor reading of LIDAR scans $\mathit{Lidar.ldata}$ of type $\mathit{ScanStamped}$, which denotes an array of timestamped LIDAR scans. These sampled sensor readings are then synchronized to associate a LIDAR scan with a position.

Mutual exclusion is an essential feature required in a distributed system with shared variables. The robot updates the shared variable $\gmap$ in the event \emph{GUpdate} using the value of its $\lmap$, which may have been updated with newly detected obstacles



%
% Algorithm~\ref{ag1}  computes reachable states of a given system based on the above functions and check $n$-invariance.
%$\mathit{Unique}(\mathit{Perms}(\UINS), P_\mathit{events})$ is the set of permutations of the robots that may result in unique behaviors,
% and we compute this set by analyzing the shared variables written to during enabled and urgent events of each robot during a round of program transitions.
% The input to \refalg{ag1} is \begin{inparaenum}[(i)]
%    \item $P$: the program,
%    \item $\mathit{inv}$: a candidate invariant function,
%    \item {\UINS}: set of robots,
%    \item $\delta$: time step size and
%    \item $n$: length executions to check.
%\end{inparaenum}
%
%%
%
%\begin{algorithm}
%\scriptsize
%\textbf{Input}: $P$, $\mathit{inv}$, $\UINS$, $\delta$, $n$\\
%$c\leftarrow \mathit{Init}(P,N)$\quad$p \leftarrow \mathit{Unique}(\mathit{Perms}(\UINS),P_{\mathit{events}})$\\
%\lIf{
%$ \mathit{Sat}(c,\neg\mathit{inv})$} { \Return `unsafe'}
%$C \leftarrow$ \{c\} \\
%\For{$i = 0$ \KwTo n}{
%\For{$j = 0$ \KwTo $\mathit{len}(p)$ }{
%$C^\prime\leftarrow\Post(C,p[j])$\\
%\For{$c$ $\mathbf{in}$ $C'$} {
%\lIf{
%$ \mathit{Sat}(c,\neg\mathit{inv})$} { \Return (`unsafe',c)}
%}
%$C^\prime \leftarrow \Final(C,p[j])$\\
%\rg{\emph{continuous reachability here}}
%}}
%\Return `safe'
%\caption{Bounded invariant checking algorithm}\label{ag1}
%\end{algorithm}
%  In the next iteration, $C$ is set to be the frontier set of configurations.
%Theorem~\ref{thm:bmcsound} summarizes the soundness of \refalg{ag1}.
%
%\begin{theorem}
%    \label{thm:bmcsound}
%    If Algorithm~\ref{ag1}  returns `unsafe' then there exists a counter-example to \emph{inv} of length at most $n$, \rg{state assumption on continuous behavior here}.
%    if Algorithm~\ref{ag1} returns `safe' then \emph{inv} is an $n$-invariant.
%\end{theorem}
%\begin{proof}
%{\em Sketch.} If algorithm \ref{ag1} returns {\em safe}, then for the $n$ loop iterations of the outer loop, given a set of states $C$ and the property \texttt{inv}, according to the algorithm $\forall c \in \Post(C), c \models \texttt{inv}$. $\Post(C)$ is the set of configurations that can be reached from $C$ using any possible sequence of rewrite rules allowed by the system before an environment transition. Therefore, if the algorithm returns safe, then the set of configurations reached from $C$ before an environment transition is also safe. \rg{continue proof of soundness of reachability of continuous behavior}
%\end{proof}
%\fTBD{Relate this to assumptions}
%


\subsection{Specify Constraints over Modules}
\label{sec:module}
\chiao{As mentioned in \refsect{semantics}, the assumptions over modules will be directed by the target invariant/property.
    What does \lgname provide to help identify the assumptions?}


\subsection{Proofs}
\begin{proof}
    Proof for Lemma~\ref{lem:noninter}:
    Suppose that given an arbitrary configuration $c$, such that $\bigwedge_{i\in \UINS}\bigwedge_{e \in \Event} \eec{\mathit{inv}}{\Post(\left\{c\right\}, i, e)}$ holds true. Consider a permutation $\incurly{i_1,i_2,\ldots,i_{\NMAX}}$ of robots executing events $\incurly{e_1, e_2, \ldots , e_{\NMAX}}$ respectively in a program transition during a round, reaching configurations $c_1, c_2, \ldots, c_{\NMAX}$ respectively. By assumption, since $\Post(\incurly{c},i,e) = c_1$, $\eec{\mathit{inv}}{c_1}$. Since our assumption was about any arbitrary (symbolic) configuration $c$, the same argument therefore can be extended to all the configurations reached by the system during this program transition, i.e. $\forall c' \in \incurly{c_1, c_2, \ldots, c_n}, \eec{\mathit{inv}}{c}$.
\end{proof}
\end{document}

