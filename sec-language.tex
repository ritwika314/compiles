\section{\lgname Language Design}
\label{sec:language}

\chiao{
\refsect{semantics} <= 2 pages,
\refsect{memory} <= 1 page,
\refsect{module} <= 0.5 page
}

\subsection{Formal syntax and semantics}
\label{sec:semantics}

The (distributed) system of robots  executing the \dmap application written in $\lgname$ can be viewed as a nondeterministic, state machine automaton~\cite{TIOAmon, Mitra07PhD}. The detailed statement-level semantics of $\lgname$ has been developed in the $\mathbb{K}$ framework and will appear elsewhere. For the purpose of the analysis here, we briefly describe the ``big-step'' semantics of \lgname which allows all the statements in an event of a single robot to be executed as an instantaneous transition.

Suppose we have $N$ robots in the system each executing the same \lgname application program. As mentioned earlier \lgname middleware enforces a synchronous \emph{round-by-round} execution, where each round lasts for $\delta$ time, where $\delta >0$ is a platform specific execution parameter.
%
In each round, each robot performs an  event.
The state update performed by a single robot executing an  event is modeled as an  transition of the state machine updating the program variables, however,  different events executed by the different robots may interleave. In between the events of successive rounds, the program variables remain constant, but {\em all\/} the \sayan{sensor and actuator variables} of all the robots  change continuously with time.  These are modeled as the $\delta$-transitions that advance time as well as the \sayan{sensor and actuator variables}.

%
Let the set of all program variables in the $\lgname$ application be $\Var$, and the set of all the sensor and actuator ports be $\mathit{Port}$. Let $\mathit{Event}_i$ be the set of events for the program for agent $i$.

%
The sets of variables and ports of an individual robot $i$ are denoted by $\Var_i$ and  $\mathit{Port}_i$.
%
A {\em valuation\/} of $\Var \cup \Port$ maps the names of the variables and ports to their types.
We denote valuations using boldface letter like $\vs, \vs_1, \vs'$, etc.
%
Given a valuation $\vs$, the value of a subset of  variables (or ports) $X$ is denoted by $\vs.X$. For example, the  valuation of all the variables of robot $i$ is $\vs.\Var_i$.
%
%
In the $\dmap$ $\lgname$ application,
\begin{inparaenum}
\item $\Var = \left\{ \gmap, \lmap, \mathit{newPoint}, \mathit{pscan}, \mathit{obstacles} \right\}$
\item $\Port = \left\{\mathit{ldata}, \mathit{trace}, \mathit{pos}, \mathit{trace}, \mathit{reached}, \mathit{target} \right\}$. \footnote{The syntax of $\lgname$ requires each \emph{actuator} and \emph{sensor} variable $v$ defined in a module $M$ to be referred to as $M.v$.}
\item For each robot $i$, $\mathit{Event}_i = \left\{ \mathit{NewPoint}, \mathit{LUpdate}, \mathit{GUpdate}\right\}$
\end{inparaenum}
%
%values that the variables can take is $\Val$, the \emph{state} of a robot $i$ is a tuple $$s_i = ( M, cp), \mbox{ where}$$
%
For the purpose of this paper, we define the  semantics of the overall system with $N$ robots as discrete transition system $\A = (C, C_0, A, \D)$,
where
(1) $C$ is the set of valuations of all the variables $\Var$ and all the ports $\mathit{Port}$ for each robot,
(2) $C_0 \subseteq Q$ is the set of initial values of the variables and port for each robot. All variables and ports have to be initialized,
(3) $A = \{\delta\} \cup \left( \cup_{i \in [N]} \mathit{Events}_i \right)$ is the set of transition labels, and
(4) $D \subseteq C \times A \times C$ is a set of transitions. The state of the system is a valuation $\vs \in C$. There are two types of transitions, namely:
\begin{enumerate}
	\item {\em event transition}: Any $a \in \Event_i$ for some $i \in [N]$, and  given $\vs, \vs'\in C$,   $(\vs,a,\vs') \in \D$ iff (i) $\vs'.\Var_i$ is obtained by applying the transition relation of $a$ to $\vs.\Var_i$, where $\vs.\Var_i$ refers to the mapping of all variables of robot $i$ to their values in the state $\vs$.
		(ii) For all $j \neq i$,
		$\vs'.\Var_j = \vs.\Var_j$.
	\item {\em $\delta$-transition}: $a = \delta$, and $(\vs,a,\vs') \in \D$ iff $\vs'.\Var = \vs.\Var$.
\end{enumerate}
\rg{We assume that the event transitions are computational steps, and take zero logical time, whereas the $\delta$-transitions denote the evolution of the sensor and the actuator ports with time, and take $\delta$ time.}\fTBD{restate in assumptions}.
\sayan{This formalization allows the ports to behave arbitrarily over $\delta$-transitions. In order to establish invariant properties of applications, therefore, we will need to constrain the behavior of the sensor and actuator ports with additional assumptions.
}


A \emph{program step} is a sequence  $\vs_0 , a_0,  \vs_1, a_1, \ldots  ,a_N \vs_N$ of length $N$, where $(\vs_i , a , \vs_{i+1}) \in D$ is an event transition. Given a program step $\ps = \vs_0 , a_0,  \vs_1, a_1, \ldots  ,a_N \vs_N$, $\ps[i]$ refers to the state $\vs_i$.
An \emph{execution} of a system is an alternating sequence of program steps and $\delta$-transitions $\ps_0, d_0 , \ps_1, d_1, \ldots$ where
\begin{itemize}
\item $\ps_i$ is a program step and $d_i$ is a $\delta$-transition,
\item $\ps_0$ is a program step $\vs_0, a_0, \vs_1, \ldots ,a_N, \vs_N $  where $\vs_0 \in C_0$,
\item Given a $\delta$-transition $d_i = (\vs, a, \vs')$, and its preceding program step $\ps_i$,  $\ps_i[N] = \vs $
\end{itemize}
%
%of each robot evolve to the 	controllers driving the robot according to the actuators that may have been set in the event. The program and sensor variables of the robot can be seen as the state variables for each robot.


%\begin{enumerate}
%    \item $M : \Var \mapsto \Val$ is the mapping of all program variables in the application known by the robot to their values.
%    \item $cp : \Var \mapsto \Val$ is the mapping of actuator and sensor variables to their values.
%\end{enumerate}
%The components of the tuple $s_i$ are accessed using the dot $`.'$ notation, for instance, $s_i.M$, etc. The value of a program variable $v$ at a state $s_i$ is denoted by $s_i.M.v$.

We define the set of reachable states of a system $\A$ , $\Reach(\A) = \{c \mathit c \in C_0 \vee \exists c^\prime \in \Reach(\A), (c^\prime, a, c) \in \D \}$. A predicate $\mathcal{P} : C \mapsto \mathbb{B}$ is an \emph{invariant} of the $\A$, if $\forall q \in \Reach(\A), \mathcal{P} = \mathit{True}$.


\subsection{Synchronous Distributed Shared Memory}
\label{sec:memory}

\chiao{
Precisely describe how this can be achieved by synchronous communication given a $\delta$ time bound.
Discuss why this abstraction over message passing with time delays is acceptable/reasonable}

\subsection{Specify Constraints over Modules}
\label{sec:module}
\chiao{As mentioned in \refsect{semantics}, the assumptions over modules will be directed by the target invariant/property.
What does \lgname provide to help identify the assumptions?
}