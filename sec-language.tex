\section{\lgname language design}
\label{sec:language}

In this section, we present the syntax and the semantics of \lgname.
%
%\subsection{Synchronous run-time system }
%
When a \lgname application is deployed on a fleet of robots, multiple instances of the same program runs on a collection of $\NMAX$ robots.
There is a known set of identifiers $\UINS=\{0,1,\dots,\NMAX-1\}$, and each robot is assigned  a unique index $\myuin \in \UINS$.
% Sayan. This is out of place here.
%Programmers can use $\NMAX$ and $\myuin$ as constant values in a \lgname program.
%
The robots communicate with each other through distributed shared memory. 
The execution of the \lgname program advances in a synchronous, \emph{round-by-round} fashion, where each round lasts for $\delta$ time,
and $\delta >0$ is a platform specific execution parameter.
During this period, the robots compute, move, and 
communicate with each other through distributed shared memory. 



\subsection{Syntax}\label{sec:syntax}

\reffig{partial-syntax} shows the partial grammar of \lgname syntax.
Each robot program consists of
\begin{inparaenum}[(a)]
\item declarations of the interfaces between the program and the sensor/actuator modules,
\item declarations of shared and local program variables, and
\item events, consisting of preconditions and effects.
\end{inparaenum}
Robot programs (rule \emph{Program}) first can import sensor/actuator modules.
The module import grammar production specifies the interfaces or ports:
it contains all input and output ports for actuators (\emph{APorts}) and sensors (\emph{SPorts}) that the program uses.
%The program can also contain shared and local declarations.
To summarize, there are following three types of names for reading/writing values:
\begin{enumerate}[(i)]
\item \emph{Sensor and actuator ports} are used to read from sensor ports and write to actuator ports of controllers.
\item \emph{Local program variables} record the state of the program.
\item \emph{Distributed shared variables} are used for coordination across robots. All shared variables can be read by all participating robots; an
      \textbf{allwrite} variable can be written by any participating robot; while an 
      \textbf{allread} variable can be written only by a single-writer.
\end{enumerate}

User can then optionally specify a statement to set the initial values of program variables~(rule~\emph{Init}).
The main body of the program is a sequence of events (rule~\emph{Event}) which include a Boolean \textbf{pre}condition
and an \textbf{eff}ect.
The effect of an event is also a statement~(rule~\emph{Effect}).
We skip the syntax rules for statements, expressions, data types, and functions due to the page limit.
A statement~(rule~\emph{Stmt}) in \lgname resembles those in most imperative languages and includes
conditional statements, function calls, assignments, blocks of statements, atomic statements for mutual exclusion, etc.
Mutual exclusion is always an essential feature when shared variables are involved.
\lgname provides a locking mechanism using the keyword \textbf{atomic} to update the shared variable safely.
The user can also define functions and abstract data types (tuples of the basic data types).

In the syntax presented in \reffig{partial-syntax}, given an nonterminal \textit{NT},
{\it NT\textsuperscript{?}} means that it is optional in the syntax at that position,
{\it NT*} refers to zero or more occurrences,
and {\it NT\textsuperscript{+}} refers to one or more occurrences.
The expression $(E1\mid E2)$ denotes that one can use either $E1$ or $E2$.

\begin{figure}

\newcommand{\zeroone}{\textsuperscript{?}\ }
\newcommand{\zeromore}{*\ }
\newcommand{\onemore}{\textsuperscript{+}\ }
\newcommand{\vbar}{{\normalfont\ |\ }}
\newcommand{\mterm}[1]{{\normalfont \textbf{#1}}}
\newcommand{\delim}{\mterm{:\ }\xspace}

\itshape
\begin{tabular}{lrl}
    Program   & ::= & Def\zeromore Import\zeromore DeclBlock Init\zeroone Event\onemore \\
    Def       & ::= & TypeDef \vbar FuncDef                                             \\
    Import    & ::= & \mterm{using} identifier \delim SPorts APorts                     \\
    SPorts    & ::= & \mterm{sensors} \delim VarDecl\onemore                            \\
    APorts    & ::= & \mterm{actuators} \delim VarDecl\onemore                          \\
              &     &                                                                   \\
    DeclBlock & ::= & AWDecl\zeroone ARDecl\zeroone LocalDecl\zeroone                   \\
    AWDecl    & ::= & \mterm{allwrite} \delim VarDecl\onemore                           \\
    ARDecl    & ::= & \mterm{allread} \delim ARVarDecl\onemore                          \\
    LocalDecl & ::= & \mterm{local} \delim VarDecl\onemore                              \\
    VarDecl   & ::= & Type identifier \vbar Type identifier \mterm{=} Val               \\
    ARVarDecl & ::= & Type identifier \mterm{[ $\NMAX$ ]}                               \\
              &     &                                                                   \\
    Init      & ::= & \mterm{init} \delim Stmt                                          \\
    Event     & ::= & identifier \delim Precond Effect                                  \\
    Precond   & ::= & \mterm{pre} \delim Expr                                           \\
    Effect    & ::= & \mterm{eff} \delim Stmt                                           \\
              &     &                                                                   \\
    Stmt      & ::= & \mterm{atomic} Stmt \vbar Stmt StmtList \vbar \textellipsis
\end{tabular}

\caption{Partial \lgname syntax rules.}\label{fig:partial-syntax}
\end{figure}


\subsection{Configurations}
\label{sec:configs}

The semantics of a \lgname program execution is based on synchronous rounds divided into \emph{event transitions} and \emph{environment transitions} that update the \emph{system configuration}.
In each round, each robot performs at most one event.
The update performed by a single robot executing an event is modeled as an instantaneous transition that updates the program variables; however, different events executed by the different robots may interleave in an arbitrary sequence.
In between the events of successive rounds, $\delta>0$ duration of time elapses, the program variables remain constant while the values held by the sensor and actuator ports may change. 
%This is captured by an {\em environment transition\/}.
%The \lgname program can only read the sampled values from and write to the ports every $\delta$ time.
%\sayan{Need to choose the words carefully here. We should discuss. Whether the variables change value continuously or discretely is a modeling question and what we want to analyze.}
These are modeled as environment transitions that advance time as well as the sensor and actuator ports.
%
Thus, each round consists of a burst of (at most \NMAX) event transitions followed by an environment transition. This is a standard model for synchronous distributed systems where the speed of computation is much faster than the speed of communication and physical movement~\cite{lynch1996a,attiyawelch}. 

We now describe the system state, or \emph{system configurations} which we use to formalize \lgname semantics.

\paragraph{System configurations.}

A \emph{system configuration} is a tuple $\gconfig = (\lset_{i\in\UINS},{S},\tau,\turn)$, where

\begin{enumerate}[(i)]
\item $\lset_{i\in\UINS}$ or $\lset$ in short is an indexed set of \emph{robot configurations}--one for each participating robot.
      $\lconfig{i}$ refers to the configuration of the $i$-th element, i.e., the $i$-th robot in the system.
\item ${S} : \Var \mapsto \Val$ is the {\em global context\/}, mapping all shared variable names to their values.
\item $\tau\in \nnreals$ is the {\em global time\/}.
\item $\turn\in\{\prog,\env\}$ is a binary \emph{bookkeeping} variable determining whether program or environment transitions are being processed.
\end{enumerate}

Bookkeeping variables are invisible in the language syntax, and only used in the semantics.


\paragraph{Robot configurations.}

A \emph{robot configurations} is used to specify the semantics of each robot.
As a \lgname program is run on a system of robots,
each participating robot would have its own set of module ports and local variables,
along with a local copy of each shared variable.
Given a \lgname program $P$, we can define $\Var$ be the set of variables,
$\Val$ be the set of values that an expression in \lgname can evaluate to,
$\Cfield$ be the set of sensor and actuator ports of the controller being used,
and $\Event$ the set of events in $P$.
%
\marginpar{\scriptsize\sayan{Cfield is an odd name. All of the above has to be defined after we fix a Koord program $P$.}}
%
The configuration of an robot is a tuple $\agnt = ({M},\cp,\turn)$, where

\begin{enumerate}[(i)]
\item ${M} : \Var \mapsto \Val$ is its \emph{local context} mapping both local and shared variables to values.
      Note that this implies $M$ includes a copy of shared variable values.
\item $\cp : \Cfield \mapsto \Val$ is the mapping of sensor and actuator ports to values.
\item $\turn \in \{\prog,\env\}$ is a bookkeeping variable indicating whether this robot should be executing a program or environment transition.
\end{enumerate}
For readability, we use the dot (``$.$'') notation to access components of a robot configuration $\agnt$.
For example, $\agnt.M$ means accessing the local context $M$ in the tuple $\agnt$.

\paragraph{Black-box functions for environment transitions}
To define the  executable \K semantics of  \lgname applications, we have to provide executable descriptions for the environment transitions. The type of this executable object ($f$) is defined by $\Cfield$, namely, 
$f: [\Cfield \mapsto \Val] \times \nnreals \mapsto [\Cfield \mapsto \Val]$.
That is, given old sensor and actuator values and a time point, $f$ should return the new values for all sensor and actual ports.
%
Depending on whether  we have an explicit or a black-box model for $f$,
the executable semantics will enable different types of analysis as we shall see in \refsect{verification}.


\subsection{Semantics}\label{sec:semantics}

%%semantic rule references
\newcommand{\StmtSeqRuleOne}{\textsc{StmtSeq1}\xspace}
\newcommand{\StmtSeqRuleTwo}{\textsc{StmtSeq2}\xspace}
\newcommand{\SelectEventRule}{\textsc{SelectEvent}\xspace}
\newcommand{\SkipEventRule}{\textsc{SkipEvent}\xspace}
\newcommand{\EndEventRule}{\textsc{EndEvent}\xspace}
\newcommand{\EventTransRule}{\textsc{EventTrans}\xspace}
\newcommand{\EndProgTransRule}{\textsc{EndProgTrans}\xspace}
\newcommand{\RobotEnvRule}{\textsc{RobotEnv}\xspace}
\newcommand{\EnvTransRule}{\textsc{EnvTrans}\xspace}
%%transition rules
\newcommand{\exprule}{\rightarrow_E\xspace}
\newcommand{\stmtrule}{\rightarrow_\mathit{stmt}\xspace}
\newcommand{\sysrule}{\rightarrow_\mathit{Env}\xspace}
\newcommand{\progtrans}{\rightarrow_\mathit{prog}\xspace}
\newcommand{\envtrans}{\rightarrow_\mathit{env}\xspace}
%%
\newcommand{\SelectEvent}{\ensuremath{\mathtt{\oplus}}\xspace}
\newcommand{\EndEvent}{\ensuremath{\mathtt{\boldsymbol{\cdot}}}\xspace}

\newcommand{\lsetp}{\{\lconfig{i}^\prime\}}
\newcommand{\lsetpp}{\{\lconfig{i}^{\prime\prime}\}}
\begin{figure}
\scriptsize
\begin{mathpar}
    \mprset{flushleft}
    \inferrule*[Right=\StmtSeqRuleOne]
    {\langle{S}, \agnt, St \rangle \stmtrule  \langle{S'}, \agnt', St' \rangle}
    {\langle{S}, \agnt, St\ StList \rangle \stmtrule  \langle{S'}, \agnt', St'\ StList \rangle }

    \inferrule*[Right=\StmtSeqRuleTwo]
    {}
    {\langle{S}, \agnt, \EndEvent\ StList \rangle \stmtrule  \langle{S}, \agnt, StList \rangle }

    \inferrule*[leftskip=1.5cm, Right=\SelectEventRule]
    {
        \agnt.\turn = \prog
        \land ``\mathit{Name}\textbf{: pre: } \mathit{Cond} \textbf{ eff: }\mathit{Body}" \in \Event
        \land \mathit{\EvalExpr{\mathit{Cond}}}
    }
    {\langle{S}, \agnt, \SelectEvent \rangle \stmtrule  \langle{S}, \agnt, \mathit{Body} \rangle }
    \\

    \inferrule*[Right=\SkipEventRule]
    {}
    {\langle{S}, \agnt, \SelectEvent \rangle \stmtrule  \langle{S}, \agnt, \EndEvent \rangle }    \\

    \inferrule*[Right=\EndEventRule]
    {}
    {\langle{S},({M},\cp,\prog), \EndEvent \rangle \stmtrule  \langle{S},({M},\cp,\env), \EndEvent \rangle }

    \inferrule*[Right=\EventTransRule]
    {
        \exists i \in \UINS, \langle S,\lconfig{i}, \SelectEvent \rangle \stmtrule
        \langle S^{\prime},\lconfig{i}', \EndEvent\rangle \\\\
        \land\ \lconfig{i}.\turn = \prog \land \lconfig{i}^\prime.\turn = \env
    }
    {(\lset, S, \tau, \prog)\rightarrow_G (\lsetp, S^{\prime}, \tau, \prog)}

    \inferrule*[Right=\EndProgTransRule]
    {
        \forall i \in\UINS, \lconfig{i}.\turn=\env
    }
    {(\lset, S, \tau, \prog)\rightarrow_G (\lset, S, \tau, \env)}
    \\

    \inferrule*[leftskip=1cm, Right=\RobotEnvRule]
    {
        \forall x \in \mathit{Keys}({S}),M' = {M}[x \mapsto S[x]] \land \cp' = \mathit{f}(\cp,\delta)
    }
    {  \langle S, (M, cp, \env) \rangle \envtrans \langle S, (M', cp', \prog) \rangle}

    \inferrule*[Right=\EnvTransRule]{
        \forall i \in \UINS, \langle S, \lconfig{i} \rangle \envtrans \langle S, \lconfig{i}^\prime \rangle \\\\
        \land \lconfig{i}.\turn = \env \land \lconfig{i}^\prime.\turn = \prog
    }
    { (\lset, {S}, \tau,\env)\rightarrow_G (\lsetp, {S}, \tau + \delta,\prog)}

%    \inferrule*[Right=\RobotEnvToProgRule]
%    {  \forall i \in \UINS, \lconfig{i}.\turn = \env \land \lconfig{i}^\prime.\turn = \prog }
%    {  (\lset, S, \tau, \env)\rightarrow_G (\lsetp, S, \tau, \env)}
%
%    \inferrule*[Right=\EnvToProgRule]
%    {  \forall i \in \UINS, \lconfig{i}.\turn = \prog }
%    {  (\lset, S,\tau, \env)\rightarrow_G (\lset, S, \tau, \prog)}
\end{mathpar}
\caption{Partial semantic rules for \lgname.}\label{fig:partial-semantics}
\end{figure}

%
We will describe only the interesting semantic rules of \lgname  above the event level.
Rule~\StmtSeqRuleOne and \StmtSeqRuleTwo show how a statement representing a sequence of statements should be executed (for more details on statement and expression level semantics we refer the readers to the extended version of the paper~\cite{}).
%

\paragraph{Per robot semantics.}
First, we present the semantics of executing an event for each robot,
which will help us discuss the semantics of the whole system.
All rules for statement semantics are of type
\[
\stmtrule \subseteq (\pws \times \pwl\times (\pwstmt \cup \{\SelectEvent, \EndEvent\})) \mapsto \wp(\pws\times \pwl \times \pwstmt\cup \{\EndEvent\}),
\]
where $\pwstmt$ refers to the set of all possible statements allowed by \lgname syntax.
This relation takes as input a tuple of (1) a global context, (2) an robot configuration, and (3) a statement,
and maps it to a set of such tuples.
The symbols `\SelectEvent' and `\EndEvent' are not in \lgname but internal syntactic structures.
`\SelectEvent' is to denote nondeterministic selection of events,
and `\EndEvent' is to indicate an ``empty" statement.

Rule~\SelectEventRule in \reffig{partial-semantics} shows that any event may be executed when the precondition $Cond$ is evaluated to true,
and by replacing \SelectEvent with the event effect $\mathit{Body}$, it ensures only one event is selected and executed.
The event effect is then executed following the semantics of each statement in $\mathit{Body}$.
Rule~\SkipEventRule allows the robot to skip the event completely.
At the end of the event, the sequence of statements becomes empty~`\EndEvent'.
Rule~\EndEventRule then makes sure the $\turn$ of the robot is set to $\env$ indicating that
an environment transition will occur afterwards.

Similarly, we define the semantics of how each robot interacts with environment including other robots.
The environment transition rule is of type
\[
\envtrans \subseteq (\pws \times \pwl) \mapsto \wp(\pws\times \pwl),
\]
which takes a global context and a robot configuration as input.
Rule~\RobotEnvRule simply states that the new local context $M'$ is
old local context $M$ updated with the global context $S$;
thus ensuring that all robots have consistent shared variable values before the next program transition.
New sensor readings $\cp'$ is then obtained by evaluating the black-box dynamics $f$ with time $\delta$.
In an actual execution, the controller would run the program on hardware,
whose sensor ports evolve for $\delta$ time between program transitions.
This formalization allows the ports to behave arbitrarily over $\delta$-transitions.
In verification of invariant properties,
additional assumptions will be needed to constrain the behavior of the sensor and actuator ports.
Finally, the $\turn$ of the robot is set back to $\prog$.


\paragraph{Global semantics.}

With the event semantic for each robot, we can then define the execution for the distributed \lgname program.
The rewrite rule is a mapping from an initial system configuration to a set of configurations.
It has the type
\[
\rightarrow_G\ \subseteq \pwg \mapsto \wp(\pwg),
\]
where $\pwg$ is the set of all possible global configurations.

Rule~\EventTransRule expresses that starting from a global configuration $\gconfig = (\lset, S, \tau, \prog)$,
a robot $i$ with the configuration $\lconfig{i}$ starts by selecting an enabled event,
executes the event via a sequence of $\stmtrule$ rewrites,
and sets its own $\turn$ to $\env$ at the end of the event execution.
The system goes from a configuration $\gconfig$ to $\gconfig^{\prime}= (\lsetp, S^{\prime}, \tau, \prog)$,
with possibly different robot configurations and global context depending on
whether any statement executed resulted in writes to shared variables.
The system can display nondeterministic behaviors arising from different robots executing their events in different orders.
After all robots enter the $\env$ turn, rule~\EndProgTransRule sets the global $\turn$ from $\prog$ to $\env$
indicating the end of program transition, and an environment transition will occur afterwards.

Rule~\EnvTransRule shows the semantics of the system configuration after rule~\EndProgTransRule.
This rule synchronizes the environment transitions of each robot and
ensure that the global time $\tau$ advances to $\tau + \delta$.


\subsection{Synchronization and Consistency}
\label{sec:sync}

Following our semantic rules in \refsect{semantics},
careful readers would notice that all event transitions of \lgname program takes \emph{zero} time.
The environment transitions however take $\delta$ time for the evolution of the sensor and actuator ports
together with the update of the local context from the global context.
In this section, we discuss how this semantic abstracts the behavior of a distributed cyberphysical system,
and how our tool chain in~\cite{ghosh2019cyphyhouse} provides a faithful implementation of such abstraction.

To reiterate, the following are the timing requirements for rule~\EventTransRule and \EnvTransRule:
\begin{inparaenum}[(a)]
\item an event transition takes \emph{zero} time,
\item new values of controller ports are sampled at the end of each round
\item shared variables should reach consistent values within $\delta$ time, and
\item a global clock is used to synchronize each $\delta$-time round.
\end{inparaenum}
The first two requirements are achievable if the time taken to complete a program transition is negligible compared to $\delta$,
and $\delta$ can be a common multiple of the sampling intervals of all controller ports in use.
These constraints are reasonable for cyberphysical systems where the computation and communication is comparatively much faster.
For \Motion module as an example, our position sensor on each device publishes every 0.01 sec~(100Hz) while the CPU on each drone is 1.4 GHz.
If we set $\delta$ to be 0.01 sec, an event transition taking 10K CPU cycles is still less than 0.1\% of $\delta$.

The remaining requirements are common research questions in distributed computing with an extensive literature.
A global clock can be achieved with existing techniques that synchronize all local clocks on robots.
In~\cite{ghosh2019cyphyhouse}, we use message passing to implement distributed shared memory for shared variables.
We ensure that the time taken to propagate values thru message passing and reach consistency is smaller than $\delta$,
and the update is then  visible in the next round of program transitions for all robots.
We therefore conclude our round based semantic with shared memory is a reasonable abstraction.
