


\section{$\lgname$ Software Stack}
\label{sec:software}

\subsection{Runtime system}



To run a $\lgname$ program (hardware or simulation), the user has to provide a configuration file, with 
\begin{inparaenum}
\item the number of agents, 
\item in case of simulation, the initial positions of the agents and the length of the simulation and 
\item in case of hardware deployment their IP addresses, 
and the localization system.
\end{inparaenum} 

\subsection{Key environment assumptions} 


\subsubsection{Periodic event execution semantics}


\subsubsection{Shared variable implementation over message passing}


\subsubsection{Known set of participants}
\subsubsection{Portability and heterogeneity}


\subsection{Simulator}
\subsubsection{gazebo environment}
\subsubsection{car model}
\subsubsection{lidar}
\subsubsection{positioning}
\subsubsection{sampled sensing}
\subsubsection{synchronization issues}
 
\subsection{The Distributed Mapping Algorithm}
\subsubsection{Motivation}
In this section, we introduce the distributed mapping problem that the $\lgname$ program shown in \reffig{mapapp} aims to solve.
Informally, the problem requires a set of robots to collaboratively agree on positioning of static \emph{obstacles} within a given area $D$, which any robot should avoid while moving in $D$. 


\rg{Domain definition}
The mapping problem is defined over a (\emph{finite}) domain $D$, where $$D  = \left\{ (x,y) \mid \exists  x_1, x_2 ,y_1, y_2 \in \mathbb{R} , x_1 \leq x \leq x_2 \wedge y_1 \leq y \leq y_2\}\right$$

\rg{grid partitioning definition}
\rg{Needs to be refined:} One can define a \emph{grid} partitioning $G:D \mapsto \mathbb{Z}\times\mathbb{Z}$ which maps each point $(x,y)$ in $D$, to a \emph{square} $(i,j), i \in Z, j\in Z$. Given a domain $D\mathbb{R}^2$, $G(D)$ refers to the set of $(i,j)$ squares that comprise all $(x,y)\in D$. This can be viewed as an embedding of a 2D-lattice in $\mathbb{R}^2$. Consider also, that there is a set of robots $S$ in which each robot can can detect obstacles at a minimum distance $d$, within a range of angles $\left[-\phi,\phi\right]$ from the current heading of the robot. We can then state the 2-d distributed mapping problem as follows. 

\rg{problem statement}
\begin{center}
\emph{
 Given a set of robots $S$, grid partitioning $G$ of a 2-d domain $D\subset \mathbb{R^2}$, construct a mapping $F: G(D) \mapsto \left\{-1,0,1\right\}$ where $F(i,j) = -1$ represents that a grid square $(i,j)\in G(D)$ is at least partially occupied by an obstacle, $F(i,j) =0$ represents $(i,j)$ is completely free of any obstacles, and $F(i,j) = -1$ represents that it is unknown whether an obstacle occupies the grid square $(i,j)$.} 
\end{center}
We also assume that initially, each robot starts at a position such that the 8 surrounding squares are all unoccupied by obstacles. 


\subsection{Library Functions}


\subsubsection{Correctness and Soundness}
\rg{define notion of soundness}
\emph{Theorem1} State correctness and soundness condition with assumptions, proof for ~\reffig{mapapp}. 

